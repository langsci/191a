\chapter{Introduction} 
\section{Niger-Congo: the state of research and the prospects for reconstruction} 

It is quite predictable that the title of this book may be met with skepticism by specialists in the comparative-historical studies of African languages. The first question that may arise is whether a Niger-Congo (NC) reconstruction is achievable at all, considered that the reconstruction of proto-languages underlying particular families and their branches has not been completed (or even properly started, as is the case for some groups and branches of NC). Before we turn to the structure of the book, let us try to answer this fundamental question. To do so, it seems reasonable to very briefly outline the present state of affairs in NC comparative studies. 

First, it should be noted that presently there is no general scientific discipline such as “NC comparative studies”. Instead, there are individual researchers who work on particular families, groups, sub-groups or branches of NC. Among these, comparative-historical Bantu studies has flourished the most. However, the Bantu languages comprise only a branch of the Southern Bantoid languages that (together with Northern Bantoid) go back to Proto-Bantoid\il{Proto-Bantoid!}. Hence Bantu is merely one of 16--17 Bantoid branches, as can be gleaned from the chart below (\tabref{tab:0.1}).\footnote{This book does not investigate the genealogical classification of Niger-Congo as a whole, nor of the individual families of this macro-family. The schemes presented here take into account the most well-known classifications (sometimes with small deviations due to the specific purposes of our study). The scheme of Bantoid languages ​​given here is based mainly on the classification in \url{https://mpi-lingweb.shh.mpg.de/numeral/Niger-Congo-Benue-Congo.htm}. It generally reproduces John Watters' classification (\citeyear{Watters1989}: 401) with some deviations, which are not considered here.}

 
\begin{table}
\caption{Bantoid languages}


\label{tab:0.1}
\begin{tabularx}{\textwidth}{Qllll}
% \lsptoprule
\rowcolor{black!10!white}
Northern Bantoid: & Dakoid & Mambiloid & Fam\il{Fam} & Tiba\il{Tiba} (F{\`{a}})\\
\rowcolor{black!20!white}
Southern Bantoid: & \textbf{Bantu} & Beboid & Yemne-Kimbi & Ekoid\\
\rowcolor{black!20!white}
~ & Jarawan & Mamfe & Mbam & Mbe\il{Mbe}\\
\rowcolor{black!20!white}
~ & Ndemli\il{Ndemli} & Tikar\il{Tikar} & Tivoid & Wide Grassfields\\
% \lspbottomrule
\end{tabularx}
\end{table}

The progress of comparative-historical studies of the Bantoid languages has been less impressive than that of Bantu studies. Proto-Bantoid\il{Proto-Bantoid}, as well as a number of other proto-languages, goes back to the Proto-Eastern-Benue-Congo\il{Proto-Eastern-Benue-Congo}. In turn, the latter (along with Proto-Western-Benue-Congo\il{Proto-Western-Benue-Congo} and possibly some other languages that do not belong to these two major groups of Benue-Congo) goes back to Proto-Benue-Congo\il{Proto-Benue-Congo} (BC). Hence, the Bantoid branch is merely one of 14--15 branches of Benue-Congo, as demonstrated by the chart below (\tabref{tab:0.2}).

 
\begin{table}[b]
\begin{tabularx}{.8\textwidth}{XXl}
\lsptoprule
*Western BC & *Eastern BC & Isolated BC\\
\midrule 
Nupoid & Kainji & Oko\il{Oko}\\
Defoid & Platoid & Akpes\il{Akpes} \\
Edoid & Cross & Ikaan\il{Ikaan}\\
Igboid & Jukunoid & Lufu\il{Lufu}\\
Idomoid & \textbf{Bantoid} & \\
\lspbottomrule
\end{tabularx}
\caption{Benue-Congo languages\\
{The inventory of Benue-Congo groups is given mainly by \citealt{Williamson1989b}: 266--269. The main difference in \tabref{tab:0.2} is that Jukunoid is separated from Platoid, which allows us to better compare the forms of numerals of these groups, as well as the fact that Lufu has been added to isolated languages. The division of the BC into the Western and Eastern branches does not always reflect the genealogical characteristics of languages.}}
\label{tab:0.2}
\end{table}

The traditional reconstruction of Proto-BC based on regular correspondences between the proto-languages underlying the separate branches listed in \tabref{tab:0.2} has developed rapidly in recent years. However (and I hope that my colleagues will take no offence at this statement), despite numerous brilliant studies dealing with the subject, this is still a relatively ‘young’ science.

Finally, in addition to Proto-BC there are probably more than ten proto-lan\-guages underlying other language families that together comprise the Niger-Congo macrofamily (see \tabref{tab:0.3}). 

 
\begin{table}
\caption{Niger-Congo languages\\
{The grouping of 12 families of NC into 5 geographical zones is convenient for technical purposes of generalization of data. So, it means nothing else. As for a genealogical tree of NC languages, as of today there are insufficient grounds for creating one, in my opinion.}
}
\label{tab:0.3}
\begin{tikzpicture}
\node[minimum width=2.2cm, minimum height=.7cm, inner sep=3pt, draw, fill=lsLightBlue] (mel)   {Mel\vphantom{j}};
\node[minimum width=2.2cm, minimum height=.7cm, inner sep=3pt, draw, fill=lsLightBlue] (atlantic) [above=of mel,yshift=-1cm] {Atlantic\vphantom{j}}; 
 
\node[minimum width=2.2cm, minimum height=.7cm, inner sep=3pt, draw, fill=lsMidDarkBlue] (kru) [right=of mel,xshift=-1cm]  {Kru\vphantom{j}};
\node[minimum width=2.2cm, minimum height=.7cm, inner sep=3pt, draw, fill=white] (mande) [above=of kru,yshift=-1cm] {Mande\vphantom{j}}; 
                                                                                
\node[minimum width=2.2cm, minimum height=.7cm, inner sep=3pt, draw, fill=lsMidDarkBlue]   (kwa) [right=of kru,xshift=-1cm] {Kwa\vphantom{j}};
\node[minimum width=2.2cm, minimum height=.7cm, inner sep=3pt, draw, fill=white] (gur) [above=of kwa,yshift=-1cm] {Gur\vphantom{j}};
\node[minimum width=2.2cm, minimum height=.7cm, inner sep=3pt, draw, fill=white] (dogon) [above=of gur,yshift=-1cm] {Dogon};
                                         
\node[minimum width=2.2cm, minimum height=.7cm, inner sep=3pt, draw, fill=lsMidDarkBlue] (ijo) [right=of kwa,xshift=-1cm]  {Ijo};
\node[minimum width=2.2cm, minimum height=.7cm, inner sep=3pt, draw, fill=lsLightGray] (ubangi) [above=of ijo,yshift=-1cm] {Ubangi};                                                            

\node[minimum width=2.2cm, minimum height=.7cm, inner sep=3pt, draw, fill=lsMidDarkBlue] (bc) [right=of ijo,xshift=-1cm] {BC\vphantom{j}};
\node[minimum width=2.2cm, minimum height=.7cm, inner sep=3pt, draw, fill=lsLightGray] (adamawa) [above=of bc,yshift=-1cm] {Adamawa\vphantom{j}};
\node[minimum width=2.2cm, minimum height=.7cm, inner sep=3pt, draw, fill=lsSoftGreen] (kordofan) [above=of adamawa,yshift=-1cm] {Kordofan\vphantom{j}};  

% \node[minimum width=2.5cm, minimum height=.7cm, inner sep=3pt] (snc) [right=of adamawa,xshift=-.5cm] {{\bfseries\color{green}Southern NC}};       
% \node[minimum width=2.5cm, minimum height=.7cm, inner sep=3pt] (nnc) [above=of snc,yshift=-1cm]  {{\bfseries\color{brown}Northern NC}};
% \node[minimum width=2.5cm, minimum height=.7cm, inner sep=3pt] (enc) [above=of adamawa,yshift=-.2cm] {{\bfseries Eastern NC}};    
% \node[minimum width=2.5cm, minimum height=.7cm, inner sep=3pt] (wnc) [above=of atlantic,yshift=-.2cm] {{\bfseries\color{blue}Western NC}};                                                            
\end{tikzpicture}
\end{table}



Most of the works presently available in NC comparative studies do not reach beyond this point. Exceptions are rare, and examples of the comparative-his\-tor\-i\-cal approach to the NC reconstruction are few. Moreover, the most significant works of this kind (e.g. those of \citealt{Westermann1927}, \citealt{Greenberg1966}, \citealt{Sebeok1971}, etc.) are not that recent and usually date to the middle of the 20th century. Comparative studies of the African macro-families had a jump start but nearly had come to little by the end of the 20\textsuperscript{th} century (important works such as  \citealt{Bendor-Samuel1989} including \citealt{Williamson1988,Williamson1989a} are few in this period).

So, what happened? 

By the 1990s, our knowledge in the field of African languages had begun to grow exponentially. Hundreds of new language descriptions had been published, and the few dozen experts working in NC comparative linguistics were simply unable to digest this avalanche of new information. 

The main problem in the 1960s was that we knew too little. From the 1980s on, we have faced the opposite problem: we know “too much”. Not only do scholars not have enough time to absorb new results, sometimes they do not even have enough time to acquaint themselves with those results. During the last four decades, amidst this dialogue between linguistic knowledge and language data, African linguists have remained in listening mode. But I am convinced that the time has come for linguists to say something new again. Unlike even ten years ago, today we are well equipped to do so.

 
First, we have really exceptional databases. The best one is the RefLex data\-base elaborated by Guillaume Segerer \citep{RefLex}. It contains more than one million words from African languages (2017), and each entry contains a link to a PDF file of the corresponding source page. It provides a huge range of information and is maximally user-friendly to comparative linguists: it can be solicited for establishing regular phonetic correspondences, for reconstruction and for ranking reflexes as well as for various kinds of statistical data analysis. This new database is being constantly updated. 

A big database is something much more than just a huge amount of data. When a database  reaches certain degree of plenitude with respect to the main families and branches of the NC macro-family, it opens up prospects for both working with the distribution of words that do exist and with the distribution of \textbf{gaps} in postulated cognates. The distribution of filled cells and lacunes is a powerful tool allowing 1) identification of important innovations, 2) targeted searches for unusual phonetic reflexes, 3) detection of diachronic semantic changes and 4) refinement of genealogical classification.

In my opinion, the opportunity to rely on both the apparent cognates as well as on the missing reflexes of reconstructed prototypes in particular languages dramatically changes the approach to the reconstruction itself.

The following case may serve as an illustration to this statement. Suppose we need to assess one of Greenberg’s proposals, e.g. a Niger-Congo root meaning ‘hill’. Among the reflexes quoted by Greenberg for this root are: “(2) Busa\il{Busa} \textit{kpi} ‘mountain’, Kweni  \textit{kpi~}; (4) Gã\il{Gã} \textit{kpɔ~}; Gwa\il{Gwa} \textit{ogba} ‘mountain’; (5) Nungu\il{Nungu} \textit{agbɔ}, Ninzam (Ninzo\il{Ninzo}) \textit{igbu}. Kordofanian: (2) Tagoi\il{Tagoi} \textit{(c)ibe}.” \citep[155]{Greenberg1966}. The phonetic correspondences underlying the comparison of these forms will not be discussed here (we will just assume that they are valid), for the main problem is elsewhere. A reader with no access to a representative lexical database on the NC languages is always uncertain about a number of key issues, including:

\begin{enumerate}
\item whether the root in question is widely attested in the families and groups for which the author postulates the reflexes?
\item whether the root is present in other NC families and groups and how widely it is attested in them?
\item are there any other roots possibly interpretable as NC terms for ‘hill’?
\end{enumerate}
The RefLex database establishes  that: 

\begin{enumerate}
\item there are plenty of forms phonetically similar to those of Greenberg (cf. e.g. Boko\il{Boko} (in the same sub-group as Busa\il{Busa}) \textit{kpii} ‘mountain’, Gwari\il{Gwari} (Nupoid, BC) \textit{{\={o}}p{\'{e}}} ‘hill, mountain’, etc), but the postulated root is at best only marginally attested in the families where Greenberg finds it. 
\item The root is absent in other branches and families (even if the proposed phonetic correspondences are approached most liberally), although, if wished, its “reflexes” can be found in any of the NC families, cf. e.g. Ibani\il{Ibani} (Ijo) \textit{kp{\'{o}}kp{\'{o}}} ‘hill’, etc. 
\item Most importantly, several other roots with the meaning ‘hill, mountain’ are distinguishable in the NC languages. All of them (unlike the one proposed by Greenberg) are valid candidates for the reconstruction of the NC prototype. One of these roots is presented in the chart below (\tabref{tab:0.4}) (one could mention some other roots nearby):
\end{enumerate}


\begin{table}
\caption{\textit{*tʊnd} ‘hill, mountain’ in Niger-Congo} 
\label{tab:0.4}
\numcolcomplete{%
\numcolone{*tʊnd}{tul- ? }
}{%
\numcoltwo{*tinti, *ton}{tōɖō}
}{%
\numcolthree{tɔ́rɔ́}{~}{tu?}
}{%
\numcolfour{~}{tʊ́ndʊ́}
}{%
\numcolfive{~}{~}{tʊ̀ndà}
}{~}

\end{table}

The exact correspondence between Proto-Bantu\il{Proto-Bantu} (\textit{*t{\`{ʊ}}ndà}, zones HJKPMNRS > (~?) \textit{*d{\'{ʊ}}nd{\`{ʊ}}},  zones EGHJKLMNRS), Ijo (Ibani\il{Ibani} \textit{t{\'{ʊ}}nd{\'{ʊ}}}) and Atlantic languages (Atlantic Bak: Manjak\il{Manjak} \textit{ntʊnda}, Atlantic North: Basari\il{Basari} \textit{e-t{\'{ə}}nd}, Bapen\il{Bapen} \textit{ɛ{}-tʌnd}, Laal\il{Laal}a\il{Laala} \textit{tundə}, Fula\il{Fula} \textit{tulde}, Wolof\il{Wolof} \textit{tund}) is reason enough to postulate the root \textit{*tʊnd} ‘hill, mountain’ at the Proto-NC\il{Proto-NC} level, especially since these languages have apparently been out of direct contact.\footnote{We shall repeat that nearby there are some other candidates for ‘mountain’ in NC, which we do not treat here.} In addition, the absence of this root in Gur{}-Ubangi{}-Adamawa may prove to be a shared innovation in these languages.

Using the databases, the focus of our research could be redirected toward the basic meaning of the lexemes (rather than on the occasional phonetic similarities between the forms). This approach may help in answering the following question: if a Proto-NC\il{Proto-NC} term for ‘mountain, hill’ existed, how did it sound? The answer would probably be as follows: this word could sound like \textit{*tʊnd}, \textit{*kong/} \textit{keng} or \textit{*kudu} (‘hill, rock, stone’), but not like \textit{dima} (PB\il{PB} \textit{*d{\`{ɩ}}m{\`{a}}}, zone EGJ), \textit{mut} (Proto-Jukunoid\il{Proto-Jukunoid} \textit{*muT}) or \textit{pi} (PB \textit{p{\`{ɩ}}d{\`{ɩ}}}, zone KLMN). 

Upon arriving at these unconventional “results”, one could bring them to the attention of specialists in particular NC languages and branches for further evaluation. Without such professional evaluation there can be no hope for success.  Moreover, in recent years it has become evident that this evaluation needs to be collaborative (i.e. made by dozens of specialists working together) for the simple reason that today no specialist can be proficient in the languages of more than one or a maximum of two NC families. Hence, it is important that these specialists are asked questions they can answer, so ideally the approach outlined above should be applied to every family within Niger-Congo. For example, according to the etymological database of the Atlantic languages (\citealt{PozdniakovSegerer2017} 3700 cognates) only \textit{*t}\textit{ʊnd} and \textit{*th}\textit{əng} are potentially interpretable as the terms for ‘hill, mountain’ in Proto-Atlantic\il{Proto-Atlantic}.

Initially I thought of numerals as of an ideal group of terms to test this approach. On the one hand, the core group of numerals must have existed in Niger-Congo. On the other hand, they represent a relatively compact lexical-semantic group with minimum potential for semantic shifts. My initial question seemed simple: what is the most probable Proto-Niger-Congo\il{Proto-Niger-Congo} root for ‘two’? The term for ‘two’ (being the only numeral on the Swadesh list) is generally recognized as one of the most persistent numerals. Why not try reconstructing it on the basis of the NC evidence? It appeared, however, that such a reconstruction is beset with difficulties, so what was originally intended as an article turned into this very book. The structure of the book is described in the section below. As I hope to demonstrate, this structure is conditioned by specific issues encountered in the course of the reconstruction of NC numerals. 

\section{Sources and the monograph structure}
\largerpage
\subsection{Sources}
Numeral terms included in the majority of lexical sources hold a privileged position. The information pertaining to the Niger-Congo numerals is more than extensive, it is nearly exhaustive. In addition to the above-mentioned RefLex database by Segerer-Flavier which contains over 17,000 entries marked as “numeral” (state  {April 2017)}) a number of other databases with expansive coverage of the Niger-Congo languages are available. One of them is the “Numeral Systems of the World's Languages” database created by Eugene S. L. Chan and edited by Bernard Comrie (Chan) The data regarding the number systems of about 4,300 languages (with hundreds of the Niger-Congo languages among them) is incorporated into it. Two or even three sources (often unique) are accessible for some of the languages via this neatly organized and user-friendly database. Another universal database that provides numerical data is “Numerals 1 to 10 in over 5000 languages” by Rosenfelder. It was consulted to a somewhat lesser extent because it only includes evidence pertaining to the first ten numerals, for which a simplified transcription is used. Finally, a number of unpublished databases that incorporate the evidence of specific Niger-Congo families and groups  were consulted, e.g. the etymological databases of Atlantic (PozdniakovSegerer2017) and Mande (Valentin Vydrin). 

As a result, a total of 2,200 sources for Niger-Congo languages were used in this study. This raises the issue of references, since it is impossible to provide a complete list of sources for every NC language. The language index at the end of this book lists the nearly 1,000 languages cited. For these 1,000 languages, the main sources I used are indicated in Appendix E. The index of sources in Appendix E is structured according to the NC main families in alphabetical order. 

For each language, I provide not only the source(s) that can be found in the bibliography, but also the name of every contributor in Chan’s database [Chan]. The list of contributors is many pages long, but their names should be known, even if their data are unpublished. This is the least I can do to express my sincere gratitude to each of them.

\subsection{Monograph structure}
Noun class affixes are present in numerical terms in the majority of the Niger-Congo languages. Many forms that are considered primary at the synchronic level have frozen noun class affixes that are no longer productive. In such cases it is extremely difficult to distinguish the etymological root within a numerical term. Without it, however, both the comparison and reconstruction of roots is impossible. This is why the second chapter of this book is devoted to the study of various uses of noun class markers in numeral terms.

The third chapter  deals with the alignment by analogy in numeral systems. As in other languages, numerals represent a lexical-semantic group that is especially subject to alignment by analogy due to its closed structure, where words are associated in a paradigm. A textbook example is the term for ‘nine’, with Indo-European *\textbf{n}{}- irregularly reflected in Proto-Balto-Slavic\il{Proto-Balto-Slavic} as \textbf{d}{}- (Russian\il{Russian} \textit{dev’at’} ‘9’ instead of the expected \textit{*nev’at’}) by analogy with the term for ‘ten’ (Russian \textit{des’at’} ‘10’). This yielded a minimum pair \textit{dev’at’} \textit{{\textasciitilde} des’at’} that forms a “class of the upper numerals” within the first ten. Adjacent numerals may be alined with each other in the NC languages by a similar formal marker. Thus, no satisfactory etymology can be suggested for the forms attested in Mumuye\il{Mumuye} (Adamawa; \textit{ziti} ‘2’ {\textasciitilde} \textit{taːti} ‘3’ {\textasciitilde} \textit{d{\~{\`ɛ}}ːt{\`{i}}} ‘4’) without the analysis of alignment by analogy. The issues pertaining to both detection and analysis of such alignments are addressed in \chapref{sec:2}.

\chapref{sec:3} offers a step-by-step reconstruction of number systems of the proto-languages underlying each of the twelve major NC families, on the basis of the step-by-step-reconstruction of numerals within each family. The term “reconstruction” related to numerals throughout this book calls for a definition.  As mentioned above, the use of this term has been questioned, mainly because systems of regular phonetic correspondences between the languages within NC families remain unknown. This is why Kay Williamson opted for the term \textit{pseudo-reconstructions} (marked with \# instead of *): “Reconstructions proposed by their authors as based on regular sound correspondences are preceded by an asterisk. Pseudo-reconstructions based on a quick inspection of a cognate set without working out sound correspondences are proceded by a \#” \citep[251]{Williamson1989b}. In his numerous online publications Roger Blench uses \# as well, but his terminology is different: he prefers the more neutral term of \textit{quasi-reconstructions}. Modern comparative studies of the NC languages is a relatively young science, so the opposition between “real” and “pseudo-/quasi-” reconstructions seems irrelevant to me at this stage. The more so that nearly all of our reconstructions (maybe with the exception of Bantu and some other branches) should be marked with \#, including the large proportion of reconstructions allegedly based on the evidence of historical phonetics. On the other hand, I think that many colleagues would agree with the following statement: although we do not know the regular phonetic correspondences between the languages that belong to different NC families, there is hardly any doubt that the NC root for ‘three’ sounded something like \textit{tat}.

Throughout this book the term “step-by-step reconstruction of number systems” (e.g in the Atlantic family) is used in reference to the method that includes the following steps:

\begin{enumerate}
\item While comparing the forms of numerical terms attested in the languages under study, their most likely prototypes were established within both of the Atlantic groups, i.e. Northern (Proto-Tenda\il{Proto-Tenda}, Proto-Jaad-Biafada\il{Proto-Jaad-Biafada}, Proto-Fula-Sereer\il{Proto-Fula-Sereer}, Proto-Wolof\il{Proto-Wolof}, Proto-Cangin\il{Proto-Cangin}, Proto-Nalu-Baga Fore-Baga Mbo\-te\-ni\il{Proto-Nalu-Baga Fore-Baga Mboteni}) and Bak (Proto-Joola\il{Proto-Joola}-Bayot\il{Proto-Joola-Bayot}, Proto-Manjak-Mankanya-Pepel\il{Proto-Manjak-Mankanya-Pepel}, Proto-Balant\il{Proto-Balant}, Proto-Bijogo\il{Proto-Bijogo}).
\item On the basis of these prototypes, the most likely forms of Proto-Northern Atlantic\il{Proto-Northern Atlantic} and Proto-Bak\il{Proto-Bak} Atlantic numerals were suggested.
\item On the basis of these more ancient forms, the most plausible reconstruction of Proto-Atlantic\il{Proto-Atlantic} numerals was offered.
\end{enumerate}

\chapref{sec:4} deals with the reconstruction of the Proto-Niger-Congo\il{Proto-Niger-Congo} numeral system on the basis of the step-by-step-reconstructions offered in \chapref{sec:3} for each of the twelve major families and a handful of isolates.  
The reconstruction described in \chapref{sec:4} inspired the analysis of the distribution of reflexes of the NC proto-forms within each of the twelve families (as well as within the isolates) in order to establish:

\begin{enumerate}
 \item the most archaic NC families / groups / branches (i.e. those that preserve the inventory of Proto-NC\il{Proto-NC} forms most fully); 
 \item NC families / groups / branches that are the most distant from Proto-Niger-Congo\il{Proto-Niger-Congo} in what pertains to the reflection of numerals. 
\end{enumerate}

 
The results of this analysis are presented in \chapref{sec:5}.

To illustrate the logic of the complex structure of the monograph, let us consider one example.\\

In \chapref{sec:3}, along with other NC families, the numerals of the Atlantic languages are analyzed (\sectref{sec:3.12}). Atlantic languages are divided into two main groups – North Atlantic (\sectref{sec:3.12.1}) and Bak Atlantic (\sectref{sec:3.12.2}).

In Sections \sectref{sec:3.12.1.1}–\sectref{sec:3.12.1.7}, systems of numerals are considered consecutively in the seven main subgroups of the North Atlantic languages. In particular, in \sectref{sec:3.12.1.3}, numerals in the Jaad\il{Jaad}-Biafada\il{Biafada} subgroup are considered and it is established that in these languages, for the numeral ‘10', the form \textbf{*}\textit{{}-po} is attested. In the final section of \sectref{sec:3.12.1}, namely in \sectref{sec:3.12.1.8} the forms of numerals in the seven northern subgroups are compared, and in particular it is concluded that for Proto-Northern Atlantic\il{Proto-Northern Atlantic}, the most probable reconstruction for the numeral '10' is the reconstruction of \textit{*pok}.

In Sections \sectref{sec:3.12.2.1}--\sectref{sec:3.12.2.4}, the numeral systems in each of the four subgroups of the second Atlantic group, namely Bak, are discussed consecutively. The final section concerning the Bak group (3.12.2.5) concludes that the only candidate for reconstructing '10' in the Proto-Bak\il{Proto-Bak} (in addition to the possible model 10 = 5 * 2) is the root *-\textit{taaj}.

In the final paragraph of \sectref{sec:3.12}, namely in \sectref{sec:3.12.3}, the systems of the North Atlantic languages and the Bak Atlantic languages are compared. This paragraph concludes that the comparative evidence points to the total absence of common roots present in both groups. The only exception to this is the root \textit{*tɔk} \textit{/} \textit{*tVk} ‘five'. Accordingly, it is concluded that it is impossible to reconstruct the Proto-Atlantic\il{Proto-Atlantic} root for the numeral '10' without the Niger-Congo context.

In \chapref{sec:4}, reconstructions for each family are compared. Accordingly, \chapref{sec:4} has a different structure. If in \chapref{sec:3} each of the sections is devoted to a particular family of languages (in particular, \sectref{sec:3.12} is devoted to the Atlantic languages), then in \chapref{sec:4} each section is devoted to the prospects for the reconstruction of each Niger-Congo numeral. So\il{So}, in \sectref{sec:4.10} all intermediate reconstructions for the numeral '10' are considered. It turns out, in particular, that the form \textit{*-taaj} reconstructed for '10' in the Proto-Bak\il{Proto-Bak} does not find parallels in other Niger-Congo branches. In contrast, the root \textit{*pok} '10', reconstructed for the North Atlantic languages, can be related to the roots reconstructed for the vast majority of Niger-Congo families (it seems to be missing only in Ijo, Dogon and Kordofanian). Based on the NC comparison, the root for '10' is reconstructed as \textit{*pu} \textit{/} \textit{*fu.}

\chapref{sec:5} traces the history of the numerals of Niger-Congo, reconstructed in \chapref{sec:4}, in each individual family of languages. Accordingly, each section, as in \chapref{sec:3}, is devoted to one of the NC families. So\il{So}, \sectref{sec:5.12} is devoted to the Atlantic languages. In particular, it is concluded that in the North Atlantic languages the term for '10' has been preserved in three sub-groups (Wolof\il{Wolof} *\textit{fukk}, Proto-Tenda\il{Proto-Tenda} *\textit{pəxw}, Proto-Jaad-Biafada\il{Proto-Jaad-Biafada} *\textit{po}). In the other subgroups it is replaced with isolated innovations. The forms of the Bak languages are also innovated.

So, the basic logic of the chosen structure of the book is as follows: we will consistently move from reconstructions in individual families (\chapref{sec:3}) to the reconstruction of each Niger-Congo numeral (\chapref{sec:4}) and to the interpretation of each individual family in the Niger-Congo context (\chapref{sec:5}). We will take into account the provisions formulated in the preliminary chapters concerning noun classes in numerals (\chapref{sec:1}) and changes by analogy in systems of numerals (\chapref{sec:2}).


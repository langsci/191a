\addchap{Acknowledgments}

Today the greatest benefit to being a researcher is the opportunity to directly contact leading specialists in the comparative studies of African languages. Even the best database does not ensure the proper interpretation of the results achieved by other scholars. In the course of my work on this monograph I have benefited from the help of many colleagues, whose comments and suggestions I greatly appreciate.  My particular thanks go to Guillaume Segerer (Atlantic languages and RefLex database), Valentin Vydrin (Mande languages), Raymond Boyd (Adamawa languages), Larry Hyman (Bantu languages and Benue-Congo in general), Mark Van de Velde (Bantu languages), Marie-Paule Ferry (Tenda languages), Pascal Boyeldieu (Bua\il{Bua!} languages and Laal\il{Laal!}), Marion Cheucle (Bantu A.80), Denis Creissels (Balant\il{Balant!}), Sylvie Voisin-Nouguier (Buy), Ekaterina Golovko (Baga Fore\il{Baga Fore!}), Odette Ambouroue (Orungu\il{Orungu!}) and many others. It is a great pleasure for me to thank you all!

My special gratitude is addressed to colleagues who read the first version of the manuscript of the book and made a number of valuable critical remarks. These are members of the editorial board of this series of Niger-Congo Comparative Studies: Valentin Vydrin, Larry Hyman, John Watters and Guillaume Segerer. I tried as much as possible to take their remarks into account. Naturally, all responsibility for inevitable mistakes and shortcomings lies with me.

I should like to express especially my gratitude for Sebastian Nordhoff for the layout of this book. Many thanks for my proofreaders  – their comments were very useful for me.


\chapter{Step-by-step reconstruction of numerals in the branches of Niger-Congo} \label{sec:3}
In this chapter we will try to create a step-by-step reconstruction of numeral systems for each separate family independent of the data from the other NC families. For each family we shall examine the range of basic numerals from ‘1’ to ‘10’ and then the numerals for ‘20’, ‘100’ and ‘1000’. We begin our overview with the largest family, Benue-Congo. 

\section{Benue-Congo} %3.1

There is no Benue-Congo classification that is accepted by all scholars. As noted, the inventory of Benue-Congo groups mainly follows the classification of Kay \citet[266-269]{Williamson1989b}. We repeat here the scheme of BC given above, in the introduction as \tabref{tab:3:bclg}.

\begin{table}
\caption{Benue-Congo languages}
\label{tab:3:bclg}
\begin{tabularx}{.8\textwidth}{XXl}
\lsptoprule
*Western BC & *Eastern BC & Isolated BC\\
\midrule 
Nupoid & Kainji & Oko\index{Oko!}\\
Defoid & Platoid & Akpes\index{Akpes!} \\
Edoid & Cross & Ikaan\index{Ikaan!}\\
Igboid & Jukunoid & Lufu\index{Lufu!}\\
Idomoid & \textbf{Bantoid} & \\
\lspbottomrule
\end{tabularx}
\end{table}

{\textmd{Let us begin our overview with the largest group of Bantoid languages.}}
 
\subsection{The Bantoid languages (including Bantu)}\label{sec:3.1.1}
The reconstruction of numerals in the Bantoid languages is based on 140 sources for the major branches of this family. What follows is the result of our step-by-step analysis of numeral systems in these languages.

 
\subsubsection{‘One’} %3.1.1.1.
\largerpage
We shall collect the main forms for ‘1’ in different branches of the Bantoid languages. The last column of \tabref{tab:3:1}. shows some isolated forms for ‘1’ which seem to be innovations.


\begin{table}
\caption{Bantoid stems for `1'}
\label{tab:3:1}
\fittable{
\small
\begin{tabular}{lllll}
\lsptoprule
\bfseries ~Branch & \bfseries ~Language & \bfseries `1' & \bfseries `1' & \bfseries `1'\\
\midrule
\multicolumn{2}{l}{Northern}\\
\midrule
{Dakoid} & {Chamba-}\il{Chamba}\textmd{Daka}\il{Chamba-Daka} & \bfseries ~ & {~} & {nòòní}\\
{*Mambiloid} & {~} & {mwi} & {cin, jer} & {~}\\
{Fam}\il{Fam} &  &  &  & {wuni}\footnote{The Fam\il{Fam} and Tiba\il{Tiba} (Fà) forms are quoted according to \citet{Blenchndc}) and \citet{Boyd1999} respectively. The online version of Boyd (\url{https://hal.archives-ouvertes.fr/hal-00323718v3}) differs from the printed one.}\\
{Tiba\il{Tiba} (Fà)} &  &  & {à-kīn-á} & \\

\tablevspace
\multicolumn{2}{l}{Southern}\\
\midrule
\bfseries *Bantu\footnote{An asterisk (*) in the second column of the tables (here and below) means that in the corresponding line all the forms are reconstructed. However, with the exception of the Proto-Bantu\il{Proto-Bantu} line, which indicates real reconstructions in BLR3 (*), all other reconstructions are \textstylegtbafwordclickable{hypothetical (\#)} and reflect the most typical form /forms attested in a particular branch of Benue-Congo. Forms that may be related are grouped in tables within the columns. The last column of the tables shows isolated forms that are likely to be innovations.} &  & \bfseries mòì/mòdì, mòtí &  & \bfseries p/m/b-ókó\\
{*Beboid} & {~} & {mwi/mu} & {~} & {baka,} \textmd{kpaŋ}\\
{*Yemne-Kimbi} & {~} & {mwe} & {~} & {~}\\
{*Ekoid} & {~} & {~} & {ji(ŋ) /rəŋ?}  & {yet?}\footnote{Concerning the form \textit{yet} in Ekoid, I quote a precious remark of John Watters (p.c.):~~‘’The actual root for Proto-Ekoid may be \textbf{-t {\textasciitilde}-d}. The /aŋ/in some Ekoid languages may be an accretion.  The \textit{yét} morphologically is /yé-t/with the CV being a class agreement prefix, and \textbf{-t} being the root. So\il{So} the \textbf{-t} may be closer to the Bantu \textit{moti}. I’m not sure how \textit{ó-mè} in Mbe\il{Mbe} figures in with the rest of Ekoid, but one possibility is that the \textbf{-mè} root derives from /me-t/. Ekoid needs further work”.}\\
{*Jarawan} & {~} & {moʔ} & {~} & {(ɗik)}\\
{*Mamfe} & {~} & {mɔt /ma} & {~} & {~}\\
{*Mbam} & {~} & {mwe/mùʔ} & {~} & {~}\\
{Mbe}\il{Mbe} & {Mbe}\il{Mbe} & {ó-mè} & {~} & {~}\\
{Ndemli}\il{Ndemli} & {Ndemli}\il{Ndemli} & {m{\`{ɔ}}h{\'{ɔ}}} & {~} & {~}\\
{Tikar}\il{Tikar} & {Tikar}\il{Tikar} & {mbɔʔ} & {~} & {~}\\
{*Tivoid} & {~} & {mɔ(m)} & {~} & {~}\\
{*Esimbi}\il{Esimbi} & {~} & {~} & {~} & {nə}\\
{Wide Grassfields} & {Befang}\il{Befang} & {moʔ~} & {~} & {~}\\
{GF: Mbam-Nkam} & {Bamileke}\il{Bamileke} & {moʔ} & {~} & {cu}\\
{GF: Mbam-Nkam} & {Ngemba}\il{Ngemba} & {mɔʔɔ} & {~} & {~}\\
{GF: Mbam-Nkam} & {Nkambe} & {moʔ(sír)} & {~} & {~}\\
{GF: Mbam-Nkam} & {Nun} & {moʔ}  & {~} & {~}\\
{GF: Momo} & {~} & {moʔ} & {~} & {fiŋ}\\
{GF: Ring}\il{Ring} & {~} & {moʔ} & {~} & {~}\\
\lspbottomrule
\end{tabular}
}
\end{table} 

At first glance, the terms for `1' in the majority of the Bantoid languages appear to be quite homogeneous, their roots being traceable to either \textit{*moʔ} or \textit{*moi/mwi} of uncertain etymology. The misleading similarity of the Bantu roots \textit{mòì,} \textit{mòdì,} \textit{mòtí} may be due to the merger of the noun class prefix \textbf{*m{\`{ʊ}}-} with the nominal base\footnote{I agree with Larry Hyman who reacted to this point: “This would suggest that ‘1’ was a noun; possible, just like ‘10’, but note that `2'--`5' are not nouns!” (p.c.).}. This hypothesis (developed in detail in \citealt{Vanhoudt1994}) has now found its way into the BLR (cf. BLR3 \textit{sub} \textit{mòdì} (NC): ‘\textit{plutôt} \textit{m{\`{ʊ}}-òdì:} \textit{voir} \textit{\citealt{Vanhoudt1994}~}’). 

Among other common Bantu forms are \textit{mócà} (zones KN), \textit{mòtí} (ABCEGHKLRS) <\textit{*m{\`{ʊ}}-òtì}, \textit{m{\'{ʊ}}égá} (zones BH) (BLR3: \textit{«mòì} \textit{+}  {suffix»}), and \textit{mòì} (ABCDEFGJKLMRS). As will be shown below, the presence of a nasal prefix in the Bantoid numerals is suggested by the distribution of these forms in Benue-Congo. Those BC branches that have nasalless roots within the nominal classes ‘one’ and ‘three’ lack the terms for ‘one’ with a nasal consonant. 

This interpretation, however, does not address two major issues, namely 1) whether the forms in question (e.g. \textit{*} \textit{-òdì/} \textit{-oti/} \textit{-oʔi}\footnote{Larry Hyman: “The glottal stop goes back to a velar in Grassfields; it could be either alveolar or velar in Tikar\il{Tikar}”.( p.c.).}) consist of one or more roots and 2) whether the open back vowel belongs to the root. 

A solution to the former problem may turn out to depend on how the latter is treated.

Within the context of Niger-Congo, it is conceivable that the Proto-Bantu\il{Proto-Bantu} \textit{òdì} may go back to *\textit{ò-dì}, with \textbf{*ò-} being a marker of the NC noun class 1 (\textbf{*ko-/} \textbf{ʔo-} according to my reconstruction). This hypothesis will receive a more detailed treatment in the next chapter. At this point, we will only note that it is quite problematic to explain the common reflexes of \textbf{*-di,} \textbf{*ti,} and \textbf{*ʔ-} in Bantu within this hypothesis. Moreover, the etymological relationship between these roots (disregarding \textit{*di} and \textit{mɔ(m)} (Tivoid), \textit{ó-mè} (Mbe\il{Mbe}), \textit{ma} (Mamfe), etc.) would be much less transparent than that in case of \textit{modi} \textit{{\textasciitilde} moti} or even \textit{-odi {\textasciitilde} -oti}.

The secondary PB\il{PB} form \textit{*ókó} (zones ABCHF) (BLR3: ’’\citealt{Janssens1994}: alternance C1 p/m/b-ókó- protoforme secondaire, cf. `seul'”) is comparable to \textit{*baka} (Beboid: Fio\il{Fio} \textit{mbákâ} \textit{{\textasciitilde} nbáhá}, Nchane\il{Nchane} (Mungong) \textit{m⁴ba³ka⁴}). It should be noted that the above considerations allow us to explain the initial consonant (and the following back vowel) in these forms as noun class morphemes, too. 

The Northern Bantoid \textit{kin/cin} is remarkable and will be addressed later in this chapter. 

The Bamileke\il{Bamileke} \textit{*tʃu} (Fefe\il{Fefe} \textit{ʃɯ}\textit{ʔ}, Medumba\il{Medumba} a\textit{ntʃʊ}\textit{ʔ}, Nda'nda' \textit{ŋtʃ{\`{ɔ}}ʔ}, etc.) is possibly related to the Bantu \textit{*tʊ} (BCDEGLP) ‘alone, empty, vain’. 

 
\subsubsection{‘Two’ and ‘Three’} %3.1.1.2.


Without exception, the reconstructed root for `two' in all Bantoid branches has an initial labial consonant, either voiced (b-) or voiceless (p-/f-). A more precise reconstruction of the proto-form is beyond my cognizance. The forms cited above do not permit a conclusion with regard to the number of roots involved (one or two). When comparing the most commonly attested forms \textit{*pa/} \textit{fe} and \textit{*baa}, it is necessary to keep in mind that at least the Proto-Bantu\il{Proto-Bantu} \textit{*bàd{\'{ɩ}}/b{\`{ɩ}}d{\'{ɩ}}} could be a reflex of \textit{*di}. In the case of \textbf{ba}- the proto-form should be interpreted as a prefix of a plural noun class (possibly class 2).\footnote{John Watters: “This analysis, if correct, could work also for most of Bantoid. So\il{So} Ekoid would derive from \textbf{ba}- prefix and \textbf{-l {\textasciitilde} -d {\textasciitilde} -n} root. However, the /b/may derive from /p/. Ekoid may derive from \textit{*-pal} and then you have the many other Bantoid languages with /p/” (p.c.).} The latter proposal finds support in the dialectal Proto-Bantu form \textit{jòdè} (zones BH) (<\textit{*jò-dè}?). The main forms show the following zonal distribution: \textit{bàd{\'{ɩ}}} (ABCHKLR), \textit{b{\`{ɩ}}d{\'{ɩ}}} (CDEFGJKLMNPS), \textit{b{\'{ɩ}}d{\`{ɩ}}} (?).

  It was repeatedly stressed that the root for `three' (*\textit{tat}) is one of the most stable in NC and in the Bantoid languages in particular. Phonetic variation within this root will be studied in \chapref{sec:4}. 

  
\begin{table}[t]
\caption{\label{tab:3:2}Bantoid stems for `2' and `3'}
\begin{tabularx}{\textwidth}{lllll}
\lsptoprule 
% \bfseries BANTOID
&  \bfseries ~Language & \bfseries `2' & \bfseries `2' & \bfseries `3' \\
\midrule
\multicolumn{3}{l}{Northern} \\
\midrule
{Dakoid} & {Chamba-}\il{Chamba}\textmd{Daka}\il{Chamba-Daka} & {~} & {bààrá} & {tárā}\\
{*Mambiloid} & {~} & {fee/fal/hal}  & {baa} & {taar}\\
{Fam}\il{Fam} &  &  & {baale} & {tawnə}\\
{Tiba}\il{Tiba}\textmd{ (Fà)} &  &  & {à-ɓȩ̄ȩ̄r-á} & {à-tár-á}\\
\tablevspace
\multicolumn{3}{l}{Southern}  \\
\midrule  
\bfseries *Bantu & {~} &  & \bfseries bàd{\'{ɩ}} /b{\`{ɩ}}d{\'{ɩ}} & \bfseries tát{\`{ʊ}} /cát{\`{ʊ}}\\
{*Beboid} & {~} & {fe} & {~} & {tat, te}\\
{*Yemne-Kimbi} & {~} & {fi(n)} & {~} & {to}\\
{*Ekoid} & {~} & {~} & {ba(l)} & {sa/ra}\\
{*Jarawan} & {~} & {~} & {ɓar} & {tat}\\
{*Mamfe} & {~} & {pay /pea} & {~} & {rat /lɛ}\\
{*Mbam} & {~} & {fande?}  & {bante?} & {tat}\\
{Mbe}\il{Mbe} & {Mbe}\il{Mbe} & {pʷâl} & {~} & {sá}\\
{Ndemli}\il{Ndemli} & {Ndemli}\il{Ndemli} & {if{\'{ɛ}}}  & {~} & {ítáá} \\
{Tikar}\il{Tikar} & {Tikar}\il{Tikar} &  & {ɓî} & {l{\^{e}}}\\
{*Tivoid} & {~} & {hal/har/vial} & {~} & {tat}\\
{*Esimbi}\il{Esimbi} & {~} & {ra-kpə?} & {~} & {kələ (< *lə?)}\\
{Wide Grassfields} & {Befang}\il{Befang} & {fe} & {~} & {táí}\\
{GF: Mbam-Nkam} & {Bamileke}\il{Bamileke} & {pu/pwe} & {bo/bie} & {tat}\\
{GF: Mbam-Nkam} & {Ngemba}\il{Ngemba} & {paa} & {baa /bəɡə} & {tarə}\\
{GF: Mbam-Nkam} & {Nkambe} & {~} & {baa} & {tar}\\
{GF: Mbam-Nkam} & {Nun} & {paa} & {baa} & {tɛt}\\
{GF: Momo} & {~} & {~} & {be} & {tat}\\
{GF: Ring}\il{Ring} & {~} & {~} & {bo/ba} & {tat}\\
\lspbottomrule
\end{tabularx}
\end{table}

  
\subsubsection{‘Four’ and ‘Five’}\label{sec:3.1.1.3}
The well-known NC root \textit{*nai} `four' is represented in all of the pertinent languages. The only exception is Grassfields, where it was replaced with the innovative \textit{*kwa/kya}. According to Roger Blench, Momo \textit{-kpi} and Ring\il{Ring} \textit{kaìkò} as well as the Proto-Eastern Grassfields\il{Proto-Eastern Grassfields} \textit{*-kùa} go back to the Proto-Benue-Congo\il{Proto-Benue-Congo} \textit{\#-kpà(ko)} (\citealt{Blench2004}: \#387). This root, however, is commonly found in Mbam-Nkam, i.e. in all Grassfields languages, and is barely attested outside this branch.

\newpage 
The root for `five' is almost invariably \textit{*tan}. One possible exception is the Ekoid form, unless \textit{*don/ron/lon} (Ekajuk\il{Ekajuk} \textit{nlɔn}, Ejagham\il{Ejagham} \textit{ér{\^{o}}n}, Nkem\il{Nkem}-Nkum\il{Nkem-Nkum} \textit{íro{\^{̱}}n}) is a reflex of \textit{*tan}).

It should be noted that the Ndemli\il{Ndemli} root \textit{itʃìjè} may be related to \textit{kwV} in the Grassfields languages. As we hope to demonstrate below, this is probably not a coincidence. 

\begin{table}[t]
\caption{\label{tab:3:3}Bantoid stems for `4' and `5'}

\fittable{
\begin{tabular}{lp{13mm}llll}
\lsptoprule 
% BANTOID &
~ &   & `4' & `4' & `5' & `5' \\
\midrule
Northern \\
\midrule
Dakoid & Chamba-\il{Chamba}Daka\il{Chamba-Daka} & nàà-sá &   & túùná &  \\
*Mambiloid &   & na(n) &   & tien/tin/con/son & ngii?\\
Fam\il{Fam} &  &  & daare & tʃwiine & \\
Tiba\il{Tiba} (Fà) &  & à-nè-á &  & à-tō̧ò̧ŋ-á, t{\={u}}ùŋ & \\
\textbf{*Bantu} & \textbf{~} & \textbf{nàì/(nàí)} & \textbf{~} & \textbf{táànò/cáànò} & \textbf{~}\\

\tablevspace 
Southern \\
\midrule
*Beboid &   & na, ne &   & ti(n) &  \\
*Yemne-Kimbi &   & ni &   &   & kpɔn\\
*Ekoid &   & ni &   &   & don/lon\footnotemark{}\\
*Jarawan &   & yi-ne? &   & towun/twan &  \\
*Mamfe &   & n(w)i &   & ta(y) &  \\
*Mbam &   & ni(s) &   & taan &  \\
Mbe\il{Mbe} & Mbe\il{Mbe} & ñî &   & tʃân &  \\
Ndemli\il{Ndemli} & Ndemli\il{Ndemli} &   & itʃìjè & ítâŋ &  \\
Tikar\il{Tikar} & Tikar\il{Tikar} & ɲî &   & ʃ{\~{\^æ}} &  \\
*Tivoid &   & ɲi(n) &   & tan &  \\
*Esimbi\il{Esimbi} &   & ɲi &   & tənə &  \\
Wide Grassfields & Befang\il{Befang} &  & k{\ᶣ}à (kɥà) & {\`{ɪ}}tʲ{\^{ə}}n &  \\
GF: Mbam-Nkam & Bamileke\il{Bamileke} &   & kwa/kwo & tan &  \\
GF: Mbam-Nkam & Ngemba\il{Ngemba} &   & kwa/kya & taa(n) &  \\
GF: Mbam-Nkam & Nkambe &   & kwe/kye & tan /ton &  \\
GF: Mbam-Nkam & Nun &   & kwa/kpa & tan /tɛn &  \\
GF: Momo &   &   & kwe & tan &  \\
GF: Ring\il{Ring} &   &   & kwi /kye /tsə & tan &  \\
\lspbottomrule
\end{tabular}
}
\end{table}
\footnotetext{John Watters: the Proto-Ekoid probably is *-ron (p.c.).}


 
\subsubsection{`Six'}\label{sec:3.1.1.4} 
  
The Grassfields languages show a common root \textit{*toʔo.} Outside Grassfields, it is attested only in Ndemli\il{Ndemli} (just like the Grassfields root for `five') and thus can hardly be reconstructed for Proto-Bantoid\il{Proto-Bantoid}. However, we cannot exclude this, if PB\il{PB} *\textit{t{\'{ʊ}}{\'{ʊ}}bá}\textbf{~}‘6’ attested in zones ABCD is related to the Grassfields forms.

\begin{table}[b!]
\caption{Bantoid stems and patterns for `6'}
\small
 \begin{tabularx}{\textwidth}{llQlQQ}
\lsptoprule
% BANTOID
&   & `6' & `6' & `6' & `6' \\
\midrule
Northern\\
\midrule
Dakoid & Chamba-\il{Chamba}Daka\il{Chamba-Daka} & {~} &   & <5? &  \\
*Mambiloid &   & {~} &   & 5+1 &  \\
Fam\il{Fam} &  & {} &  & 5+1 & \\
Tiba\il{Tiba} (Fà) &  & {} &  & 5+1 & \\
\tablevspace

Southern\\
\midrule
\textbf{*Bantu} & \textbf{~} & {\textbf{tándà}  \textbf{<}  3redupl.?}  & \textbf{t{\'{ʊ}}{\'{ʊ}}bá~} &  & \textbf{càmb-,} \textbf{kaaga}\\
*Beboid &   & {~} &   &   & so\\
*Yemne-Kimbi &   & {3PL?}  &   &   &  \\
*Ekoid &   & {3+3} &   &   &  \\
*Jarawan &   & {~} &   & 5+1 &  \\
*Mamfe &   & {~} &   &   & kene? \\
*Mbam &   & {3PL} &   & 5+1 &  \\
Mbe\il{Mbe} & Mbe\il{Mbe} & {3+3}  &   &   &  \\
Ndemli\il{Ndemli} & Ndemli\il{Ndemli} & {~} & tóhó &   &  \\
Tikar\il{Tikar} & Tikar\il{Tikar} & {~3PL?}  &   &   & \\
*Tivoid &   & {3redupl., 2*3?}  &   &   &  \\
*Esimbi\il{Esimbi} &   & <3redupl.? &   &   &  \\
Wide Grassfields & Befang\il{Befang} & {~} & ⁿd{\`{ʊ}}fú &   & \\
GF: Mbam-Nkam & Bamileke\il{Bamileke} & {~} & toɣo &   &  \\
GF: Mbam-Nkam & Ngemba\il{Ngemba} & {~} & toʔo &   &  \\
GF: Mbam-Nkam & Nkambe & {~} & ntunfu &   &  \\
GF: Mbam-Nkam & Nun & {~} & ntúwó/tuʔo &   &  \\
GF: Momo &   & {~} &   &   & foɣ\\
GF: Ring\il{Ring} & {~} &   & tufa &   &  \\
\lspbottomrule
\end{tabularx}
\end{table}

\newpage    
As in some other NC branches, three patterns that can be used to derive `6' from `3' are attested in the Bantoid languages (the following observations are even more relevant in the case of the patterns for `eight' based on `four'):

\begin{enumerate}
\item The change of a class prefix (or its addition): Ajumbu\il{Ajumbu} \textit{tò} ‘3’ > \textit{kʲà-tò} ‘6’; this pattern is possibly attested in Tutomb (Mbam) \textit{p{\'{ɛ}}-dààt} ‘3’ > \textit{pí-tʃín-dìt} ‘6’, Elip\il{Elip} \textit{b{\'{ʊ}}-dád̥} ‘3’ > \textit{b{\'{ʊ}}-thín-dàd̥} ‘6’ (this pattern is marked ‘3PL’ in the table above). To strengthen the etymology for `six' in Tutomb, it should be noted that in Tunen\il{Tunen} (another Mbam language) that has \textit{*tat} ‘3’ > \textit{lal} (\textit{b{\'{ɛ}}-lál{\'{ɔ}}}), the term for `six' also contains [l]: \textit{p{\'{ɛ}}-l{\'{ɛ}}ⁿdálɔ.}
  
\item The combination of `three' and `two': Lyive\il{Lyive}: \textit{hjâl} ’2’, \textit{tàt} ‘3’, \textit{k{\`{ə}}l{\`{ə}}-k{\`{ə}}-tàt} ‘6’ (<‘2*3’?).
\item The reduplication of `three' (or the simple addition ‘3+3’): Ekajuk\il{Ekajuk} \textit{n-ra} ‘3’ >\textit{n-ra-ke-ra}~‘6’, Ejagham\il{Ejagham} \textit{é-sá} ‘3’ > \textit{è-sá-ɡà-sá}~‘6’, Nkem\il{Nkem}-Nkum\il{Nkem-Nkum} \textit{i-ra} ‘3’ > \textit{i-ra-ra} ‘6’, Mbe\il{Mbe} \textit{b{\'{ɛ}}-sá} ‘3’ >\textit{b{\`{ɛ}}-s{\^{e}}-sár}‘6’, Tiv\il{Tiv} \textit{ú-tá{\'{r}}} ‘3’ > \textit{á-tér-á-tá{\'{r}}} (this pattern is marked as ‘3+3’ in the table above). 
\end{enumerate}
The Kenyang\il{Kenyang} (Mamfe) form \textit{b{\'{ɛ}}-tándât} ’6’ (cf. \textit{b{\'{ɛ}}-rát} ‘3’) deserves special discussion. This form is reminiscent of the common Bantu form \textit{tándà} ‘6’ attested in zones DGM. Its extended variant \textit{tándàt{\'{ʊ}}} is found in EFGJS, while the GNS zones use the form \textit{tántàt{\'{ʊ}}} which is even more interesting. Are the Bantu \textit{tándà} forms cited above based on `3'?  If so, \textit{*tat-tat} > \textit{tatat} (\textit{tántàt{\'{ʊ}}}) in the languages to which Dahl's law is applicable as well (> \textit{tandat,} \textit{tanda}).

In this case, the form \textit{t{\'{ʊ}}{\'{ʊ}}bá} (zones ABCD) that can be interpreted as ‘*3*2’: *\textit{tat-X-ba} may also be a derivative form.  

If so, the aforementioned Bantu forms (as well as the Kenyang\il{Kenyang} form) are probably not innovations. They may reflect a Proto-Bantoid\il{Proto-Bantoid} model where `six' is based on `three'. It should be noted that a close parallel to the Kenyang form is attested in the Mbam branch: Nomaande\il{Nomaande} \textit{be-tíndétú} ‘6’.

In sum, it appears that the most probable word-formation pattern for `six' in Proto-Bantoid\il{Proto-Bantoid} is ‘3+3’ or ‘3PL’.

   
\subsubsection{`Seven'}\label{sec:3.1.1.5}
\begin{table}
\caption{\label{tab:3:5}Bantoid stems and patterns for `7'}
\footnotesize
\begin{tabularx}{\textwidth}{llXllll}
\lsptoprule

% BANTOID &
~ &   & `7' & `7' & `7' & `7' & `7' \\
\midrule
Northern\\
\midrule 
Dakoid & Chamba-\il{Chamba}Daka\il{Chamba-Daka} &   &   &   &   & dùtím\\
*Mambiloid &   &   &   &   & 5+2 &  \\
Fam\il{Fam} &  &  &  &  & 5+2 & \\
Tiba\il{Tiba} (Fà) &  &  &  &  & 5+2 & \\
\textbf{*Bantu} & \textbf{~} & \textbf{càmbà-d{\`{ɩ}}/càmb{\`{ʊ}}-à-d{\`{ɩ}}} & \textbf{6+1?}  & \textbf{~} & \textbf{~} & \textbf{púngàt{\'{ɩ}}}\\

\tablevspace
Southern\\
*Beboid &   & fumba? & 6+1 & 4+3 &   &  \\
*Yemne-Kimbi &   &   &   & 4+3 &   &  \\
*Ekoid &   & sima? &   & 4+3? &   &  \\
*Jarawan &   &   &   &   & 5+2 &  \\
*Mamfe &   &   & 6+1 &   &   &  \\
*Mbam &   &   & 6+1 &   &   &  \\
Mbe\il{Mbe} & Mbe\il{Mbe} &   &   &   & 5+2 &  \\
Ndemli\il{Ndemli} & Ndemli\il{Ndemli} & sàᵐbá &   &   &   &  \\
Tikar\il{Tikar} & Tikar\il{Tikar} & ʃâmɓì &   &   &   &  \\
*Tivoid &   &   & `6+1 &   & 5+2 &  \\
*Esimbi\il{Esimbi} &   &   &   &   & 5+2 &  \\
Wide Grassfields & Befang\il{Befang} &   &   & 4+3 &   &  \\
GF: Mbam-Nkam & Bamileke\il{Bamileke} & samba &   &   &   &  \\
GF: Mbam-Nkam & Ngemba\il{Ngemba} & samba &   &   &   &  \\
GF: Mbam-Nkam & Nkambe & samba &   &   &   &  \\
GF: Mbam-Nkam & Nun & samba &   & 4+3 &   &  \\
GF: Momo &   & sambe &   &   &   &  \\
GF: Ring\il{Ring} &   & samba &   &   &   &  \\
\lspbottomrule
\end{tabularx}
\end{table}

The case of `seven' seems pretty straightforward. In the majority of the Bantoid branches (including Bantu) the root is \textit{*samba/camba}. However, there is still a question whether this root is indeed primary: its Bantu reflex is strikingly similar to the root for `six'. \tabref{tab:3:6} shows some selected examples.

\begin{table}
\caption{\label{tab:3:6}Similarities between `6' and `7' in Bantu} 
\footnotesize
\begin{tabularx}{\textwidth}{Xll}
\lsptoprule
& `6' & `7' \\
\midrule 
PB\il{PB} & càmbànò (HL)/\newline cààmànò (ABCHLR)/\newline càmbombo (L) & càmbà-d{\`{ɩ}}/càmb{\`{ʊ}}-à-d{\`{ɩ}}\\
A40 Bankon\il{Bankon} & bi-sámà & bi-sámb{\`{ɔ}}k\\
A80 Kol\il{Kol} & twáb & tábɛl\\
B20 Mbangwe\il{Mbangwe} & -syami & ntsaami\\
B60 Mbere\il{Mbere} & -syaami & ntsaami\\
B70 Teke-Tege\il{Teke-Tege} & ósámìnì & ónsààmì\\
B80 Tiene\il{Tiene} & ísyam & nsam\\
C40 Sengele\il{Sengele} & ísama~ & ísambiálé\\
C90 Ndengese\il{Ndengese} & isamo & isambé\\
\lspbottomrule
\end{tabularx}
\end{table}

    
It is noteworthy that the terms for `six' and `seven' show similarity not only in case of the root in question, but in case of other roots as well, e.g. J50:
Fuliiru\il{Fuliiru} \textit{-lindátù} ‘6’{\textasciitilde} -\textit{linda} ‘7’,
Shi\il{Shi} \textit{ńdarhu} ‘6’{\textasciitilde} \textit{ńda} ‘7’.  This similarity is usually conditioned by one of the following factors:


\begin{itemize}
\item the terms for `six' and `seven' follow the patterns ‘10–4’ and ‘10–3’ respectively: 
Yeyi\il{Yeyi} (Bantu R40) \textit{vùndʒà {\'{ɛ}} n{\'{ɛ}}{\'{ɛ}}} ‘6’ (‘10’ ‘break’ ‘4 (fingers)’), 
\textit{vùndʒà {\'{ɛ}} táâːtō} ‘7’ (‘10’ ‘break’ ‘3 (fingers)’. 
This, however, is very rarely attested.
\item the term for `seven' is based on `six' (`6+1'). This pattern is much more common (see \tabref{tab:3:7}).
\item The similarity may also be due to the derivation of these terms from `five' using ‘5+1’ and ‘5+2’ patterns, respectively (this is the most common case). It should be noted that there is another, much less transparent pattern for `seven' (`X+2' or `5+X'). It is frequently attested not only in the Bantoid languages, but also in the Mande languages.
\item Finally, we may be dealing with an alignment by analogy. \todo{maybe reference the relevant tables here}
\end{itemize}

\begin{table}
\caption{\label{tab:3:7}Common stems for `6' and `7' in Bantu}
\begin{tabularx}{\textwidth}{XXl} 
\lsptoprule
& `6' & `7' \\
\midrule 
J50 Fuliiru\il{Fuliiru} & -lindátù & -linda\\
J50 Shi\il{Shi} & ńdarhu & ńda\\
A80 Byep\il{Byep} & tʷ{\'{ɔ}}p & tʷ{\'{ɔ}}p ɓ{\`{ə}}l (6+?)\\
C10 Yaka\il{Yaka} & βúè & βúè nà -m{\`{ɔ}}tí (6+1)\\
D30 Budu\il{Budu} & m{\`{ɛ}}ɗìà & m{\`{ɛ}}ɗìàníkà (lit: níkà `to come') \\
M20 Malila\il{Malila} & {\'{ʊ}}mʊtʰaːⁿda & {\'{ʊ}}mʊtʰaːⁿda~na j{\v{e}}ːkʰa (6+1)\\
B10 Myene\il{Myene} & òɾówá & òɾwáɣén{\^{o}} (6+1)\\
\lspbottomrule
\end{tabularx}
\end{table}

\begin{table}
\caption{\label{tab:3:8}'6' and `7' from `5' in Bantu}
\begin{tabularx}{\textwidth}{XXX}
\lsptoprule
& `6' & `7' \\
\midrule 
H10 Koongo\il{Koongo} & sàmbánù & sàmbú-wàlì (wálì ‘2’)\\
K20 Nyemba\il{Nyemba} & pàndù & pàndù vàlì~(-vali ‘2’)\\
K60 Mbala\il{Mbala} & sambanu & nsambwadi (mbadi ‘2’)\\
L30 Luba-Katanga\il{Luba-Katanga} & isamba & isambaibindi (ibindi ‘2’)\\
R10 Khumbi\il{Khumbi} & epándú & epándúvalí (valí ‘2’)\\
\lspbottomrule
\end{tabularx}
\end{table}

 \largerpage 
Staying within the Bantoid family, it is difficult to say which of these explanations should be applied in the present case. If it is alignment by analogy, we should reconstruct a Proto-Bantoid\il{Proto-Bantoid} primary root *\textit{samba/camba} for `seven' and then explain the many irregular shifts in the forms of `six' (e.g. t > s) by analogy with this root (as shown above, the Proto-Bantu\il{Proto-Bantu} `six' is based on `three' (*tat)). We may also be dealing with a derived proto-form \textit{*sam-ba/cam-ba} with the second element probably going back to `two'.

 
\subsubsection{‘Eight’}\label{sec:3.1.1.6}


\begin{table}[b!]
\caption{\label{tab:3:9}Bantoid stems and patterns for `8'}
\fittable{ 
\begin{tabular}{lllll}
\lsptoprule

% BANTOID &
~ &   & `8' & `8' & `8' \\
\midrule
Northern\\
\midrule
Dakoid & Chamba-\il{Chamba}Daka\il{Chamba-Daka} &   &   & 7+1\\
*Mambiloid &   &   &   & 5+3\\
Fam\il{Fam} &  &  &  & 5+3\\
Tiba\il{Tiba} (Fà) &  &  &  & 5+3\\

\tablevspace
Southern\\
\midrule
\textbf{*Bantu} & \textbf{~} & \textbf{nainai(}4 redupl.)\textbf{/} \textbf{nake}  & \textbf{~} & \textbf{~}\\
*Beboid &   & ɲaŋ (<4?) &   &  \\
*Yemne-Kimbi &   & 4 redupl. &   &  \\
*Ekoid &   & 4+4 &   &  \\
*Jarawan &   &   &   & 5+3\\
*Mamfe &   & 4PL &   &  \\
*Mbam &   & 4 redupl. &   &  \\
Mbe\il{Mbe} & Mbe\il{Mbe} & 4 redupl. &   &  \\
Ndemli\il{Ndemli} & Ndemli\il{Ndemli} &   & f{\`{ɔ}}ːm{\'{ɔ}} &  \\
Tikar\il{Tikar} & Tikar\il{Tikar} &   &   &  \\
*Tivoid &   & 4 redupl. &   &  \\
*Esimbi\il{Esimbi} &   & 4 redupl. &   &  \\
Wide Grassfields & Befang\il{Befang} &  & éfómó &  \\
GF: Mbam-Nkam & Bamileke\il{Bamileke} &   & fum/hum/fo? &  \\
GF: Mbam-Nkam & Ngemba\il{Ngemba} &   & famə &  \\
GF: Mbam-Nkam & Nkambe &   & waami &  \\
GF: Mbam-Nkam & Nun &   & fame &  \\
GF: Momo &   &   & fami/foŋ &  \\
GF: Ring\il{Ring} &   &   & faamə &  \\
\lspbottomrule
\end{tabular}
}
\end{table}

Both Grassfields and Ndemli\il{Ndemli} share the common primary root for `nine' (\textit{*famV}). We have already seen this distribution, 
which only suggests that Ndemli belongs to the Grassfields branch (at least on the basis of their numeral systems). The majority of other branches point to the reconstruction of the term for `eight' as based on `four' (either by means of reduplication or by the noun class switch, or both).
   
\subsubsection{`Nine'} 
\begin{table}
\caption{\label{tab:3:10}Bantoid stems and patterns for `9'}

\fittable{
\begin{tabular}{lllllll}
\lsptoprule

% BANTOID &
~ &   & `9' & `9' & `9' & `9' & `9' \\
\midrule
Northern \\
\midrule
Dakoid & Chamba-\il{Chamba}Daka\il{Chamba-Daka} &   &   &   &   & kú{\={u}}m\\
*Mambiloid &   &   & 5+4 &   &   &  \\
Fam\il{Fam} &  &  & 5+4 &  &  & \\
Tiba\il{Tiba} (Fà) &  &  & 5+4 &  &  & \\
\tablevspace 

Southern\\
\midrule
\textbf{*Bantu} & \textbf{~} & \textbf{bùá} & \textbf{~5+4} & \textbf{~} & \textbf{~10-1} & \textbf{kèndá/} \textbf{jèndá}\\
*Beboid &   & bùkə? &   &   &   & fum{\textsubdot{b}}ɔ?\\
*Yemne-Kimbi &   &   & 5+4 &   &   &  \\
*Ekoid &   &   & 5+4 &   & 10-1 &  \\
*Jarawan &   &   & 5+4 &   &   &  \\
*Mamfe &   &   &   & 8+1 &   &  \\
*Mbam &   &   & 5+4 & 8+1 &   &  \\
Mbe\il{Mbe} & Mbe\il{Mbe} &   & 5+4 &   &   &  \\
Ndemli\il{Ndemli} & Ndemli\il{Ndemli} & bùʔ{\`{ɛ}} &   &   &   &  \\
Tikar\il{Tikar} & Tikar\il{Tikar} &   & 5+4? &   &   &  \\
*Tivoid &   &   & 5+4 & 8+1 &   &  \\
*Esimbi\il{Esimbi} &   &   & 5+4 &   &   &  \\
Wide Grassfields & Befang\il{Befang} &   & 5+4 &   &   &  \\
GF: Mbam-Nkam & Bamileke\il{Bamileke} & fuʔu &   &   &   &  \\
GF: Mbam-Nkam & Ngemba\il{Ngemba} & buʔu /puʔu &   &   &   &  \\
GF: Mbam-Nkam & Nkambe & b{\`{ʉ}}ʔ{\^{ʉ}}? búum? &   &   & 10-1? &  \\
GF: Mbam-Nkam & Nun & puʔu? &   &   &   & cipo?\\
GF: Momo &   & bok &   &   &   & ko? \\
GF: Ring\il{Ring} &   &   &   &   & 10-1 &  \\
\lspbottomrule
\end{tabular}
}
\end{table}

\newpage 
It seems likely that there was a primary root for `nine' in Proto-Bantoid\il{Proto-Bantoid}. It can be tentatively reconstructed as \textit{*bukV}.\footnote{John Watters: ‘’Given the distribution of these forms for ‘nine’ I would conclude that Proto-Bantoid\il{Proto-Bantoid} likely used 5+4 and that \textit{*bukV} was an innovation in the pre-Bantu era when Proto-Bantu\il{Proto-Bantu} had not yet separated from what became Grassfields and other closely located Bantoid groups’’.} In Bantu, this root is found in the ABCDHL zones. The most common pattern `5+4' (as well as the less frequently attested `10–1') often develops independently in various languages. A marginal pattern ‘8+1’, attested in Mamfe, Mbam and Tivoid is noteworthy. Because of its rarity, it is relevant for the genetic classification of the Bantu languages, since it is hard to imagine that this form developed independently in each of these branches. The last column of the table below lists bases that are exclusively found in a specific Bantoid branch. 

 
\subsubsection{`Ten'} %3.1.1.8.
\begin{table}
\caption{\label{tab:3:11}Bantoid stems for `10'}
\footnotesize
\begin{tabularx}{\textwidth}{ll QQQQQ}
\lsptoprule
% BANTOID &
~ &   & `10' & `10' & `10' & `10' & `10' \\
\midrule
Northern\\
\midrule
Dakoid & Chamba-\il{Chamba}Daka\il{Chamba-Daka} &   & kú{\={u}}m-k{\'{ə}}r{\'{ə}}r{\'{ə}} &   &   &  \\
*Mambiloid &   &   &   & cóŋ &   & job-, jer, jula ? f{\'{ɛ}}ŋ ?\\
Fam\il{Fam} &  &  &  &  &  & kwoy\\
Tiba\il{Tiba} (Fà) &  &  &  &  &  & \mbox{à-wó̧ó̧b-á}\\
\tablevspace

Southern\\ 
\midrule
\textbf{*Bantu} & \textbf{~} & \textbf{~} & \textbf{k{\'{ʊ}}mì/} \textbf{kámá} & \textbf{~} & \textbf{~} & \textbf{dòngò}\\
*Beboid &   & jo-fi/jo-fu &   &   &   &  \\
*Yemne-Kimbi &   & jo-fu &   & koɲ? &   &  \\
*Ekoid &   & fo &   &   &   & gol, wobo\\
*Jarawan &   &   &   &   &   & lum\\
*Mamfe &   & fia, bjo &   &   &   &  \\
*Mbam &   &   &   &   & p-wat/b-wad &  \\
Mbe\il{Mbe} & Mbe\il{Mbe} & fw{\^{ɔ}}r &   &   &  &  \\
Ndemli\il{Ndemli} & Ndemli\il{Ndemli} &   & dʒòm &   &   &  \\
Tikar\il{Tikar} & Tikar\il{Tikar} &   & w{\^{u}}m &   &   &  \\
*Tivoid &   & puɛ &  *ham &   & pɔt &  \\
*Esimbi\il{Esimbi} &   &   &   &   &   & bu ɣu? (< 9?)\\
Wide Grassfields & Befang\il{Befang} &   & éɣúm &   &   &  \\
GF: Mbam-Nkam & Bamileke\il{Bamileke} &   & ɣam &   &   &  \\
GF: Mbam-Nkam & Ngemba\il{Ngemba} &   & ɣám &   &   &  \\
GF: Mbam-Nkam & Nkambe &   & ʔum &   &   & ri/ru\\
GF: Mbam-Nkam & Nun &   & ɣom &   &   &  \\
GF: Momo &   &   & ɣum &   &   &  \\
GF: Ring\il{Ring} &   &   & ɣəm &   &   &  \\
\lspbottomrule
\end{tabularx}
\end{table}

At least two Bantoid roots (\textit{*fu} and \textit{*kum/} \textit{kam}) may be useful for our reconstruction purposes. Both of them are attested in no fewer than six of the Bantoid branches (note also the Chamba\il{Chamba}-Daka\il{Chamba-Daka} \textit{kú{\={u}}m} `nine'). The Mambiloid languages show the greatest variety of roots.

It should be noted that a separate Proto-Bantoid\il{Proto-Bantoid} form for `ten' is not traceable in some of the pertinent languages. Despite this, it has been preserved as a part of the term for `twenty', e.g. `ten' is attested as \textit{é-p{\'{ɔ}}ːt} in Ipulo\il{Ipulo} (Tivoid). This form is probably related to Tiv\il{Tiv} \textit{púè/} \textit{púwè} and Lyive\il{Lyive} e\textit{pù{\`{ɛ}}} and may be attested in the Mbam branch as well (Nubaca\il{Nubaca} \textit{mwa-pwat} ‘ten’, etc.).

It is clear, however, that the Ipulo\il{Ipulo} `twenty' (\textit{i-ham}) is derived from the Proto-Bantoid\il{Proto-Bantoid} term for `ten' by means of a noun class switch. The same can be applied to Bhele\il{Bhele} (D30): \textit{mɔk{\'{ɔ}}} ‘ten’ but \textit{e-kómi} \textit{í-ɓalé} ‘20’ (\textit{í-ɓalé} ‘two’). The root \textit{kam} will be discussed below in connection to the terms for `hundred'. 

\newpage  
\subsubsection{‘Twenty’}\label{sec:3.1.1.9}
It is not necessary to quote the forms for `twenty', since in the majority of the Bantoid branches (including Bantu) this term is based on `ten' and follows the pattern ‘10*2’. Some minor but peculiar variations should be noted here, but all of them are of little significance for our reconstruction. E.g. the term for `twenty' often employs the plural noun class with the two components in agreement. However, non-compound forms based on `ten' or `two' in the plural are also attested. For instance, in one of the Bafut\il{Bafut} dialects \textit{báà} ‘two’, \textit{tà-w{\^{u}}m} \textit{/} \textit{n{\`{ɨ}}-w{\^{u}}m} ’ten’ > \textit{m{\`{ɨ}}-wúm} \textit{mí-mbáà} ‘twenty’, while \textit{tà-ɡh{\^{u}}m} ’ten’ {\textasciitilde} \textit{m{\`{ɨ}}-ɡhum} ‘twenty’ in another. At the same time, Limbum\il{Limbum} \textit{báː} ‘two’ {\textasciitilde} \textit{{\textsubdot{m}}-báː} ‘twenty’. These patterns (especially the former) are common in the majority of the Bantu languages as well. 

Primary roots for `twenty' are rarely attested. They may go back to the lexical base `man' (e.g. in D30 Komo\il{Komo} \textit{nkpá} \textit{búi} ‘twenty’ = ‘whole person’), `head' (Suga\il{Suga} (Mambiloid)) \textit{ɓʉʉ} \textit{bíb} ‘twenty’ <\textit{ɓʉʉ} ‘head’) or some other lexical bases (e.g. Bantu A50: Bafia\il{Bafia} \textit{{\`{ɨ}}-tín} /\textit{m{\`{ʌ}}-tín} ‘twenty’ < `score')\footnote{John Watters: “The Bakor group of Ekoid attest something like \textit{*-t{\^{e}}n} and Mbe\il{Mbe} has \textit{-t{\^{e}}l}. The other two Ekoid groups have a form \textit{-rim} or \textit{-sam}. I would reconstruct for Proto-Ekoid \textit{*-t{\^{e}}l} or \textit{*-t{\^{e}}n} which is like Bantu Bafia\il{Bafia}. They are a few hundred kilometers apart with many languages and a significant mountain range in between, so this is not borrowing’’ (p.c.).}. 

 
\subsubsection{‘Hundred’ and ‘thousand’} %3.1.1.10.
\begin{table}
\caption{\label{tab:3:12}Bantoid stems for `100'}
\footnotesize
\begin{tabularx}{\textwidth}{ll lll@{}l@{}Ql}
\lsptoprule

% BANTOID & 
~ &   & `100' & `100' & `100' & `100' & `100' & `100' \\
\midrule
Northern\\
\midrule 
Dakoid & Chamba-\il{Chamba}Daka\il{Chamba-Daka} & 20*5 &   &   &   &   &  \\
*Mambiloid &   & 20*5 &   &   &   &   & < fula\\
\tablevspace 

Southern\\
\midrule
\textbf{*Bantu} & \textbf{~} & \textbf{~} & \textbf{~} & \textbf{~} & \textbf{~} & \textbf{kámá}, \textbf{gànà}, \textbf{tʊa}, \textbf{jànda} & \textbf{~}\\
*Beboid &   &   &   & gbi &   &   &  \\
*Yemne-Kimbi &   &  &   & gbi?ŋwe? &   &   &  \\
*Ekoid &   & 20*5 &   &   &   &   &  \\
*Jarawan &   &  & 10*10 &   &   & luru? & < Hausa\\
*Mamfe &   & 20*5 &   &   &   &   &  \\
*Mbam &   &   &   &   &   &   & < Engl.\\
Mbe\il{Mbe} & Mbe\il{Mbe} & 20 *5 &   &   &   &   &  \\
Ndemli\il{Ndemli} & Ndemli\il{Ndemli} &   &   &   &   &  {\`{m}}bókó &  \\
Tikar\il{Tikar} & Tikar\il{Tikar} &   &   &   &   &  nɗuʔ &  \\
*Tivoid &   & 20*5 &   &   &   &   &  \\
*Esimbi\il{Esimbi} &   &   & 10*10 &   &   &   & < Engl\\
Wide Grassfields & Befang\il{Befang} &   &   &   &   &  b{\`{ɔ}}míⁿdáŋɡàŋ &  \\
GF: Mbam-Nkam & Bamileke\il{Bamileke} &   &   &   & k(h)u &   &  \\
GF: Mbam-Nkam & Ngemba\il{Ngemba} &   &   &   & k(h)i/kirə &   &  \\
GF: Mbam-Nkam & Nkambe &   &   &   & ŋk{\`{ʉ}}? & rdʒèe? &  \\
GF: Mbam-Nkam & Nun &   &   &   & ŋku &   &  \\
GF: Momo &   &   &   &   & ki, ko &   &  \\
GF: Ring\il{Ring} &   &   &   &   & ɣ{\'{ɨ}}/vi & ntu? &  \\
\lspbottomrule
\end{tabularx}
\end{table}

It appears that the term for `hundred' cannot be reconstructed for Proto-Bantoid\il{Proto-Bantoid}: in most of the branches the pattern employed is ‘20*5’,\footnote{John Watters: ‘’The distribution of this form is suggestive of an older vigesimal system for Bantoid rather than a decimal one. I would take the decimal ones as innovations’’ (p.c.).} whereas in some of the branches the term is borrowed. Both Grassfields and Bantu show innovations. The Grassfields root may be tentatively reconstructed as \textit{*ku}. Several roots are known for Bantu, their use being limited to certain zones: \textit{kámá} ABCDHL, \textit{gànà} DEFGJNPS, \textit{tʊa} DL, \textit{jànda} MNP. None of these roots is attested with this meaning elsewhere in the Bantoid languages, except for Bantu. The similarity of \textit{kámá} with the root reconstructed for `ten' is noteworthy. Moreover, it is attested with the meaning `thousand' in at least three of the Bantoid branches as the table below shows (\tabref{tab:3:13}).

\begin{table}
\caption{\label{tab:3:13}Bantoid stems for `1000'}
\small
\begin{tabularx}{\textwidth}{QQQl}
\lsptoprule
% BANTOID & 
~ &   & `1000' & `1000' \\
\midrule
Northern\\
\midrule 
Dakoid & Chamba-\il{Chamba}Daka\il{Chamba-Daka} &   & 100*10\\
*Mambiloid &   &   & ndúúŋ `sack', < Fula\\
\tablevspace 

Southern\\
\midrule
\textbf{*Bantu} & \textbf{~} & \textbf{~} & \textbf{nùnù,} \textbf{p{\`{ʊ}}mb{\`{ɩ}},} \textbf{k{\'{ʊ}}t{\`{ʊ}}}\\
*Beboid &   &   & cuku\\
*Yemne-Kimbi &   & kam? & kia? \\
*Ekoid &   &   & 200*5?\\
*Jarawan &   &   & ?\\
*Mamfe &   &   & nka? \\
*Mbam &   &   & < Engl.\\
Mbe\il{Mbe} & Mbe\il{Mbe} &   & 400*2+200\\
Ndemli\il{Ndemli} & Ndemli\il{Ndemli} &   & kòlí \\
Tikar\il{Tikar} & Tikar\il{Tikar} & ŋkæm &  \\
*Tivoid &   &   & 20*10, engl.\\
*Esimbi\il{Esimbi} &   &   & < engl\\
Wide Grassfields & Befang\il{Befang} &   &  {\'{ɪ}}tʃ{\'{ə}}n {\textasciitilde} étʃ{\'{ə}}n\\
GF: Mbam-Nkam & Bamileke\il{Bamileke} &   & tsa /sa? \\
GF: Mbam-Nkam & Ngemba\il{Ngemba} & kamə? & tsuʔu? \\
GF: Mbam-Nkam & Nkambe &  & cuki? \\
GF: Mbam-Nkam & Nun &   & 100*10\\
GF: Momo &   &   & < engl\\
GF: Ring\il{Ring} &   & kam &  \\
\lspbottomrule
\end{tabularx}
\end{table}

The root \textit{kam} allows multiple interpretations. We will return to it after the evidence from other Benue-Congo branches has been examined. 

\clearpage 
The Proto-Bantoid\il{Proto-Bantoid} numeral system can be reconstructed as in \tabref{tab:3:14}.

\begin{table}
\caption[Proto-Bantoid numeral system]{\label{tab:3:14}Proto-Bantoid\il{Proto-Bantoid} numeral system\footnotemark} 
\begin{tabularx}{\textwidth}{rX rl}
\lsptoprule
{1} & m-o-ʔ, m-o-i, m-o-ti, mo-di &  {7} & samba/camba (< *c/saN+2?)\\
{2} & pa/fe, badi (*ba-di?) &  {8} & na-nai (< 4 redupl.)\\
{3} & tat &  {9} & bukV\\
{4} & nai &  {10} & fu, kum/kam\\
{5} & tan &  {20} & 10*2\\
{6} & ta-ta(t) (< 3 redupl.?) &  {100} & gbi? ki? 20*5? kam?\\
&  &  {1000} & ?\\
\lspbottomrule
\end{tabularx}
\end{table}
\footnotetext{My competence does not allow me to reconstruct the tones in the numeral Bantoid languages, especially in Benue-Congo.}

According to Kay Williamson, the base for `one' in Benue-Congo should be reconstructed as \textit{\#-kani}. The only form quoted in support of this hypothesis in her first article (\citealt{Williamson1989b}: 255) is a supposed Bantoid reflex of the root in Tiba\il{Tiba} (\textit{a-kina} ‘1’). Later (\citealt{Williamson1992}: 396) she adduced one more Bantoid form, a Southern Bantoid Esimbi\il{Esimbi} term \textit{keni} ‘1’. That Williamson gives too much weight to these two marginal Bantoid forms is evident from the fact that she reconstructs this base not only for Benue-Congo, but for Niger-Congo as well. This leads her to the idea (probably expressed in the latter work for the first time) that Niger-Congo originally roots had a triconsonantal structure, hence her reconstruction of the proto-form for ‘one’ as \textit{**-‘kə’gəni}. This Niger-Congo etymology will be studied in detail below. At this point we will only note that the Esimbi form cited above is strikingly unusual for the Bantoid languages and was probably misinterpreted. The form \textit{kēn{\={ə}}} ‘1’ is indeed attested in some of the Esimbi sources (see Brad Koenig, \url{https://mpi-lingweb.shh.mpg.de/numeral/Esimbi.htm}). However, in other sources the form \textit{ɔ-nə} is attested (Cristin Kalinowski in (Chan)), so the term for ‘eleven’ is \textit{bùɣù} \textit{nə-nə} (\textit{bùɣù} ‘10’). In other words, the base for ‘one’ in Esimbi is \textit{-ni/-n{\={ə}}} (!), while the first syllable should be interpreted as the noun class prefix, just as in other numerals (cf. the forms \textit{m{\={ə}}rākp{\={ə}}} ‘2’, \textit{mōɲī} ‘4’, \textit{māt{\={ə}}n{\`{ə}}} ‘5’, etc. in Koenig). 

As for Tiba\il{Tiba}, it is still not certain whether this language indeed belongs to the Bantoid group (cf. \citealt{Boyd1999}, where Tiba is considered an Adamawa language). The only Bantoid forms that could have been used by Williamson in support of her hypothesis are found in some of the Northern Mambiloid languages, cf. Twendi\il{Twendi} (Cambap) \textit{tʃínī}, Mambila\il{Mambila} \textit{tʃ{\'{ɛ}}n} (with palatalization assumed). However, these forms are extremely marginal as well, so they cannot give ground for the proto-language reconstruction (in any case, not for Proto-Bantoid\il{Proto-Bantoid}).

 
\subsection{Benue-Congo (the Bantoid languages excluded)}
After the numerals of the Bantoid languages, let's consider the numerals in each of the other groups within this vast family, namely Cross, Defoid, Edoid, Idomoid, Igboid, Jukunoid, Kainji, Platoid, Nupoid (Sections \ref{sec:3.1.2.1}--\ref{sec:3.1.2.9}) and in some isolated BC languages – Ikaan\il{Ikaan}, Akpes\il{Akpes}, Oko\il{Oko} and Lufu\il{Lufu} (Sections \ref{sec:3.1.3.1}--\ref{sec:3.1.3.4}). After this, we will generalize the results obtained in order to try to reconstruct the numerals of Proto-BC (\sectref{sec:3.1.4}).


\subsubsection{Cross}\label{sec:3.1.2.1}
Let us consider the typical stems for numerals in the Cross languages. 

\begin{table}
\caption{\label{tab:3:15}Cross stems for `1'}  
\begin{tabularx}{\textwidth}{lQQll}
\lsptoprule

% Cross &
~ & ‘1’ & ‘1’ & ‘1’ & ‘1’\\
\midrule
1. Bendi\il{Bendi}\\
\midrule 
Bendi\il{Bendi} & ken &   & -b{\'{ɔ}}ŋè? & \\
\tablevspace 

2. Delta-Cross\\
\midrule 
Upper &   & ni (D\footnotemark{}: *gʷá-nì) & w{\`{ɔ}}n, guŋ? & m{\'{ɔ}}{\`{ɔ}}?\\
Central &   & nin &   &  \\
Lower & sin/cin, ki/ge, kiet/keed (D:*cèèd) &   &   &  \\
Ogoni\il{Ogoni} & z{\`{\~i}}{\`{\~i}} & nɛ(n) &   &  \\
\lspbottomrule
\end{tabularx}
\end{table}
\footnotetext{Here and below, index D introduces the reconstruction proposed by \citet{Dimmendaal1978}.}
Let us dwell on this table, using it as an example for understanding the majority of the subsequent tables given in this book. Almost every table represents the \textstyleshorttext{synthesis} of the primary data. We cannot publish all of these primary forms. Let's make an exception. In order to make clear to the reader on what basis the generalizations were made, we present in Appendix D all the forms available for the numerals `1' in the Cross languages, including intermediate Proto-Upper Cross\il{Proto-Upper Cross} and Proto-Lower Cross\il{Proto-Lower Cross} reconstructions, proposed by \citet{Dimmendaal1978} and \citet{Connell1991}. From the Appendix D, it is clear that Connell accepts the Dimmentaal hypothesis, according to which in Upper Cross \textit{*gʷá-} is interpreted as a prefix, and the lexical stem is represented by \textit{*-ni}, attested also in Central Delta-Cross and Ogoni\il{Ogoni}. Based on the 60 sources listed in Appendix D, in table 3.15 for the numeral `1', the root \textit{ni(n)} is allocated. The table also identifies the second root for `1', also possibly represented in the three branches of their five. Connell reconstructs it as \textit{*cèèd}, but the data from various Lower Delta-Cross, as well as from Dendi\il{Dendi}, suggests that perhaps we are dealing with a palatalization of the velar before the front vowel: \textit{*ked} \textit{/} \textit{ket} \textit{/} \textit{kin} > \textit{ced} \textit{/} \textit{cin} (unfortunately, for most groups of the Niger-Congo, including Cross, we do not have sufficient grounds for reconstructing the tones). Finally, the third root presented in Icheve\il{Icheve} \textit{à-mɔɔ} is probably related to Bantu.

\paragraph*{‘Two’ (\tabref{tab:3:16})}
~

\begin{table}
\caption{\label{tab:3:16}Cross stems for `2'.}


\begin{tabularx}{\textwidth}{lXll}
\lsptoprule

% CROSS &
~ & `2' & `2' & `2' \\
\midrule
1. Bendi\il{Bendi}\\
\midrule 
Bendi\il{Bendi} &   & fe, ha? &  \\
\tablevspace 

2. Delta-Cross\\
\midrule
Upper &   & fa(n)/poo (D:*ppán) &  \\
Central &   &   & jal/yal/zal/wal\\
Lower & bà (D:*íbà) &   &  \\
Ogoni\il{Ogoni} & bà{\`{ɛ}}/bɛrɛ &   &  \\
\lspbottomrule
\end{tabularx}
\end{table}

The roots \textit{*bae} and \textit{*po/pa} are noteworthy. 

\newpage 
\paragraph*{‘Three’ and ‘Four’ (\tabref{tab:3:17})}

The common Niger-Congo roots are attested for these numerals in all of the branches (*\textit{ta(t)/} \textit{ca(t)} and \textit{*na(n)} respectively). 

\begin{table}
\caption{\label{tab:3:17}Cross stems for `3' and `4'}
\begin{tabularx}{\textwidth}{lXlXl}
\lsptoprule
% CROSS & 
~ & `3' & `3' & `4' & `4' \\
\midrule
1. Bendi\il{Bendi}\\
\midrule 
Bendi\il{Bendi} & kie/cia/cat &   & ne &  \\

\tablevspace
2. Delta-Cross\\
\midrule 
Upper & tat/tan/*sa/, kia(t) (D: ttán {\textasciitilde} ttáD) & naan? & na (D: *nàŋì {\textasciitilde} này) &  \\
Central & sar/rar &   & ɲa &  \\
Lower & tá (D:*ítá) &   & nàaŋ/nìàŋ (D:*ìnìàŋ) &  \\
Ogoni\il{Ogoni} & taa &   & nia & 3+1\\
\lspbottomrule
\end{tabularx}
\end{table}


\paragraph*{‘Five’ (\tabref{tab:3:18})}

Two roots can be postulated for Cross, namely \textit{*tan} and its alternative, tentatively described as \textit{*gbo(k).}  

\begin{table}
\caption{\label{tab:3:18}Cross stems for `5'}
\begin{tabularx}{\textwidth}{l XXl}
\lsptoprule
% CROSS &
~ & `5' & `5' & `5' \\
\midrule
1. Bendi\il{Bendi}\\
\midrule 
Bendi\il{Bendi} & taŋ &   & dʲoŋ~\\
\tablevspace
2. Delta-Cross\\
\midrule 
Upper & t{\'{ə}}{\'{ə}}n/tāɲ/zen/cen & gbo/buo(k) &  \\
Central &   & oɣ/wʊ? &  \\
Lower & tîŋ/tin/tion, go?(D:*ítíòn) &   &  \\
Ogoni\il{Ogoni} & *r{\`{ɛ}} & ʔòò/vòò/wò/*ʔa &  \\
\lspbottomrule
\end{tabularx}
\end{table}

\newpage
\paragraph*{‘Six’ to ‘Nine’ (\tabref{tab:3:19})}


At this stage it seems reasonable to maintain the forms and patterns represented in the last line of the table. 
\begin{table}
\caption{\label{tab:3:19}Cross stems and patterns for `6'-'9'}
\begin{tabularx}{\textwidth}{l XXl lXl}
\lsptoprule
% CROSS &
~ & `6' & `6' & `6' & `7' & `8' & `9' \\
\midrule
1. Bendi\il{Bendi}\\
\midrule 
Bendi\il{Bendi} & 5+1 &   &   & 5 + 2 & 5 + 3 & 5 + 4\\

\tablevspace
2. Delta-Cross\\
\midrule 
Upper & 5+1 &   & ránē , 3+3 & 5+2, 4+3 & 4+4 & 10-1, 5+4\\
Central &   & di(n) &   & ɗùal/ɗuən & 4PL & súɣó\\
Lower & 5+1 &   &   & 5+2 & 5+3 & 5+4\\
Ogoni\il{Ogoni} & 5+1 & nìʔ{\~{\`i}}? & ʔ{\`{ɔ}}r{\`{ɔ}}? & 5+2 & 5+3 & 10-1, 5+4\\
\textbf{CROSS} & \textbf{5+1} & \textbf{diʔ}  & \textbf{3+3} & \textbf{5+2} & \textbf{4+4} & \textbf{10-1,} \textbf{5+4}\\
\lspbottomrule
\end{tabularx}
\end{table}

\paragraph*{‘Ten’, ‘Twenty’, and ‘Hundred’ (\tabref{tab:3:20})}

It should be noted that providing a detailed reconstruction for each of the Cross numerals lies beyond the scope of the present investigation, so there is probably no point in trying to establish which of the roots for ‘ten’ (\textit{*kpo} or \textit{*job}~) should be reconstructed in the Proto-Cross\il{Proto-Cross} (especially impossible without external evidence).
\begin{table}
\caption{\label{tab:3:20}Cross stems and patterns for `10', `20' and `100'}
\small

\begin{tabularx}{\textwidth}{l QQ lQl Q}
\lsptoprule

% CROSS &
~ & `10' & `10' & `20' & `20' & `20' & `100' \\
\midrule
1. Bendi\il{Bendi}\\
\midrule 
Bendi\il{Bendi} & kpu, hwo, fo &   & ci/si &   & jam & 20*5\\
\tablevspace

2. Delta-Cross\\
\midrule
Upper &   & jo(b)/zob/\newline jop (D:*jòb) & ti & lop, nip\newline (D:*níb) & zol … & 20*5\\
\tablevspace
Central &   & ɗ{\`{ɪ}}oβ &   & lisiíβ/rusuβ & poɣ, 2PL & kùròn, 5*20, 80+20\\
\tablevspace
Lower & kɔp\newline (D:*lùgòp) & duob/duop, dugu/lugu &   & e-dip\newline (D: *édíp) &   & i-kie\newline (D: *íkí{\`{ɛ}})\\
\tablevspace
Ogoni\il{Ogoni} & òb, ʔò &   &   &   &  tub/cu & 5*20\\
\textbf{CROSS} & \textbf{kpo} & \textbf{job} & \textbf{ti/} \textbf{ci?} & \textbf{dip?} & \textbf{~} & \textbf{20*5}\\
\lspbottomrule
\end{tabularx}
\end{table}


The Cross languages are highly divergent in regard to numerals (an exception should be made for ‘three’ and ‘four’ which are remarkably stable in Cross, as well as in the other NC branches). However, the forms cited above do not provide sufficient reason to suggest a closer relationship within any randomly selected pair of the Cross branches. Hence, it would be too daring to interpret the roots attested in both of these branches as shared innovations. Let us count the numbers of related numeral forms in different pairs of the Cross branches (\tabref{tab:3:21}).

\begin{table}
\caption{\label{tab:3:21}Number of related numerals in different pairs of the Cross branches}
\begin{tabularx}{\textwidth}{XXXXl}
\lsptoprule
~ & Central & Lower & Ogoni\il{Ogoni} & Upper\\
\midrule
Bendi\il{Bendi} & \textbf{0} & 4 & 4 & 5\\
Central &   & 2 & 2 & 4\\
Lower &  &   & 5 & 4\\
Ogoni\il{Ogoni} &  &  &   & 4\\
\lspbottomrule
\end{tabularx}
\end{table}

This distribution is remarkable with regard to the total absence of shared forms (with the ‘three’ and ‘four’ excluded) between Bendi\il{Bendi} and Central Cross. Keeping this in mind, all of the established alternative roots and patterns can be reserved for a later discussion. At this point the following reconstruction of the Proto-Cross\il{Proto-Cross} numerals can be suggested (\tabref{tab:3:22}).

\begin{table}
\caption{\label{tab:3:22}Numeral system of Proto-Cross\il{Proto-Cross}(*)}
\begin{tabularx}{\textwidth}{lXll}
\lsptoprule
\textbf{1} & *kin/cin, *ni(n), *gboŋ/gwan & \textbf{7} & 5+2\\
\textbf{2} & *bae, *po/pa & \textbf{8} & 4+4\\
\textbf{3} & *ta(t)/ca(t) & \textbf{9} & 10-1, 5+4\\
\textbf{4} & *na(n) & \textbf{10} & *kpo/kop, fo? ʔo? *job\\
\textbf{5} & *tan, *gbo(k) & \textbf{20} & *ti/ci ? dip ?\\
\textbf{6} & 5+1, diʔ, 3+3 & \textbf{100} & 20*5\\
\lspbottomrule
\end{tabularx}
\end{table}

\clearpage
\subsubsection{Defoid}\label{sec:3.1.2.2}
The Defoid branch is relatively compact: it is composed of four languages including Yoruba\il{Yoruba} and its dialects. Historical phonetics of these languages should be considered for a proper reconstruction of the Defoid numeral system, because most of the terms show great phonetic variety. E.g. for ‘four’ several forms are attested: \textit{-nɛ} (Ariɡidi\il{Ariɡidi}), \textit{-j{\~ē}} (Ayere\il{Ayere}), \textit{-rin/-h{\~{ɛ}}/-{\~{ɛ}}} (Yoruba), \textit{-l{\`{ɛ}}} (Igala\il{Igala}). The main forms are given in the following table (\tabref{tab:3:23}), and their reconstruction will be discussed below. 

\begin{table}
\caption{\label{tab:3:23}Defoid numerals}

\small
\begin{tabularx}{\textwidth}{rQQllQQ}
\lsptoprule

~ & Ariɡidi\il{Ariɡidi} (dial.) & Ayere\il{Ayere} (dial.) & Yoruba\il{Yoruba} & Igala\il{Igala} & *Yoruba-\il{Yoruba}Igala\il{Igala} & *Proto-Defoid\\
\midrule 
1 & k{\`{ɛ}}{\'{ɛ}}-ɲ{\~{ɛ}} & {\~{\`i}}-k{\~{\v{a}}} & ē-n{\textsubtilde{í}}, {\`{ɔ}}-k{\textsubtilde{\={ɔ}}} & é-ɲ{\'{ɛ}} /{\v{o}}-kâ & \textbf{*ɲ{\'{ɛ}}} , \textbf{ka(n)} & \textbf{*ɲ{\'{ɛ}}} , \textbf{ka(n)}\\
2 & kè-ji & ì-dʒì & è-jì & è-dʒì & \textbf{*jì} & \textbf{*jì}\\
3 & ke-dà & ī-tā & {\`{ɛ}}-tā & {\`{ɛ}}-ta & \textbf{*tā} & \textbf{*tā}\\
4 & ke-nɛ & {\~ī}-j{\~ē} & {\`{ɛ}}-r{\textsubtilde{ī}} & {\`{ɛ}}-l{\`{ɛ}} & \textbf{*lɛ(n)} & \textbf{*lɛ(n)/} \textbf{ne,} \textbf{je}\\
5 & ké-nt{\`{ɔ}} & {\~ī}-t{\~{\'u}} & à-r{\textsubtilde{ú}} & {\`{ɛ}}-lú & \textbf{*lú(n)} & \textbf{*lú(n)/tu(n)}\\
6 & ke-fà & ì-fà & {\`{ɛ}}-fà & {\`{ɛ}}-fà & \textbf{*fà} & \textbf{*fà}\\
7 & ke-ɸi & ī-dʒʷī & è-jē & è-bʲe & \textbf{*byē} & \textbf{*byē}\\
8 & ke-rò & ī-rō & {\`{ɛ}}-j{\={ɔ}} & {\`{ɛ}}-dʒɔ & \textbf{*j{\={ɔ}}} & \textbf{*jo/} \textbf{ro}\\
9 & ké-ndà & {\~ī}-d{\~{\^a}} & {\`{ɛ}}-s{\textsubtilde{\'{ɔ}}} & {\`{ɛ}}-lá & \textbf{*sá(n)} & \textbf{*sá(n),} \textbf{dà}\\
10 & ké-è & ī-ɡʷá & {\`{ɛ}}-wá & {\`{ɛ}}-ɡʷá & \textbf{*gwá} & \textbf{*gwá}\\
20 & u-ɡbɔr{\`{ɔ}} & ē-ɡb{\={ɔ}}l{\={ɔ}} & ō-g{\textsubtilde{ú}} & ó-ɡʷú & \textbf{*gwú(n)} & \textbf{*gwú(n)}\textbf{/} \textbf{gbolo}\\
100 & 20*5 & 20*5 & 20*5 & 20*5 & \textbf{20*5} & \textbf{20*5}\\
\lspbottomrule
\end{tabularx}
\end{table}

Following the Proto-Yoruba-Igala\il{Proto-Yoruba-Igala} reconstruction (Pozdniakov, ms), the terms \textit{*lɛ(n)} ‘4’, \textit{*lú(n}) ‘5’ and \textit{*sá(n)} ‘9’ are reconstructed on the basis of the following regular phonetic correspondences (\tabref{tab:3:24}).

\begin{table}
\caption{\label{tab:3:24}Fragment of the Yoruba\il{Yoruba}-Igala\il{Igala} phonetic reconstruction}

\begin{tabularx}{\textwidth}{XXl} 
\lsptoprule
& Yoruba\il{Yoruba} & Igala\il{Igala}\\
\midrule
\textbf{*l} & r & l\\
\textbf{*r} & r & d\\
\textbf{*d} & d/j & d\\
\textbf{*n} & l/n & n\\
\textbf{*s} & s & l\\
\textbf{*} \textbf{ʃ} & s & r\\
\textbf{*c} & ʃ & c\\
\lspbottomrule
\end{tabularx}
\end{table}

These examples illustrate the phonetic correspondences coming from *l `(\tabref{tab:3:25}).

\begin{table}
\caption{\label{tab:3:25}*L-stems in Proto-Yoruba-Igala\il{Proto-Yoruba-Igala} and their regular reflexes}
\small
\begin{tabularx}{\textwidth}{XXXl}
\lsptoprule

 {Meaning} 		& *Yoruba-\il{Yoruba}Igala\il{Igala} 	& Yoruba\il{Yoruba} 			& Igala\il{Igala} 	\\
\midrule                                                                                                                         
{animal,}  {meat} 	& {\'{ɛ}}l{\~{ɔ}} 			& ər{\~{ɔ}} 				& {\'{ɛ}}la 		\\
{toad} 			& àkèlé 				& àkèré 				& àkèlé 		\\
{four} 			& {\`{ɛ}}l{\~{ı}} 			& {\`{ɛ}}r{\~{ı}} 			& {\`{ɛ}}l{\`{ɛ}} 	\\
{five} 			& {\`{ɛ}}l{\textsubtilde{ú}} 	& àr{\textsubtilde{ú}} 		& {\`{ɛ}}lu 		\\
{ant} 			& èlìlà 				& èèrà 					& èlìlà 		\\
{ashes} 		& élílú 				& eérú 					& élúlú 		\\
{feel} 			& gb{\'{ɔ}} òlíl{\textsubtilde{ù}} 	& gb{\'{ɔ}} òór{\textsubtilde{ù}} 	& é-gbúlù 		\\
{star} 			& ìlàw{\`{ɔ}} 				& ìràw{\`{ɔ}} 				& ìlàwò 		\\
{small} 		& kékélé 				& kékeré 				& kékélé 		\\
{buy} 			& là 					& rà 					& é-là		 	\\
{see} 			& lí 					& rí 					& é-lí		 	\\
% 
{plow} 	& lo 	& roko 	& é-lo\\
{body} 	& {\'{ɔ}}la 	& ara 	& {\'{ɔ}}la\\
{word}  	& {\`{ɔ}}là 	& {\`{ɔ}}r{\`{ɔ}} 	& {\`{ɔ}}là\\
{sun} 	& ólìl{\textsubtilde{ù}} 	& òòr{\textsubtilde{ù}} 	& ólù\\
{sleep} 	& oólũ 	& oorũ 	& ólu\\
{neck} 	& {\'{ɔ}}l{\textsubtilde{ù}} 	& ɔr{\textsubtilde{ù}} 	& {\'{ɔ}}l{\`{ɔ}}\\
{thirst} 	& òl{\textsubtilde{ù}}gbə 	& òr{\textsubtilde{ù}}gbə 	& òlùgbə\\
{ring} 	& {\'{ɔ}}l{\textsubtilde{ù}}-ìka 	& òrùka 	& èlìka\\
{run} 	& sVlé 	& sáré 	& é-rúlé\\
{fat} 	& ùla 	& {\`{ɔ}}rá 	& ùlà\\
{seed} 	& úlú 	& irú 	& úlú\\
\lspbottomrule
\end{tabularx}
\end{table}

\clearpage 
Yoruba\il{Yoruba} [s] is correspondent to Igala\il{Igala} [r] (< *ʃ) or [l] (< *s) in at least six examples, see \tabref{tab:3:26} below.

\begin{table}
\caption{\label{tab:3:26}Reflexes of *ʃ and *s in Yoruba-Igala}\il{Yoruba}\il{Igala}


\begin{tabularx}{.8\textwidth}{XXXl}
\lsptoprule
Meaning & *Yoruba-\il{Yoruba}Igala\il{Igala} & Yoruba\il{Yoruba} & Igala\il{Igala}\\
\midrule
{leg} & {\'{ɛ}}ʃ{\`{ɛ}} & əs{\`{ɛ}} & {\'{ɛ}}r{\`{ɛ}}\\
{fruit} & èʃo & èso & {\`{ɛ}}ro\\
{block/} {close} & ʃé & sé & é-ré\\
{launch} & ʃɔ & sɔ & é-rɔ\\
{nine} & {\`{ɛ}}s{\textsubtilde{\'{ɔ}}} & {\`{ɛ}}s{\textsubtilde{\'{ɔ}}} & {\`{ɛ}}lá\\
{sleep} & s{\textsubtilde{ù}} & s{\textsubtilde{ù}} & *é-lu-\\
\lspbottomrule
\end{tabularx}
\end{table}

The reconstruction of the term for ‘seven’ (\textit{*byē})  is based on the following correspondences (\tabref{tab:3:27}).

\begin{table}
\caption{\label{tab:3:27}One more fragment of the Yoruba\il{Yoruba}-Igala\il{Igala} regular correspondences}


\begin{tabularx}{.8\textwidth}{XXl} 
\lsptoprule
& Yoruba\il{Yoruba} & Igala\il{Igala}\\
\midrule 
*by & j & by\\
*j & j & j\\
*b & b & b\\
\lspbottomrule
\end{tabularx}
\end{table}

The reflexes of \textbf{*by-} can be represented as follows (\tabref{tab:3:28}).

\begin{table}
\caption{\label{tab:3:28}Reflexes of *by in Yoruba-Igala}\il{Yoruba}\il{Igala}


\begin{tabularx}{.8\textwidth}{lXXl}
\lsptoprule

Meaning & *Yoruba-\il{Yoruba}Igala\il{Igala} & Yoruba\il{Yoruba} & Igala\il{Igala}\\
\midrule
{dog} & abyá & ajá & abyá\\
{blood} & {\`{ɛ}}by{\`{ɛ}} & {\`{ɛ}}j{\`{ɛ}} & {\`{ɛ}}by{\`{ɛ}}\\
{seven} & ebye & èje & ebye\\
\lspbottomrule
\end{tabularx}
\end{table}

\newpage 
Finally, the terms \textit{*gwá} ‘10’ and\textit{*gwú(n)} ‘20’ are reconstructed in view of \textbf{*gw} > Yoruba\il{Yoruba} \textbf{w} (before [a]) /\textbf{g} (before [u]) {\textasciitilde} Igala\il{Igala} \textbf{gw}~ (\tabref{tab:3:29}).  

\begin{table}
\caption{\label{tab:3:29}Reflexes of *gw in Yoruba\il{Yoruba}-Igala\il{Igala}}
\begin{tabularx}{\textwidth}{lXXX}
\lsptoprule
Meaning & *Yoruba-\il{Yoruba}Igala\il{Igala} & Yoruba\il{Yoruba} & Igala\il{Igala}\\
\midrule
{ten} & {\`{ɛ}}gwá & {\`{ɛ}}wá & {\`{ɛ}}gwá\\
{beans} & {\`{ɛ}}gwà & {\`{ɛ}}wà & {\`{ɛ}}gwà\\
{dig} & gwà & wà & é-gwà\\
{swim} & gwà & w{\`{ɛ}} & é-gwà\\
{sweat} & (ò)úgw{\textsubtilde{ù}} & òóg{\textsubtilde{ù}} & úgwù\\
{bone} & égw{\textsubtilde{ú}}gw{\textsubtilde{ú}} & egũgũ & ógwúgwú\\
{ascend} & gw{\textsubtilde{ù}} & g{\textsubtilde{ù}} & é-tə-gwù\\
{war} & ógwũ & ogũ & ógwu\\
{twenty} & ōgw{\textsubtilde{ú}} & ōg{\textsubtilde{ú}} & ó-ɡwú \\
{vulture} & úgw{\textsubtilde{ú}}nú & ig{\textsubtilde{ú}}nug{\textsubtilde{ú}} & úgwúnú\\
\lspbottomrule
\end{tabularx}
\end{table}

These correspondences are treated here in detail because they may be of special interest for the comparative study of the Defoid languages.

\subsubsection{Edoid}\label{sec:3.1.2.3}
The following reconstruction is based on nearly forty sources which represent twenty languages within this group. The reconstruction proposed by Elugbe was also considered.

Being no specialist in the comparative study of the Edoid languages (unlike Elugbe), I don’t feel competent enough to criticize his ideas. Elugbe likely had his reasons for reconstructing the same consonant (\textbf{*ch}-) in the terms for ‘three’, ‘five’, ‘six’ and ‘seven’. Indeed, the comparison of data from the four Edoid branches confirms that the terms for ‘three’ and ‘five’ (but not for ‘seven’) have the same initial consonant. This is common for many of the NC branches (and probably for the Proto-NC\il{Proto-NC} as well).

In view of this, I would like to suggest a simplified reconstruction that is closer, in my opinion, to the actually attested forms (\tabref{tab:3:30}).

\begin{table}
\caption{\label{tab:3:30}Edoid numeral systems and Proto-Edoid\il{Proto-Edoid}}
\small
\begin{tabularx}{\textwidth}{r QQQQQQ}
\lsptoprule
~ & 1. Delta & 2. North-Central & 3. Northwestern & 4. Southwestern & Proto-Edoid\il{Proto-Edoid} 
(Elugbe) & \textbf{*Proto-Edoid}\il{Proto-Edoid}\\
\midrule
1 & βʊ & kpa, wo/gwo & kpa & vʊ &   & \textbf{kpa,} \textbf{wo/gwo/vu}\\
2 & βə/βa & va & va & vɛ & i-və & \textbf{va/və}\\
3 & saa & sa & sa & sa & ɩɩ-chaGɩ & \textbf{sa}\\
4 & ni & ne & ni & ni & niə & \textbf{ni}\\
5 & súwón /syònì & sen /ʃen & sie & soi/siorin/jorin & ii-chiNənhi & \textbf{sien/} \textbf{su(w)on}\\
6 & 3PL & 3+3? & 3+3 & 3PL? & chaN & \textbf{3PL,} \textbf{3+3}\\
7 & 5+2 & hiron/hilon, 5+2 & sie/hi/rhi & ɣwr{\'{ɛ}}/hre & i-chiə & \textbf{ghie?}\\
8 & 4PL, 4 redupl & renren /lelen & nien & re(r)e & nhɩNanhɩ & \textbf{4} \textbf{redupl.}\\
9 & 10-1 & sin(rin), tili & 5+4 & rhi(r)i, zi & i-ciənhi & \textbf{cien/} \textbf{sin}\\
10 & gbeny/gbei & gbe & gbe & kpe/xwe & gbeNi & \textbf{gbe,} \textbf{kpe}\\
20 & jow/yei & gie/je & gboro, ghe/ze/ye & dhe/ɟè/ʒè & u-gheGi {\textasciitilde} u-ɟh & \textbf{gie/} \textbf{jie}\\
100 & 20*5 & 20*5 & 10PL & 20*5 &   & \textbf{20*5}\\
1000 &   & ria /li, gbele & 500*2 & du, riorin &   & \textbf{du,} \textbf{ria/li}\\
\lspbottomrule
\end{tabularx}
\end{table}

\newpage 
\subsubsection{Idomoid}\label{sec:3.1.2.4}
\largerpage[3]
The roots attested in about ten of the Idomoid languages are represented in \tabref{tab:3:31}.

\begin{table}[b!]
\caption{\label{tab:3:31}Idomoid numerals} 
\begin{tabularx}{\textwidth}{lXll}
\lsptoprule 
1 & nze/je/nye/ye, kpokpoh?\footnote{Please note that hypothetically related forms are separated by a slash (/), whereas unrelated ones are separated by a comma.} & 7 & 5+2, renyi\\
2 & pa, miyeh? & 8 & 5+3\\
3 & ta/la & 9 & 5+4\\
4 & n{\`{ɛ}}, ndo, he & 10 & gwo/wo, jwo\\
5 & do/lo, ho, ro/rwo & 20 & fu/hu, su\\
6 & rowo/riwi, ji, hili & 100 & 20*5, 10*10\\
\lspbottomrule
\end{tabularx}
\end{table}
\clearpage 

It should be noted that the data on the Yatye-Akpa branch (one of the two Idomoid branches) is systematically absent. The analysis is based on the Akweya languages only, so unexpected issues may arise.

 
\subsubsection{Igboid}\label{sec:3.1.2.5}
This is a small group consisting of several languages. The forms which could be found in modern Igboid languages are listed in \tabref{tab:3:32}.

\begin{table}
\caption{\label{tab:3:32}Igboid numerals}
\begin{tabularx}{.8\textwidth}{lXrl}
\lsptoprule
1 & tù, ŋìn{\'{ɛ}} (Ekpeye)\il{Ekpeye}? & 7 & saà\\
2 & b{\'{ɔ}} & 8 & 5+3\\
3 & t{\'{ɔ}} & 9 & totu /tolu \\
4 & n{\'{ɔ}} & 10 & ɗì/ri/li\\
5 & sé & 20 & ɡw{\'{\~ʊ}} /ɣʰ{\={ʊ}}, kpɔrɔ\\
6 & ʃ{\H{i}}i & 100 & 20*5\\
~ &   & 1000 & puk(w)u\\
\lspbottomrule
\end{tabularx}
\end{table}

Interestingly, the terms for ‘one’ attested in the Igboid languages (as found in \citealt{Koelle1963}) are subject to significant variation. The following forms are noteworthy: ‘1’ – \textstylehelp{{\={I}}sóāma} \il{Īsóāma}\textstylefun{\textit{oo-te}}\textstylefun{,} \textstylehelp{{\'{I}}ṣiēle} \il{\'{I}ṣiēle}\textstylefun{\textit{mfuu}}\textstylefun{,} \textstylehelp{{\'{A}}bādṣa} \il{\'{A}bādṣa}\textstylehelp{\textit{na}}\textstylehelp{, Aro}\il{Aro}\textstylehelp{} \textstylefun{\textit{mbɔ}}\textstylefun{,} \textstylehelp{Mb{\textsubbar{ó}}fīa} \il{Mbọfīa}\textstylefun{\textit{mpoŋ}} (the transcription of the forms and languages follows Koelle). The rest of the numerals quoted by Koelle are essentially the same as the ones found in \tabref{tab:3:33}.

\clearpage
\subsubsection{Jukunoid}\label{sec:3.1.2.6}
\begin{table}
\caption{\label{tab:3:33}Jukunoid numerals}


\begin{tabularx}{\textwidth}{rllQQ}
\lsptoprule

~ & 1. Bete\il{Bete} (Juk.)\il{Bete (Juk.)} & 2. Central & 3. Yukuben-\il{Yukuben}Kuteb\il{Kuteb} & Proto-Jukunoid\il{Proto-Jukunoid}\\
\midrule
1 & ʃ{\'{ɪ}}ʃe & (d)zun/(d)zuŋ & nzo, ji?, yʊn?, ŋɡēmé?, t{\'{ə}}ŋ? & *d)zun? ʃ{\'{ɪ}}ʃe? t{\'{ə}}ŋ? \\
2 & há & pye(na) & pa(n)/fa(n) & *pa(n) /fa(n)\\
3 & tà & (t)sara & ta & *ta\\
4 & ɲè & nye(na) & ɲi, nje/nzì & *nye\\
5 & tsòŋ & (t)swa(na) & t(s)oŋ & *tsoŋ\\
6 & 5+1 & 5+1 & 5+1 & *5+1\\
7 & 5+2 & 5+2 & 5+2 & *5+2\\
8 & 5+3 & 4 redupl., 5+3 & 5+3 & *4 redupl., 5+3\\
9 & 5+4 & 5+4? & 5+4 & *5+4\\
10 & wo & dub (< Hausa?), dz(w)e & kur? kuwub, bji/bzi, jwēr & *jwe, wo? kur?\\
20 & ? & 'body' (á-dì) & kam /k(w)om & *’body' (di)\\
100 & ? & 20*5 & 20*5, Hausa & *20*5\\
1000 & ? & < Hausa & Hausa & < Hausa\il{Hausa}\\
\lspbottomrule
\end{tabularx}
\end{table}

Tentative reconstructions for the three major branches of this relatively small family are presented in the table above. The terms for ‘one’ and ‘ten’ vary significantly.

 
\subsubsection{Kainji}\label{sec:3.1.2.7}
The comparative analysis of the Kainji group is hindered by the fact that there is no linguistic description for the majority of its languages. However, there is a great range in numerical terms within those languages, for which reliable data is available. The following analysis is based on thirty pertinent sources, including the comparative list of forms compiled by \citet{Dettweiler1993}. What follows is a step-by-step analysis of the available data that will hopefully yield some answers.   

\subsubsubsection{‘One’}\label{sec:3.1.2.7.1}

\begin{table}
\caption{Kainji stems for `1'}
\begin{tabularx}{\textwidth}{rX llll}
\lsptoprule 
% Kainji &
~ &  Language & `1' & `1' & `1' & `1' \\
\midrule
Eastern\\
\midrule
Jera & Iguta\il{Iguta} &   &   & dínkā &  \\
Jera & Janji\il{Janji} &   &   & diŋkɛ & ɪnde\\
Jera & Bunu\il{Bunu} &   & ù-{\`{ŋ}}ŋínì & díŋkà &  \\
Jera & Buji\il{Buji} &   &   & díŋkà &  \\
Amo\il{Amo} & Amo\il{Amo} &   &   & *lu-ruŋ &  \\
\tablevspace 

Western\\
\midrule 
Basa\il{Basa} & Basa\il{Basa} & h{\~{i}}n &   &   &  \\
Duka\il{Duka} & C'lela & tʃ{\~{\'i}} &   &   &  \\
Duka\il{Duka} & Hun-Saare(\il{Hun-Saare}Duka)\il{Duka} & cɔɔn &   &   &  \\
Duka\il{Duka} & Ut-Ma'in\il{Ut-Ma'in} & tʃ{\={ɘ}}ːn &   &   &  \\
Duka\il{Duka} & Rijau\il{Rijau} & tʃoon &   &   &  \\
Duka\il{Duka} & Darangi\il{Darangi} & tʃooɾ &   &   &  \\
Duka\il{Duka} & Bunu\il{Bunu} & dɨɨ &   &   &  \\
Duka\il{Duka} & Iri\il{Iri} & dən &   &   &  \\
Duka\il{Duka} & Dukku\il{Dukku} & dɛn &   &   &  \\
Duka\il{Duka} & Giro\il{Giro} & dɨɨn &   &   &  \\
Kambari & Tsishingini\il{Tsishingini} (Kambari) &   & íyyán &   &  \\
Kambari & Agaushi\il{Agaushi} (Tsikimba) &   &   &   & `-tè\\
Kambari & Kambali\il{Kambali} (Koelle) &   & ííɲa &   &  \\
Kamuku & Western Acipa\il{Acipa} (Cicipu) &   &   &   & t{\^{o}}ː\\
Kamuku & Kamuku (dial.) &   & {\~{\'i}}j{\'{ɑ}} &   &  \\
Kamuku & Hungworo\il{Hungworo} (Hungwere) &   & {\~{\'i}}ːj{\~{\'ə}} &   &  \\
Kamuku & Pongu\il{Pongu} (Pangu) & h{\~{\'i}}ː &   &   &  \\
Kamuku & Kamuku (Koelle) & h{\textsubtilde{í}}{\textsubtilde{í}}a &   &   &  \\
Kamuku & Fungwa\il{Fungwa} & h\~\i &   &   &  \\
Reshe\il{Reshe} & Reshe\il{Reshe} (Tsureshe) & tsúnn{\`{ɛ}} &   &   &  \\
\lspbottomrule
\end{tabularx}
\end{table}

The grouping principles for the forms included in this table are admittedly haphazard. On the one hand, the relationship between some of the forms arranged into the same column (e.g. \textit{h{\~{i}}n,} \textit{tʃ{\={ɘ}}ːn} and \textit{dɛn} or \textit{dínkā} and\textit{*lu-ruŋ}) is not immediately apparent. On the other hand, some of the forms placed in separate columns might be etymologically related (e.g. \textit{dɨɨn} Giro\il{Giro} and \textit{dínkā} Iguta\il{Iguta}). In these circumstances it seems reasonable to go back to the reconstruction of the Kainji term for ‘one’ on the basis of the data provided by other Benue-Congo branches (see \sectref{sec:3.1.4}).

\subsubsubsection{‘Two’}\label{sec:3.1.2.7.2}
\begin{table}[b!]
\caption{\label{tab:3:35}Kainji stems for `2'}
\footnotesize
\begin{tabularx}{\textwidth}{llQQll}
\lsptoprule

% Kainji & 
~ &   & `2' & `2' & `2' & `2' \\
\midrule
Eastern\\
\midrule 
Jera & Iguta\il{Iguta} &   &   & r{\`{ɛ}}ːpú &  \\
 Jera & Janji\il{Janji} &   & tɪ-rɛ \mbox{({\textasciitilde}wa-{\textasciitilde}a-)} & -rèèpó &  \\
Jera & Bunu\il{Bunu} &   &   &   &  \\
Jera & Buji\il{Buji} &   &   & rèpó &  \\
Amo\il{Amo} & Amo\il{Amo} &   &   &   & im-ba\\
\tablevspace

Western\\
\midrule 
Basa\il{Basa} & Basa\il{Basa} & jèbí\newline\mbox{(yééwi)} &   &   &  \\
Duka\il{Duka} & C'lela &   & ʔíl{\`{ɨ}} &   &  \\
Duka\il{Duka} & Hun-Saare(\il{Hun-Saare}Duka)\il{Duka} &   & yoor &   &  \\
Duka\il{Duka} & Ut-Ma'in\il{Ut-Ma'in} &   & j{\={ɘ}}ːr &   &  \\
Duka\il{Duka} & Rijau\il{Rijau} &   & jooɾ &   &  \\
Duka\il{Duka} & Darangi\il{Darangi} &   & jooɾ &   &  \\
Duka\il{Duka} & Bunu\il{Bunu} &   & jɔɔɾ &   &  \\
Duka\il{Duka} & Iri\il{Iri} &   & jooɾ &   &  \\
Duka\il{Duka} & Dukku\il{Dukku} &   & juuɾ &   &  \\
Duka\il{Duka} & Giro\il{Giro} &   & jooɾ &   &  \\
Kambari & Tsishingini\il{Tsishingini} (Kambari) &   & ì-ɾ{\`{ɛ}} &   &  \\
Kambari & Agaushi\il{Agaushi} (Tsikimba) &   & -rè &   &  \\
Kambari & Kambali\il{Kambali} (Koelle) &   & íí-lɛ &   &  \\
Kamuku & Western Acipa\il{Acipa} (Cicipu) & jápù &   &   &  \\
Kamuku & Kamuku (dial.) & ⁿd{\'{ə}}ɰ{\`{ə}} &   &   &  \\
Kamuku & Hungworo\il{Hungworo} (Hungwere) &   & ʔʲ{\~{\^ə}}-dʒ{\`{ə}} &   &  \\
Kamuku & Pongu\il{Pongu} (Pangu) &   & ɾ{\^{e}}ːnù &   &  \\
Kamuku & Kamuku (Koelle) &   &   &   & wúúlee\\
Kamuku & Fungwa\il{Fungwa} & jó:gó &   &   &  \\
Reshe\il{Reshe} & Reshe\il{Reshe} (Tsureshe) &   &   &   & rìs{\={ə}}\\
\lspbottomrule
\end{tabularx}
\end{table}

\largerpage[2]
The above considerations regarding the term for ‘one’ are applicable to the term for ‘two‘ as well. The inventory of forms found in \tabref{tab:3:35} is 
\pagebreak
neither helpful 
for the reconstruction of the Proto-Kainji\il{Proto-Kainji} term for ‘two’, nor suggestive of the morphemic analysis of the pertinent forms within each of the branches. As we hope to demonstrate below, additional information that may prove useful for the reconstruction of the term for ‘two’ can be obtained through the analysis of the term for ‘seven’.

  
\subsubsubsection{‘Three’, ‘Four’ and ‘Five’}\label{sec:3.1.2.7.3}
\begin{table}[h!]
\caption{\label{tab:3:36}Kainji stems for `3'-'5'}
\small

\begin{tabularx}{\textwidth}{lQllll}
\lsptoprule

% Kainji & 
~ &   & `3' & `4' & `5' & `5' \\
\midrule
Eastern\\
\midrule 
Jera & Iguta\il{Iguta} & tààr{\={u}} & nàːnzī &   & ʃùːbì\\
Jera & Janji\il{Janji} &   & tɪ-naze &   & ʧibi\\
Jera & Bunu\il{Bunu} &   & nà:zé &   & ʃí:bì\\
Jera & Buji\il{Buji} &   & nàzé &   & ʃíbí\\
Amo\il{Amo} & Amo\il{Amo} &   & nnas & n-ntaun &  \\
\tablevspace

Western\\
\midrule 
Basa\il{Basa} & Basa\il{Basa} & tàtɔ & néʃì (nááʃii) & táná &  \\
Duka\il{Duka} & C'lela & t{\'{ɨ}}ːt͡ʃù & náːsé & t{\~{\'a}} &  \\
Duka\il{Duka} & Hun-Saare(\il{Hun-Saare}Duka) \il{Duka} & tett & náss & táán~ &  \\
Duka\il{Duka} & Ut-Ma'in\il{Ut-Ma'in} & t{\={ɘ}}t & náːs & tán &  \\
Duka\il{Duka} & Rijau\il{Rijau} & tɪtʰ & nəss & taan &  \\
Duka\il{Duka} & Darangi\il{Darangi} & tɪtʰ & nas & taan &  \\
Duka\il{Duka} & Bunu\il{Bunu} & tɪtʰ & nas & tan &  \\
Duka\il{Duka} & Iri\il{Iri} & tɪɪt & nass & taan &  \\
Duka\il{Duka} & Dukku\il{Dukku} & tɨɨt & nas & taan &  \\
Duka\il{Duka} & Giro\il{Giro} & tɨtʰ & nass & taan &  \\
Kambari & Tsishingini\il{Tsishingini} (Kambari) & tàʔàtsú & n{\'{ə}}{\downstep}ʃín & táː{\downstep}wún &  \\
Kambari & Agaushi\il{Agaushi} (Tsikimba) &   & `-n{\'{ə}}ʃì & `-t{\'{ã}}\~u &  \\
Kambari & Kambali\il{Kambali} (Koelle) & tááatsu & nóóʃin & táá{\textsubbar{u}} &  \\
Kamuku & Western Acipa\il{Acipa} (Cicipu) & tâːtù & nósì & t{\~{\^a}}u &  \\
Kamuku & Kamuku (dial.) & t{\'{ɑ}}t{\`{ɔ}} & n{\'{ə}}ʃì & t{\'{ɑ}}{\`{ɑ}} &  \\
Kamuku & Hungworo\il{Hungworo} (Hungwere) & tâ{\textseagull{t}}{\`{ɔ}}~ & ùn{\'{ə}}s{\~{\`i}} & sàtá &  \\
Kamuku & Pongu\il{Pongu} (Pangu) & tâːtù & n{\~{\'ə}}ːʃ{\~{\`i}} & tá &  \\
Kamuku & Kamuku (Koelle) & tááto & náʃii & taa {\textasciitilde} tááa &  \\
Kamuku & Fungwa\il{Fungwa} &   & nó:ʃì & tá &  \\
Reshe\il{Reshe} & Reshe\il{Reshe} (Tsureshe) & tàtswā & nāʃ{\~{\'e}} & t{\~{\={ɔ}}} &  \\
\lspbottomrule
\end{tabularx}
\end{table}

Unlike the terms for ‘one’ and ‘two’, the numerals covering the sequence from ‘three’ to ‘five’ are quite homogeneous and thus can be reliably reconstructed (just as in the majority of other NC branches). The provisional forms suggested for ‘three’, ‘four’, and ‘five’ are \textit{*tat}, \textit{*nas,} and \textit{*tan} respectively. The latter form can also be reconstructed for Eastern Kainji on the basis of the Amo\il{Amo} evidence. Thus  \textit{ʧibi} (\textit{ʧi-bi}?) ‘five’ is an innovation of the Jera subgroup.

\subsubsubsection{‘Six’ and ‘Seven’}\label{sec:3.1.2.7.4}
\begin{table}
\caption{\label{tab:3:37}Kainji stems and patterns for `6'-'7'}
\fittable{
\footnotesize
\begin{tabular}{lll lllll ll} 
\lsptoprule 
% & Kainji
&   &   & `1' & `2' & `5' & `6' & `7' & `7' \\
\midrule
& Eastern\\
\midrule 
1 & Jera & Iguta\il{Iguta} &   &   &   & twàːsì &   & súnāːrí\\
2 & Jera & Janji\il{Janji} &   & tɪ-rɛ &   & tase &   & sunare\\
3 & Jera & Bunu\il{Bunu} &   &   &   & tá:sè {\textasciitilde}tà:sé &   & súnà:ré\\
4 & Jera & Buji\il{Buji} &   &   &   & tásé &   & súnàrí\\
5 & Amo\il{Amo} & Amo\il{Amo} &   &   & n-ntaun & ku-toʧin & kuzor &  \\
\tablevspace 

 & Western\\
\midrule
6  & Basa\il{Basa} & Basa\il{Basa} & h{\~{i}}n &   & táná & tʃìhin & tʃéndʒe &  \\
7  & Duka\il{Duka} & C'lela & tʃ{\~{\'i}} & *ʔí-l{\`{ɨ}} & t{\~{\'a}} & t͡ʃíh{\~{\`i}} & t{\~{\`a}}ʔíl{\`{ɨ}} &  \\
8  & Duka\il{Duka} & Hun-Saare\il{Hun-Saare} & c{\textsubbar{o}}{\textsubbar{o}}n & * yoo-r & táán~ & cînd & tá'yoor &  \\
9  & Duka\il{Duka} & Ut-Ma'in\il{Ut-Ma'in} & tʃ{\={ɘ}}ːn & *j{\={ɘ}}ː-r & tán & ʃìʃìn & tàʔèr &  \\
10 & Duka\il{Duka} & Rijau\il{Rijau} & tʃoon & *joo-ɾ & taan & tʃiin & ta’jooɾ &  \\
11 & Duka\il{Duka} & Darangi\il{Darangi} & tʃooɾ & *joo-ɾ & taan & tʃin & taŋ’joɾ &  \\
12 & Duka\il{Duka} & Bunu\il{Bunu} & dɨɨ & *jɔɔ-ɾ & tan & tʃiin & ta’juu &  \\
13 & Duka\il{Duka} & Iri\il{Iri} & dən & *joo-ɾ & taan & tʃinnd & ta’jooɾ &  \\
14 & Duka\il{Duka} & Dukku\il{Dukku} & dɛn & *juu-ɾ & taan & tʃɪŋ & ta’jaaɾ &  \\
15 & Duka\il{Duka} & Giro\il{Giro} & dɨɨn & *joo-ɾ & taan & tʃind & ta’jooɾ &  \\
16 & Kambari & Tsishingini\il{Tsishingini} &   & ì-ɾ{\`{ɛ}} & táːwún & t{\`{ə}}ːlí & tʃìnd{\`{ɛ}}ɾ{\'{ɛ}} &  \\
17 & Kambari & Agaushi\il{Agaushi} & -tè & -rè & -t{\'{ã}}\~u & -t{\`{ə}}:lì & ʧìndèrè &  \\
18 & Kambari & Kambali\il{Kambali} &   & íí-lɛ, *rɛ & táá{\textsubbar{u}} & t{\'{ɔ}}{\'{ɔ}}li & tsíndɛɛrɛ &  \\
19 & Kamuku & West.Acipa\il{Acipa} &   &  *jà & t{\~{\^a}}u & tóɾíh{\~{\`i}}~ & tíndàjà &  \\
20 & Kamuku & Cinda\il{Cinda} &   &  *ɰ{\`{ə}} & t{\'{ɑ}}{\`{ɑ}} & t{\'{ə}}n{\'{ə}}hì & t{\'{ə}}nd{\'{ə}}ɰ{\`{ə}} &  \\
21 & Kamuku & Hungworo\il{Hungworo} &   & ʔʲ{\~{\^ə}}-dʒ{\`{ə}}, *ɾʲ{\={ə}} & sàtá & {\={u}}-{\textseagull{t}}únìh{\~ī} & {\={u}}-t{\'{ə}}nd{\`{ə}}ɾʲ{\={ə}} &  \\
22 & Kamuku & Pongu\il{Pongu} & h{\~{\'i}}ː & ɾ{\^{e}}ːnù, *ɾ{\`{ə}} & tá & tʃíníhì & t{\~{\'ə}}nd{\'{ə}}ɾ{\`{ə}} &  \\
23 & Kamuku & Kamuku & h{\textsubtilde{í}}{\textsubtilde{í}}a & *lee & taa {\textasciitilde} tááa & túnui & tandálee &  \\
25 & Kamuku & Fungwa\il{Fungwa} & h\~\i &  *lò & tá & ʧ\~{í}h\~{ì} & tíndàlò &  \\
25 & Reshe\il{Reshe} & Reshe\il{Reshe} & tsúnn{\`{ɛ}} &   & t{\~{\={ɔ}}} & tēnz{\={ɔ}} & tàns{\~ā} &  \\
\lspbottomrule
\end{tabular}
}
\end{table}

\newpage 
Some of the previously discussed terms for ‘one’, ‘two’ and ‘five’ are quoted in the table above alongside the terms for ‘six’ and ‘seven’. Such grouping might facilitate a better understanding of compound numerals (if ‘six’ and ‘seven’ are indeed compounds) as well as the methodological and theoretical aspects behind their reconstruction. In addition, it might help to establish whether parts of compound numerals can be used to enhance the reconstruction of the primary numerical terms such as ‘one’, ‘two’, and ‘five’.

The compound nature of the term for ‘seven’ is betrayed by its ‘length’: the forms quoted in the table normally have two to three syllables, whereas the primary numerals are as a rule mono- or (rarely) bisyllabic.

At the same time, in some of the cases the pattern ‘7=5+2’ is immediately apparent (cf. languages 7-11, 13-15). 

At this point, however, we will deal with those languages that show only faint (or no) traces of the pattern in question (‘7=5+2’). E.g. in Tsishingini\il{Tsishingini} (16) we have to assume the pattern ‘7=X+2’, where ‘X’ is an unknown element, whereas in language 12 the pattern is ‘7=5+X’ (the relationship between ‘X’ and the term for ‘two’ is questionable). 

Let us assume that the Proto-Kainji\il{Proto-Kainji} terms for ‘two’ and ‘five’ are *CL\textbf{-re}  (cf. e.g. Duka\il{Duka}\textit{*jo-re} > \textit{joor}) and *\textit{tan} respectively. In this case, the compound term for ‘seven’ would be \textit{*tan-(}CL)\textit{-re} or *\textit{tan-X} (connector)-(CL)\textit{-re}. The most typical diachronic scenarios for the emergence of the ‘X’-patterns effective on the synchronic level are as follows:

\begin{enumerate}
\item Both basic elements of the compound ‘seven’ (i.e. reflexes of the terms for ‘two’ and ‘five’) are preserved in the language, as is the compound itself (sometimes slightly modified in accordance with the relevant phonotactic rules). Cf. e.g. the Darangi\il{Darangi} (11) evidence: \textit{*jo-re} > \textit{joor} ‘2’, \textit{*tan} > \textit{taan} ‘5’, \textit{*taan-jo-re} > \textit{taŋ’joɾ} ‘7’. In this case, the reconstruction comes down to the simple statement that in the Darangi language ‘7=5+2’.
\item The compound ‘seven’ (even if slightly modified) is preserved in the language, while the term for ‘two’ is replaced with an innovation. Let us assume that in the Basa\il{Basa} language (6) \textit{jèbí} (Koelle: \textit{yééwi}) ‘2’ <\textit{*jo-bi} (innovation), \textit{táná} ‘5’ (the reflex of \textit{*tan}), \textit{tʃéndʒe} \textit{<} \textit{*tan-re} ‘7’. In this case, *\textit{tan-re} > \textit{tan-dʒe} > \textit{tendʒe} (regressive assimilation) > \textit{tʃendʒe} (palatalization before the front vowel). Hypothetical as it may be, this example is phonetically plausible.
\end{enumerate}

\newpage 
Any of these model processes may result in the loss of phonetic resemblance between a derived form and its source. This may lead to a situation where a derivation pattern is no longer recognizable by speakers. As a consequence, the term for ‘seven’ becomes opaque on the synchronic level and can no longer be analysed as ‘5+2’.

This means that the replacement of the original term for ‘two’ by an innovation does not affect the compound term for ‘seven’, i.e. that its second part is not automatically replaced. Moreover, in case there is sufficient evidence that the second of the aforementioned scenarios was applied, we may enhance the reconstruction of the primary term for ‘two’ on the basis of the compound term for ‘seven’. E.g. the form \textit{tʃéndʒe} suggests that the original Basa\il{Basa} root for ‘two’ was \textit{*dʒe} \textit{/} \textit{re} and not \textit{*bi} as in the majority of the Kainji languages.

The available pertinent forms point toward the reconstruction of the Proto-Kainji\il{Proto-Kainji} form as  \textit{*tan-da-re} (‘5’-connector-‘2’). The reconstructed forms for ‘two’ (marked with [*] in \tabref{tab:3:37}) suggest a Proto-Kainji form \textit{*re} ‘2’ and the pattern *’7=5+2’. The Eastern Kainji forms for ‘seven’ are probably innovations.

However, some of the forms attested for ‘seven’ may point toward the reconstruction of ‘two’ as \textit{*ba/bi} in Proto-Kainji\il{Proto-Kainji}. In this case our reference list should be expanded by adding dialects that were not included for reasons of space: it is not possible to quote every single NC source every time. E.g. Cawai\il{Cawai} (Eastern Kainji) \textit{a-ba} ‘2’, \textit{a-tar-ba} ‘7’, Ngwoi\il{Ngwoi} (Hungworo\il{Hungworo}) \textit{e-bia} ‘2’, \textit{sa-bia} ‘7’ (the root \textit{*ba/} \textit{bi} is also suggested by Eastern: Gure\il{Gure} \textit{pi-ba}, Gyem\il{Gyem} \textit{ve}, Piti\il{Piti} \textit{ba}, Surubu\il{Surubu} \textit{ka-va}).

The forms  for ‘six’ are more problematic since they may go back to a primary root (or roots). They may be tentatively reconstructed as \textit{*ci(hi)n,} \textit{*tas,} and \textit{*tel}. We will come back to these forms in order to enhance their reconstruction in case similar forms are detected in other BC branches. 

  
\subsubsubsection{‘Eight’}\label{sec:3.1.2.7.5} 
\begin{table}
\caption{\label{tab:3:38}Kainji stems and patterns for `8'} 
\begin{tabularx}{\textwidth}{ll Xll}
\lsptoprule
% Kainji &
~ &   & `8' & `8' & `8' \\
\midrule
Eastern\\ 
\midrule 
Jera & Iguta\il{Iguta} & ùr{\={u}} &   &  \\
Jera & Janji\il{Janji} & uro &   &  \\
Jera & Bunu\il{Bunu} & ùrú &   &  \\
Jera & Buji\il{Buji} & úrú &   &  \\
Amo\il{Amo} & Amo\il{Amo} &   &   & kuliv\\
\tablevspace

Western\\
\midrule
Basa\il{Basa} & Basa\il{Basa} &   & tɔndatɔ (5+3) &  \\
Duka\il{Duka} & C'lela & j{\'{ɨ}}ːɾù &   &  \\
Duka\il{Duka} & Hun-Saare(\il{Hun-Saare}Duka)\il{Duka} & yéér~ &   &  \\
Duka\il{Duka} & Ut-Ma'in\il{Ut-Ma'in} & éːr &   &  \\
Duka\il{Duka} & Rijau\il{Rijau} & eeɾ &   &  \\
Duka\il{Duka} & Darangi\il{Darangi} & eɾ &   &  \\
Duka\il{Duka} & Bunu\il{Bunu} & ɛɛɾ &   &  \\
Duka\il{Duka} & Iri\il{Iri} & ɪɪɾ &   &  \\
Duka\il{Duka} & Dukku\il{Dukku} & ɛɛɾ &   &  \\
Duka\il{Duka} & Giro\il{Giro} & ɛɛɾ &   &  \\
Kambari & Tsishingini\il{Tsishingini} (Kambari) &   &   & kùnl{\`{ə}}\\
Kambari & Agaushi\il{Agaushi} (Tsikimba) &   &   & kúnl{\`{ə}}i\\
Kambari & Kambali\il{Kambali} (Koelle) &   &   & kúnlo\\
Kamuku & Western Acipa\il{Acipa} (Cicipu) &   &   & kùrílːò\\
Kamuku & Kamuku (dial.) &   & t{\'{ə}}nt{\'{ɑ}}t{\`{ɔ}} (5+3) &  \\
Kamuku & Hungworo\il{Hungworo} (Hungwere) &   & {\={u}}-tátà{\textseagull{t}}{\={ɔ}} (5+3) &  \\
Kamuku & Pongu\il{Pongu} (Pangu) &   & t{\'{\~ə}}ndáːtù (5+3) &  \\
Kamuku & Kamuku (Koelle) &   & túndaat (5+3) &  \\
Kamuku & Fungwa\il{Fungwa} &   & tíndátù (5+3) &  \\
Reshe\il{Reshe} & Reshe\il{Reshe} (Tsureshe) &   & dálànz{\`{ɔ}} &  \\
\lspbottomrule
\end{tabularx}
\end{table}
The Eastern Kainji and Duka\il{Duka} forms (if related) suggest that the primary root \textit{*-ru} should be reconstructed for ‘eight’ in Proto-Kainji\il{Proto-Kainji}. At this point, let us reserve a preliminary form *\textit{u-ro/} \textit{ji-ru} for further comparison. In most of the Kamuku languages the pattern ‘8=5+3’ is traceable (but note the Western Acipa\il{Acipa} form that is comparable to those attested in Kambari and possibly Amo\il{Amo} (Eastern)). This points towards an alternative form of uncertain morphological structure (\textit{*kunle(v)/} \textit{kunlo} ‘8’).
% \todo[inline]{check text is not centered}
  
\newpage   
\subsubsubsection{‘Nine’ and ‘Ten’}\label{sec:3.1.2.7.6}

%MOVED THIS TABLE UP OUT OF CHRONOLOGICAL ORDER
\begin{table}[b]
\caption{\label{tab:3:41}Kainji {summarized data for BC reconstruction}}
\begin{tabularx}{\textwidth}{rXrl}
\lsptoprule
1 & *tsin, hin, din, jan/yan, *te … & 7 & *5+2\\
2 & *re, *ba/bi, -pu? & 8 & *ro/ru, *5+3, *kunle(v)/kunlo\\
3 & *tat & 9 & *5+4, *10-1, *jiro\\
4 & *nas & 10 & *pwa, *turu, *kuri, *kup/kpa\\
5 & *tan & 20 & *10*2, *ʃín/ʃík\\
6 & *ci(hi)n, *tas (< 3?), *tel & 100 & ?\\
\lspbottomrule
\end{tabularx}
\end{table}


\begin{table}
\caption{\label{tab:3:39}Kainji stems and patterns for `9' and `10'}
\footnotesize
\begin{tabularx}{\textwidth}{lQ QlllQ}
\lsptoprule
% Kainji & 
~ &   & `9' & `9' & `9' & `10' & `10' \\
\midrule
Eastern\\
\midrule 
Jera & Iguta\il{Iguta} &   & t{\`{ɔ}}rb{\`{ɔ}} (10-1) &   &   & b{\={u}}-túːrú\\
Jera & Janji\il{Janji} &   & toroəi (10-1) &   &   & turo, kɪrəu\\
Jera & Bunu\il{Bunu} &   & tò:rêj (10-1) &   &   & bì-tú:rú; rú-kúrí\\
Jera & Buji\il{Buji} &   & toroj (10-1) &   &   & bì-túrú; rì-kùrì\\
Amo\il{Amo} & Amo\il{Amo} &   & ku-tivi &   &   & ku-lidir *li-kure\\
\tablevspace 

Western\\
\midrule 
Basa\il{Basa} & Basa\il{Basa} & tʃíndʒìʃì (5+4) &   &   & u{\'{m}}pwá &  \\
Duka\il{Duka} & C'lela &   &   & dóːɾè & ʔóːpá &  \\
Duka\il{Duka} & Hun-Saare(\il{Hun-Saare}Duka)\il{Duka} &   &   & jír{\textsubbar{ò}} & {\textsubbar{o}}pp &  \\
Duka\il{Duka} & Ut-Ma'in\il{Ut-Ma'in} &   &   & dʒʷ{\={ɘ}}ːr & {\={ɔ}}p &  \\
Duka\il{Duka} & Rijau\il{Rijau} &   &   & dʒiɾɔ & ɔpʰ &  \\
Duka\il{Duka} & Darangi\il{Darangi} &   &   & dʒiɾɔ & ’ɔpʰ &  \\
Duka\il{Duka} & Bunu\il{Bunu} &   &   & dʒiɾɔ & ɔpʰ &  \\
Duka\il{Duka} & Iri\il{Iri} &   &   & dʒɪɾɔ & ɔpʰ &  \\
Duka\il{Duka} & Dukku\il{Dukku} &   &   & dʒɪɾɔ & ɔpʰ &  \\
Duka\il{Duka} & Giro\il{Giro} &   &   & dʒɛdɔ & ɔp &  \\
Kambari & Tsishingini\il{Tsishingini} (Kambari) & kùttʃí &   &   & kùppá &  \\
Kambari & Agaushi\il{Agaushi} (Tsikimba) & kùʧì &   &   & kùpà &  \\
Kambari & Kambali\il{Kambali} (Koelle) & kúciici &   &   & hókp{\textsubbar{a}} &  \\
Kamuku & Western Acipa\il{Acipa} (Cicipu) & kùtítːí (5+4) &   &   & ùkúpːà &  \\
Kamuku & Kamuku (dial.) & t{\'{ə}}nd{\'{ə}}ʃì (5+4) &   &   & òp{\'{ɑ}} &  \\
Kamuku & Hungworo\il{Hungworo} (Hungwere) & {\={u}}t{\'{ə}}n{\`{ə}}s{\~ī} (5+4) &   &   & īkópʲè &  \\
Kamuku & Pongu\il{Pongu} (Pangu) & t{\~{\'u}}ndúʃì (5+4) &   &   & úpwá &  \\
Kamuku & Kamuku (Koelle) & tándaaʃii (5+4) &   &   & ópaa &  \\
Kamuku & Fungwa\il{Fungwa} & tíndíʃì (5+4) &   &   & úpá &  \\
Reshe\il{Reshe} & Reshe\il{Reshe} (Tsureshe) & tānāʃ{\~{\'e}} (5+4) &   &   & úpwà &  \\
\lspbottomrule
\end{tabularx}
\end{table}

There are several forms and patterns for ‘nine’ whose reconstruction is equally plausible: ‘9=5+4’, *\textit{tor(b)oj} (possibly < *’10-1’), \textit{*jiro}. Each of the forms/patterns is characteristic of a particular sub-group of languages. The term for ‘ten’ is reconstructed as \textit{*pwa}, with its reflexes attested in all Western Kainji branches. Three alternative forms (\textit{*turu}, \textit{*kuri,} \textit{*kup/} \textit{kpa}) are found in Eastern Kainji, where they are employed for counting and in quantity measures.

\subsubsubsection{‘Twenty’ and ‘Hundred’}\label{sec:3.1.2.7.7} 
\begin{table}
\caption{\label{tab:3:40}Kainji stems and patterns for `20' and `100'}
\footnotesize
\begin{tabularx}{\textwidth}{lQQllQ}
\lsptoprule
% Kainji & 
~ &   & `20' & `20' & `20' & `100' \\
\midrule
Eastern\\
\midrule 
Jera & Iguta\il{Iguta} &   &   & 12+8 & 12*8+4\\
Jera & Janji\il{Janji} &   &   &   &  \\
Jera & Bunu\il{Bunu} &   &   &   & rì:mú\\
Jera & Buji\il{Buji} &   &   & 10*2 & *ri-nu\\
Amo\il{Amo} & Amo\il{Amo} &   &   & akut-2 & li-kalt\\
\tablevspace

Western\\
\midrule 
Basa\il{Basa} & Basa\il{Basa} & wéʃi (K:wóóʃi) &   &   & dupu íjèbi\newline (50*2) \\
Duka\il{Duka} & C'lela & dᵊkʷ{\`{ɛ}}z{\`{ɛ}} &   &   & kʷ{\`{ɛ}}tt͡ʃt{\~{\'a}} /vz{\'{ɨ}}ŋɡù \\
Duka\il{Duka} & Hun-Saare(\il{Hun-Saare}Duka)\il{Duka} & {\textsubbar{ɛ}}r-kw{\textsubbar{o}}{\textsubbar{o}}z~ &   &   & kw{\textsubbar{o}}{\textsubbar{o}}z-{\textsubbar{ɛ}}t táán (20 * 4 ),  o-zùnɡu\\
Duka\il{Duka} & Ut-Ma'in\il{Ut-Ma'in} &   & {\={ɘ}}rʃīk &   & {\={ɘ}}ʔʃīk{\={ɘ}}ʔtán \newline(20 * 5 )\\
Duka\il{Duka} & Rijau\il{Rijau} &   &   &   &  \\
Duka\il{Duka} & Darangi\il{Darangi} &   &   &   &  \\
Duka\il{Duka} & Bunu\il{Bunu} &   &   &   &  \\
Duka\il{Duka} & Iri\il{Iri} &   &   &   &  \\
Duka\il{Duka} & Dukku\il{Dukku} &   &   &   &  \\
Duka\il{Duka} & Giro\il{Giro} &   &   &   &  \\
Kambari & Tsishingini\il{Tsishingini} (Kambari) &   & úːʃín &   & ?\\
Kambari & Agaushi\il{Agaushi} (Tsikimba) &  &   &  kà-màngà &  \\
Kambari & Kambali\il{Kambali} (Koelle) &   & úʃ{\textsubbar{i}} &   &  \\
Kamuku & Western Acipa\il{Acipa} (Cicipu) &   &   & 10*2 & 10*10, mándá\\
Kamuku & Kamuku (dial.) &   &   & 10*2 & ɗ{\`{ə}}rí \newline(< Hausa\il{Hausa}) or dè òp{\'{ɑ}}\\
Kamuku & Hungworo\il{Hungworo} (Hungwere) &   &   & 10*2 & íh{\={ɔ}}ŋɡʷà, 10*10\\
Kamuku & Pongu\il{Pongu} (Pangu) & w{\'{ə}}ʃí &   &   & bìj{\~{\'i}}n{\~{\'ə}}\\
Kamuku & Kamuku (Koelle) &  &   &  10*2 &  \\
Kamuku & Fungwa\il{Fungwa} &   & kùʤìjò &   & ìkwà:ku,\newline < Hausa\\
Reshe\il{Reshe} & Reshe\il{Reshe} (Tsureshe) &   &   & ál{\`{ə}}s{\`{ə}} & rán{\={ə}}k{\={u}}\\
\lspbottomrule
\end{tabularx}
\end{table}

The diversity of patterns for `hundred' may indicate the absence of the term in Proto-Kainji\il{Proto-Kainji}. The term for `twenty' likely followed the pattern ‘20=10*2’. However, the form \textit{*ʃín/} \textit{ʃík} attested in three of the Western Kainji branches is noteworthy. 

  
\subsubsubsection{Summary}
It should be noted that a full reconstruction of the Kainji numeral system is not presently achievable for a number of reasons: some of the forms have multiple alternative variants, many terms are not attested outside Kainji (or have an obscure morphological structure), the elements of the compound terms are not always identifiable (e.g. in the patterns ‘7={X}+2’ or ‘7=5+{X}’), etc.

The numerals attested within this group are so peculiar (at least for a non-specialist in the Kainji languages like myself) that one may wonder whether the Kainji group should indeed be treated as a branch of Benue-Congo. In any case, it seems reasonable to record all the forms reconstructable within the Kainji sub-groups. These forms and patterns are represented in the table below (\tabref{tab:3:41}).


\newpage  
\subsubsection{Platoid}\label{sec:3.1.2.8}
\subsubsubsection{\textbf{‘One}\textbf{’} (\tabref{tab:3:42})}


The grouping of roots here is admittedly provisional, because their morphological structure is often obscure. In addition, phonetic changes that may have taken place are unknown. It is very difficult to propose any etymological interpretation for the forms represented in the table. Which of them could be attributed to the Proto-Platoid\il{Proto-Platoid} is unclear (*(\textit{y})\textit{in} represents a possibility, in case noun class markers are indeed incorporated into the numerical terms).


\begin{table}
\caption{\label{tab:3:42}Platoid stems for `1'}
\fittable{
\begin{tabular}{rllllll}
\lsptoprule

1. & Alumu-Tesu\il{Tesu} & Tesu\il{Tesu} &  &  &  & à-nyimbere\\
2. & Ayu\il{Ayu} & Ayu\il{Ayu} & ɪ-dɪ &   &   &  \\
3. & Biromic & Birom\il{Birom} &   & ɡw-īnìŋ/(d)-īnìŋ &   &  \\
3. & Biromic & Eten\il{Eten} &  dáy &   &   &  \\
4. & Cenral & Izere\il{Izere} &   &  z-iníŋ &   &  \\
4. & Cenral & Irigwe\il{Irigwe} &   &   &   & ˀzrú\\
4. & Cenral & Kaje\il{Kaje} (dial.) &   &   &   &  yiruŋ/yirəŋ\\
4. & Cenral & Tyap\il{Tyap} &   &   &  a-nyuŋ &  \\
5. & Hyamic & Hyam\il{Hyam} &   & ʒ-ìnì &   &  \\
6. & Ninzic & Mada\il{Mada} &   &  *nɛn &   & ɡy{\={ə}}r\\
6. & Ninzic & Ninzo\il{Ninzo} &   &  *nì &   & jír\\
7. & Northern & Ikulu\il{Ikulu} &   &   &   & íńjí\\
8. & Southeastern & Fyam\il{Fyam} &   &  kʲ-éŋ, *in &   &  \\
9. & Southern & Lijili\il{Lijili} & lō̥ &   &   &  \\
10.& Taroid & Tarok\il{Tarok} (dial.) &   &   & ù-z{\`{ɨ}}ŋ, *ɗ{\'{ɨ}}ŋ? &  \\
11.& Western & Yeskwa\il{Yeskwa} (dial.) &   &   &   & è-nyí\\
11.& Western & Rukuba\il{Rukuba} (dial.) &   &  ɡy-ín &   &  \\
11.& Western & Eggon\il{Eggon} (dial.) &   &   &   &  á-ki{\'{ə}}n\\
11.& Western & Eggon\il{Eggon} (dial.) &  ò-rí &   &   &  \\
11.& Western & Hasha\il{Hasha} &   & nʸ-ìnāŋ &   &  \\
?  & Sambe\il{Sambe} &   & ɲ-íɲínā &   &  \\
\lspbottomrule
\end{tabular}
}
\footnotesize\raggedright
Tesu\il{Tesu} data are taken from \citealt{BlenchKato2012}.
\end{table}

\subsubsubsection{\textbf{‘Two’,} \textbf{‘Three’} \textbf{and} \textbf{‘Four’} (\tabref{tab:3:43})}


The roots for ‘two’ containing voiced and voiceless labials are attested in the Platoid languages (as well as in some other BC branches). They may be tentatively reconstructed as \textit{*pa/} \textit{fa/} \textit{ha} and \textit{*ba/} \textit{wa}.


\begin{table}[t]
\caption{\label{tab:3:43}Platoid stems for `2', `3' and `4'}
\begin{tabularx}{\textwidth}{rlQ lQQl}
\lsptoprule
 &   &   & `2' & `2' & `3' & `4' \\
\midrule
1. &Alumu-Tesu\il{Tesu} & Tesu\il{Tesu} &  & à-hùrwi & à-taatɔ & a-anɛ\\
2. &Ayu\il{Ayu} & Ayu\il{Ayu} & ahwa/afah &   & a-taar & a-naŋaʃ\\
3. &Biromic & Birom\il{Birom} &   & -bā & -tāt & -nāːs\\
3. &Biromic & Eten\il{Eten} &  fà &   &  tàt/tʃàt & nàːs\\
4. &Cenral & Izere\il{Izere} &  fà &   &  taar & nààs\\
4. &Cenral & Irigwe\il{Irigwe} &   & ˀʍʲè & ˀtsʲ{\`{ɛ}} & ˀni\\
4. &Cenral & Kaje\il{Kaje} (dial.) &  '-hwa &   &  '-tat & -nai\\
4. &Cenral & Tyap\il{Tyap} &  a-feaŋ &   &  a-tat & a-naai\\
5. &Hyamic & Hyam\il{Hyam} & f{\textsubbar{e}}ri, *fo &   & taat & naaŋ\\
6. &Ninzic & Mada\il{Mada} &   & y-wā, *gba & tar & nly{\={ɛ}}\\
6. &Ninzic & Ninzo\il{Ninzo} & há &  *gba & tár & n{\={ə}}(s)\\
7. &Northern & Ikulu\il{Ikulu} & íń-pààlá &   & íń-táá & íń-nāā\\
8. &Southeastern & Fyam\il{Fyam} &  por &   &  táár &  naas\\
9. &Southern & Lijili\il{Lijili} &   & à-bē̥ & à-tʃé̥~ & à-nàró̥\\
10.& Taroid & Tarok\il{Tarok} (dial.) &  ù-pàr{\'{ɨ}}m &   &  ù-ʃáɗ{\'{ɨ}}ŋ & ù-nèɗ{\'{ɨ}}ŋ\\
11.& Western & Yeskwa\il{Yeskwa} (dial.) &   & èn-và & èn-tât & èn-nà\\
11.& Western & Rukuba\il{Rukuba} (dial.) &  '-hàk &   &  -tát & -nàs\\
11.& Western & Eggon\il{Eggon} (dial.) &  à-hàà &   &  à-tráá & ù-ɲí\\
11.& Western & Eggon\il{Eggon} (dial.) &  {\`{ɔ}}-hà &   &  {\`{ɔ}}-cá & ò-ɲì\\
11.& Western & Hasha\il{Hasha} & à-pʷò &   & ā-tāt & à-nìŋ\\
? & Sambe\il{Sambe} & bèkà-fà & kà-tú & kà-tār/béká-tār & kà-n{\`{ɛ}}/bèkà-nè\\
\lspbottomrule
\end{tabularx}
\end{table}

\newpage 
The roots for ‘three’ and ‘four’ are more stable. Some of their reflexes suggest that the Proto-Platoid\il{Proto-Platoid} forms must have been close to the NC forms: \textit{*tat} ‘3’ and *\textit{nai} \textit{/} \textit{*nas} ‘4’.

\subsubsubsection{\textbf{‘Five}\textbf{’} \textbf{and} \textbf{‘Six’} (\tabref{tab:3:44})}

\begin{table}
\caption{\label{tab:3:44}Platoid stems and patterns for `5' and `6'}
\small
\fittable{
\begin{tabular}{rll llll}
\lsptoprule

&   &   & `5' & `5' & `6' & `6' \\
\midrule
1. &Alumu-Tesu\il{Tesu} & Tesu\il{Tesu} & a-túŋgú &  & t{\'{ɛ}}r{\'{ɛ}}kífí (<3?) & \\
2. &Ayu\il{Ayu} & Ayu\il{Ayu} & a-tuɡen &   & a-tɛɛr (3PL) &  \\
3. &Biromic & Birom\il{Birom} & -t{\={u}}ŋ{\={u}}n &   &   & -t{\=ī}ːmìn~\\
3. &Biromic & Eten\il{Eten} &   &  wí &  tàːrà (<3) &  \\
4. &Cenral & Izere\il{Izere} &  tùwùn &   &  ìɡà-ràːr (3PL) &  \\
4. &Cenral & Irigwe\il{Irigwe} & ˀtɕʷò{\^{o}} &   & rí-tsʲ{\'{ɛ}}~(3PL) &  \\
4. &Cenral & Kaje\il{Kaje} (dial.) &   &  -pfwɔn &  kə-tat (3PL) &  \\
4. &Cenral & Tyap\il{Tyap} &   &  a-fwuon &  a-taa (3PL) &  \\
5. &Hyamic & Hyam\il{Hyam} & twoo &   & twaa-ni~(5+1) &  \\
6. &Ninzic & Mada\il{Mada} & tun &   & tān-n{\`{ɛ}}n (5+1) &  \\
6. &Ninzic & Ninzo\il{Ninzo} & ʈʷí &   & tā-nì (5+1) &  \\
7. &Northern & Ikulu\il{Ikulu} & íń-c{\={u}}{\={u}} &   & íń-cúnú (5+1?) &  \\
8. &Southeastern & Fyam\il{Fyam} &  tóón &   &  táár-in (5+1) &  \\
9. &Southern & Lijili\il{Lijili} & à-só̥ &   & mìn-zí (3PL?) &  \\
10.& Taroid & Tarok\il{Tarok} (dial.) &  ù-túkún &   &  ù-kp{\'{ə}}-ɗ{\'{ɨ}}ŋ (X+1?) &  \\
11.& Western & Yeskwa\il{Yeskwa} (dial.) & èn-tyúò &   & èn-cí (5+1) &  \\
11.& Western & Rukuba\il{Rukuba} (dial.) &  -túŋ &   &  tàiŋ &  \\
11.& Western & Eggon\il{Eggon} (dial.) &  ò-tnó &  *fúúɲ & ù-fín~(5+1?) & \\
11.& Western & Eggon\il{Eggon} (dial.) &  {\`{ɔ}}-tn{\^{ɔ}} &  *f{\^{ɔ}}ɲ & {\`{ə}}-f{\~{\'i}}(5+1?) & \\
11.& Western & Hasha\il{Hasha} & ā-t{\={u}}k{\={u}}n &   &   & à-kʷìp\\
? & Sambe\il{Sambe} & kà-t{\^{u}}n &   &   & kù-h{\`{ɔ}}/d{\`{ɔ}}g{\`{ɔ}}-h{\`{ɔ}}\\
\lspbottomrule
\end{tabular}
}
\end{table}

The term for ‘five’ is reconstructed as \textit{*tu(ku)n}. It is likely that there was no primary term for ‘six’ in the Proto-Platoid\il{Proto-Platoid} group: in all pertinent languages (except for Eggon\il{Eggon}, Hasha\il{Hasha} and Sambe\il{Sambe}) the term in question either follows the pattern ‘5+1’ or is built by adding a plural class to the term for ‘three’.

\subsubsubsection{‘Seven’ and ‘eight’ (\tabref{tab:3:45})}

\begin{table}
\caption{\label{tab:3:45}Platoid stems and patterns for `7' and `8'}
\small

\begin{tabularx}{\textwidth}{rlQ lQl}
\lsptoprule

& ~ &   & `7' & `8' & `8' \\
\midrule
1. &Alumu-Tesu\il{Tesu} & Tesu\il{Tesu} & t{\'{ɛ}}r{\'{ɛ}}kífí naɲí  (6+X) &  & tsyátsyá\\
2. &Ayu\il{Ayu} & Ayu\il{Ayu} & a-taraŋaʃ (3+4) & a-na-ba-boɡ  (4+X) &  \\
3. &Biromic & Birom\il{Birom} & -tāːmà (5+2) &   & -rwīːt~\\
3. &Biromic & Eten\il{Eten} & nìtà  (4+3) &  nàràs  (4+X) &  \\
4. &Cenral & Izere\il{Izere} & kà-nàsàtáár  (4+3) &   & ì-kárá\\
4. &Cenral & Irigwe\il{Irigwe} & natsʲ{\'{ɛ}} (4+3) &   & klaǹvà\\
4. &Cenral & Kaje\il{Kaje}  (dial.) & tiːruŋ  (cf. yiruŋ `1') & nai-mʊwak  (4+X) &  \\
4. &Cenral & Tyap\il{Tyap} & a-natat  (4+3) & a-ninai\newline  (4 redupl.) &  \\
5. &Hyamic & Hyam\il{Hyam} & twarfo (5+2)? & naaraŋ  (4+X) &  \\
6. &Ninzic & Mada\il{Mada} & tāmɡbā  (5+2) & tāndà  (5+3) &  \\
6. &Ninzic & Ninzo\il{Ninzo} & tāŋɡbā  (5+2) & tāndàr  (5+3) &  \\
7. &Northern & Ikulu\il{Ikulu} & t{\'{ɔ}}{\`{ɔ}}pāā  (5+2) & níǹnāā\newline (4 redupl.) &  \\
8. &Southeastern & Fyam\il{Fyam} & támor  (5+2) &   & tʃínít\\
9. &Southern & Lijili\il{Lijili} & mú-tá &   & rúnó̥ \\
10.& Taroid & Tarok\il{Tarok}  (dial.) & ù-fàŋ-ʃát  (X+3) & ù-n{\`{ə}}nnè\newline  (4 redupl.) &  \\
11.& Western & Yeskwa\il{Yeskwa}  (dial.) & tònvà  (5+2) & tóndát  (5+3) &  \\
11.& Western & Rukuba\il{Rukuba}  (dial.) & taŋbák  (5+2) & taːrat  (5+3) &  \\
11.& Western & Eggon\il{Eggon}  (dial.) & à-fóhà (5+2) & à-fóté (5+3) &  \\
11.& Western & Eggon\il{Eggon}  (dial.) & {\`{ɔ}}-f{\'{ɔ}}hà  (5+2) & {\`{ɔ}}-f{\'{ɔ}}t{\'{ɛ}} (5+3) &  \\
11.& Western & Hasha\il{Hasha} & à-kʷìp nʸīnāŋ  (cf. 6, 4) & nànìŋ\newline (4 redupl.) &  \\
? & Sambe\il{Sambe} & k{\={ɔ}}r{\={ɔ}}nk{\'{ɛ}}rā /kúrk{\'{ə}}nrā &   & ī-t{\'{ɔ}}r\\
\lspbottomrule
\end{tabularx}
\end{table}

Word-building patterns for the term for ‘seven’ are normally quite transparent: ‘7=5+2’ is attested in the majority of the sub-groups, whereas ‘7=4+3’ is more rare. The same can be applied to the term for ‘eight’, which either follows the pattern ‘8=5+3’ or is built by partial reduplication of ‘four’ (4 redupl.). Sometimes the archaic primary terms for ‘two’ and ‘five’ are traceable in the forms for ‘seven’ and ‘eight’ (such forms are marked with an asterisk in the respective tables).

\subsubsubsection{\textbf{‘Nine’} \textbf{and} \textbf{‘Ten’} (\tabref{tab:3:46})}

\begin{table}
\caption{\label{tab:3:46}Platoid stems and patterns for `9' and `10'} 
\footnotesize 
\begin{tabularx}{\textwidth}{rlp{1cm} Qllp{1cm}l}
\lsptoprule

& ~ &   & `9' & `9' & `10' & `10' & `10' \\
\midrule
1. &Alumu-Tesu\il{Tesu} & Tesu\il{Tesu} & tsyátsyá naɲí  (8+X) &  &  &  & gòròmàvɔ\\
2. &Ayu\il{Ayu} & Ayu\il{Ayu} & a-tu-lu-boɡ  (5+4?) &   &   &  i-ʃoɡ/\newline a-ja-la-boɡ & \\
3. &Biromic & Birom\il{Birom} & syāː-tāt (12- 3) &   &   &   & 12-2\\
3. &Biromic & Eten\il{Eten} & dùːdʒàŋ  (10-X) &   &   &   & dùːb{\`{ɔ}}\\
4. &Cenral & Izere\il{Izere} & kàtúb{\'{ɔ}}k  (5+X?) &   &   & kù-s{\'{ɔ}}k &  \\
4. &Cenral & Irigwe\il{Irigwe} &   & kruvájá &   & ʃʷá &  \\
4. &Cenral & Kaje\il{Kaje}  (dial.) & kumʊwiːruŋ  (10-1?) &   & *ku? & swak &  \\
4. &Cenral & Tyap\il{Tyap} & akubunyuŋ  (10-1?) &   & *kub? & swak &  \\
5. &Hyamic & Hyam\il{Hyam} & mbwan kɔb (10-1) &   & k{\'{ɔ}}b &   &  \\
6. &Ninzic & Mada\il{Mada} & tīyār  (X-1?) &   &   &   & ɡùr\\
6. &Ninzic & Ninzo\il{Ninzo} & tīr(s)  (3-X?) &   &   &   & w{\={u}}r\\
7. &Northern & Ikulu\il{Ikulu} &   & t{\'{ɔ}}{\`{ɔ}}llāā & nù-k{\={ɔ}}p &   &  \\
8. &Southeastern & Fyam\il{Fyam} & téres  (3-X?) &   &   &   & dukút\\
9. &Southern & Lijili\il{Lijili} & zà-tʃé̥  (X-3?) &   &   &   & zà-bè̥ \\
10.& Taroid & Tarok\il{Tarok}  (dial.) & ùfàŋz{\'{ɨ}}ŋt{\'{ɨ}}ŋ  (X+4) &   & ù-ɡb{\'{ə}}pei &   &  \\
11.& Western & Yeskwa\il{Yeskwa}  (dial.) &   & tyú{\^{o}}rá & ó-kóp &   &  \\
11.& Western & Rukuba\il{Rukuba}  (dial.) & taːras  (3-X?) &   &   &   & u-wùruk\\
11.& Western & Eggon\il{Eggon}  (dial.) & àfúúɲí ( 5+4) &   & ó-kpo &   &  \\
11.& Western & Eggon\il{Eggon}  (dial.) & {\`{ɔ}}f{\^{ɔ}}ɲí ( 5+4) &   & {\`{ɔ}}-kb{\'{ɔ}} &   &  \\
11.& Western & Hasha\il{Hasha} & nànìŋ màrēŋ  (4+X) &   &   &   & ā-w{\={u}}k\\
? & Sambe\il{Sambe} &   & tōrō/kà-t{\'{ɔ}}r{\'{ɔ}} &   &   & j{\`{ɔ}}-wō\\
\lspbottomrule
\end{tabularx} 
\end{table}

\clearpage 
It is likely that the term for ‘nine’ attested in Ikulu\il{Ikulu}, Yeskwa\il{Yeskwa} and Sambe\il{Sambe} (\textit{toro/cora}) is primary. The hypothetical inter-relationship of these roots may be of interest for the Proto-Platoid\il{Proto-Platoid} reconstruction, because these languages do not belong to the same sub-group. The forms of ‘nine’ in the majority of the languages show traces of ‘five’, ‘four’, ‘ten’ and ‘one’, which suggests that two alternative patterns (‘9=5+4’ or ‘9=10-1’) could have been in use. Some rare patterns (e.g. ‘9=12-3’ (Birom\il{Birom}) and ‘9=8+X (Tesu\il{Tesu})) are of interest for the linguistic typology.

According to \citet{Bouquiaux1962} the term for ‘twelve’ (\textit{k{\={u}}r{\={u}}}) is attested in Birom\il{Birom}.  In this language ‘21’ (\textit{k{\={u}}r{\={u}}} \textit{ná} \textit{syāː-tāt}) = ‘12+9’ (\textit{syāː-tāt}), while ‘80’ (\textit{bāk{\={u}}r{\={u}}} \textit{bātīː} \textit{mìn} \textit{ná} \textit{rwīːt}) = ‘12*6’ (-\textit{t{\=ī}ː} \textit{mìn}) + ‘8’\textit{(-rwīːt}). The pattern ‘9=12-3’ is not totally unexpected within this context. A similar system can be traced in the Mada\il{Mada} language. As stated in our source (Abiel Barau Kato), “Like many languages in Platoid area, Mada has an old duodecimal numeral system up to 24.”\footnote{\href{https://mpi-lingweb.shh.mpg.de/numeral/Ninzo.htm}{https://mpi-lingweb.shh.mpg.de/numeral/Ninzo}\il{Ninzo}\href{https://mpi-lingweb.shh.mpg.de/numeral/Ninzo.htm}{.htm}} The Mada terms for ‘twelve’ and ‘twenty-one’ are \textit{tsɔ}~and \textit{tsɔtīyār} (\textit{tīyār} ‘9’) respectively. The same root for ‘twelve’ (\textit{tsó} ‘12’) is found in Ninzo\il{Ninzo} for which our source notes that ‘’In the traditional counting system, to count beyond twelve (12), that is from thirteen onwards, entails counting in sets of twelve.’’\footnote{\href{https://mpi-lingweb.shh.mpg.de/numeral/Ninzo.htm}{https://mpi-lingweb.shh.mpg.de/numeral/Ninzo}\il{Ninzo}\href{https://mpi-lingweb.shh.mpg.de/numeral/Ninzo.htm}{.htm}} Moreover, the same root is attested in Tesu\il{Tesu} (\textit{tsɔ} ‘12’). According to Uche Aaron, a primary root \textit{{\`{ɔ}}-cʷ{\'{ɔ}}}~‘12’ is discernible in Eggon\il{Eggon} (beside the composite term ‘12=10+2’). This root is also found in Rukuba\il{Rukuba} (Che) in \textit{u-s{\'{ɔ}}k} ‘12’. The duodecimal numeral system as attested in this language is of the utmost sophistication. According to Luc Bouquiaux: ‘’There are two words for number `72', \textit{kitu} and \textit{atu}, 144 can be expressed as \textit{atu} \textit{ahak} and 200 is \textit{atu} \textit{ahak} \textit{ni} \textit{isɔk} \textit{inas} \textit{ni} \textit{hak} \textit{ni} \textit{taːrat}~( 72 * 2) + (12 * 4) + 8.’’\footnote{\href{https://mpi-lingweb.shh.mpg.de/numeral/Rukuba.htm}{https://mpi-lingweb.shh.mpg.de/numeral/Rukuba}\il{Rukuba}\href{https://mpi-lingweb.shh.mpg.de/numeral/Rukuba.htm}{.htm}} Other languages in this group normally use less exotic systems. In some of them, however, e.g. in Eten\il{Eten}, ‘’The highest number that can be counted in traditional way is 144,’’\footnote{\url{https://mpi-lingweb.shh.mpg.de/numeral/Aten.htm}}, i.e. ‘12*12’. To sum up, it seems that a primary term for ‘twelve’ can be reconstructed on the Proto-Platoid\il{Proto-Platoid} level, hence the pattern for ‘nine’ should most probably be reconstructed as *’9=12-3’. 

The system outlined above adds a new perspective to the forms with the meaning ‘ten’. Presumably, there was a Proto-Platoid\il{Proto-Platoid} primary term for ‘ten’ that may be tentatively described as *\textit{kop}. The alternative forms \textit{sok/swak} may be etymologically related to the forms for ‘twelve’ cited above. If so, their change of meaning may have resulted from the adoption of a decimal system. The root \textit{gur/wur} is distinguished as well.

The specific nature of the Platoid numeral system prevents us from providing separate forms for ‘twenty’ and ‘hundred’. The pattern *’20=12+8’ traceable in a number of pertinent languages is reconstructed for Proto-Platoid\il{Proto-Platoid}. A compound nature is also assumed for ‘hundred’.

The results pertaining to the advanced reconstructions of numerals in Proto-Platoid\il{Proto-Platoid} are summed up in the table below (\tabref{tab:3:47}).

\begin{table}
\caption{\label{tab:3:47}Proto-Platoid\il{Proto-Platoid} numeral system (*)}
\begin{tabularx}{\textwidth}{lXlX}
\lsptoprule
{1} & (y)in, di(n), jir, nìŋ & {7} & 5+2, 4+3\\
{2} & pa/fa/ha, ba/wa. & {8} & 4 redupl., 5+3\\
{3} & tat & {9} & 5+4, 10-1, 12-3, tu(ku)n\\
{4} & nai/nas & {10} & kop, gur/wur\\
{5} & tu(ku)n & {20} & 12+8\\
{6} & 5+1, 3PL & {100} & ?\\
\lspbottomrule
\end{tabularx}
\end{table}

\newpage  
\subsubsection{Nupoid}\label{sec:3.1.2.9}
Let us try to reconstruct the Proto-Nupoid numeral system.

\begin{table}
\caption{\label{tab:3:48}Nupoid numerals and Proto-Nupoid (*)}


\begin{tabularx}{\textwidth}{rQQQQQ}
\lsptoprule

Nupoid & Ebira\il{Ebira} & Gbari\il{Gbari} & Kakanda\il{Kakanda} & Nupe\il{Nupe} & \textbf{*Nupoid}\\
\midrule
1 & {\`{ɔ}}{\`{ɔ}}-ny{\={ɪ}} & ɡbᵐaː-ɾí,*w{\~{i}} & ɡú-ní & ni-ní & \textbf{ni/} \textbf{nyi,} \textbf{wi?} \textbf{ri}?\\
2 & {\`{ɛ}}{\`{ɛ}}-vā & ŋʷ{\~{\^a}}-ba & ɡú-bà & ɡú-bà & \textbf{ba}\\
3 & {\`{ɛ}}{\`{ɛ}}-tá & ŋʷ{\~{\^a}}-tʃa & ɡú-tá & ɡú-tá & \textbf{ta}\\
4 & {\`{ɛ}}{\`{ɛ}}-nà & ŋʷ{\~{\^a}}-ɲi & ɡú-ni & ɡú-ni & \textbf{na/} \textbf{ni}\\
5 & {\`{ɛ}}{\`{ɛ}}-h{\'{ɪ}} & ŋʷ{\~{\^a}}-tⁿù & ɡú-tũ & ɡú-tsũ & \textbf{tun/} \textbf{tnu/tsun,} \textbf{hi?}\\
6 & h{\H{ɪ}}-n{\H{ɔ}}-ny{\={ɪ}} (5+1) & tⁿú-w{\~{i}}  (5+1) & ɡú-tua-ɲ{\~{\`i}}~ (5+1) & ɡú-tswà-ɲ{\~{i}}~ (5+1) & \textbf{5+1}\\
7 & h{\H{ɪ}}-{\H{m}}-bā (5+2) & tⁿâ-ba  (5+2) & ɡú-tua-bà (5+2) & ɡú-twà-bà~ (5+2) & \textbf{5+2}\\
8 & h{\H{ɪ}}-{\H{n}}-tá (5+3) & tⁿ{\~{â}}-tʃa  (5+3) & ɡú-tò-tá (5+3) & ɡú-to-tá (5+3) & \textbf{5+3}\\

9 & h{\H{ɪ}}-ǹ-nà (5+4) & tⁿâ-ɲi  (5+4) & ɡú-tua-ni~ (5+4) & ɡú-tw{\~{\`a}}-ni (5+4) & \textbf{5+4}\\
10 & {\`{ɛ}}{\`{ɛ}}-w{\'{ʊ}} & ŋʷ{\~{\^a}}-wò & ɡú-wo & ɡú-wo & \textbf{wo}\\
20 & òò-h{\={u}},*tʃ{\H{ɛ}} & wo-ʃì & e-ʃ{\~{\'i}} & e-ʃi & \textbf{ʃi,} \textbf{hu?} \\
100 & {\={ɛ}}-tʃ{\H{ɛ}}-h{\'{ɪ}}~ (20x5) & 40*2+20 & ʃìt-ũ  (20*5) & ʃit-sũ~ (20*5) & \textbf{20*5}\\
1000 &  400*5??? &  100*10 &   &  kpá-tsũ~  (200*5) & ?\\
\lspbottomrule
\end{tabularx}
\end{table}

The Nupoid group is relatively small and homogeneous and poses no problem for reconstruction.  

\clearpage
\subsection{Isolated BC  languages} %3.1.3.

\subsubsection{Ikaan}\label{sec:3.1.3.1}
\il{Ikaan}The following description of the Ikaan\il{Ikaan} numeral system (\tabref{tab:3:49}) is based on the analysis of data from a number of its dialects.

\begin{table}
\caption{\label{tab:3:49}Proto-Ikaan\il{Proto-Ikaan} numeral system (*)}


\begin{tabularx}{\textwidth}{rXrl}
\lsptoprule

{1} & ʃí & {7} & h-ránèʃì ('6+1’)\\
{2} & wà & {8} & nàːnáʲ (4 redupl.)\\
{3} & tāːs /h-rāhr & {9} & h-ráòʃì (X-1)\\
{4} & nāʲ /nā/náh{\textsubtilde{í}} & {10} & ò-pú/fú\\
{5} & tòːn/h-r{\`{ʊ}}ːn/s{\textsubtilde{ò}}n/c{\textsubtilde{\`{ʊ}}}n{\textsubbar{v}} & {20} & ù-ɡb{\'{ɔ}}r{\'{ɔ}} (< `sack'), * à-ɡbá\\
{6} & h-ɾàdá/sàdá/sàrá & {100} & à-ɡbá à-h-ruǹ(20*5)\\
\lspbottomrule
\end{tabularx}
\end{table}

 
\subsubsection{Akpes} %3.1.3.2.
\il{Akpes}
\begin{table}
\caption{\label{tab:3:50}Akpes\il{Akpes} numerals}


\begin{tabularx}{\textwidth}{rXrl}
\lsptoprule

{1} & í-ɡbōn, ē-kìnì & {7} & ī-tʃēnētʃ(ì)\\
{2} & ī-dīan(ì) & {8} & ā-nāānīŋ(ì) (4 redupl.)\\
{3} & ī-sās(ì) & {9} & {\`{ɔ}}-kp{\={ɔ}}l{\`{ɔ}}ʃ(ì)\\
{4} & ī-nīŋ(ì) & {10} & ī-yōf(ì), *t-ēfī\\
{5} & ī-ʃōn(ì) & {20} & {\={ɔ}}-ɡb{\={ɔ}}(l{\={ɔ}})\\
{6} & ī-tʃānās(ì) & {100} & ī-ɡb{\'{ɔ}} ʃōnì~(20*5)\\
\lspbottomrule
\end{tabularx}
\end{table}

The original BC forms for ‘five’ (\textit{*tan}) and ‘one’ may have been preserved in the term for ‘six’. These forms will be treated below as hypothetical.

\clearpage
\subsubsection{Oko}\label{sec:3.1.3.3}
\il{Oko}
\begin{table}
\caption{\label{tab:3:51}Oko\il{Oko} numerals }


\begin{tabularx}{.75\textwidth}{rXrl}
\lsptoprule

{1} & {\`{ɔ}}-{\'{ɔ}}rɛ, {\`{ɔ}}-j{\'{ɛ}}rɛ & {7} & ú-f{\'{ɔ}}mb{\`{ɔ}}r{\`{ɛ}} (5+2)\\
{2} & {\`{ɛ}}-b{\`{ɔ}}r{\`{ɛ}} & {8} & {\`{ɔ}}n{\'{ɔ}}k{\'{ɔ}}nɔk{\'{ɔ}}nɔ(4 redupl.?)\\
{3} & {\`{ɛ}}-ta & {9} & ù-b{\'{ɔ}}{\`{ɔ}}r{\`{ɛ}}(10-1)\\
{4} & {\`{ɛ}}-na & {10} & {\`{ɛ}}-fɔ\\
{5} & ù-pi & {20} & {\'{ɔ}}-ɡbɔlɔ\\
{6} & {\`{ɔ}}-p{\'{ɔ}}n{\`{ɔ}}{\'{ɔ}}rɛ (5+1) & {100} & í-pì\\
\lspbottomrule
\end{tabularx}
\end{table}

 
\subsubsection{Lufu}\label{sec:3.1.3.4}
\il{Lufu}

\begin{table}
\caption{\label{tab:3:52}Lufu\il{Lufu} numerals}


\begin{tabularx}{.75\textwidth}{rXrl}
\lsptoprule

{1} & ù-tí & {7} & 5+2\\
{2} & (ba)-máhà & {8} & 5+3\\
{3} & bá-tá & {9} & 5+4\\
{4} & ba-ɲì & {10} & ú-wó\\
{5} & bá-tsó & {20} & e-ce\\
{6} & 5+1 & {100,} {1000} & ? \\
\lspbottomrule
\end{tabularx}
\end{table}

 
\subsection{Proto-Benue-Congo}\label{sec:3.1.4}
\il{Proto-Benue-Congo}
\subsubsection{‘One’}\label{sec:3.1.4.1}
The reconstruction of the term for ‘1’ is objectively the most challenging (the term is especially difficult to reconstruct in languages with noun classes and complex systems of determinatives). This situation is even more complicated in the Benue-Congo languages, since more than one reconstruction of the term has been suggested. The existing hypotheses must be studied here, especially because the ones pertaining to the etymology of the term were proposed by Kay Williamson, the leading specialist in NC comparative studies. Moreover, Kay \citet{Williamson1989b} used her reconstruction of the term for ‘one’ as an argument in favor of triconsonantal structure of Niger-Congo roots. This hypothesis has been actively developed by Roger Blench (\citeyear*{Blench2012a} etc.). 

It should be noted that our evidence does not support Kay Williamson’s reconstruction. Furthermore, her hypothesis regarding the triconsonantal nature of Niger-Congo roots is, in my opinion, untenable. The Bantoid data utilized by Williamson was discussed above. Now let us review the evidence she uses in support of her hypotheses. Originally she treated the root \textit{\#-kani} ‘1’ as one of the basic BC roots (‘old root’, \citealt{Williamson1989b}: 255). Later she changed her approach (on the basis of a wider NC context, namely on the data from the ljo languages) suggesting a derivation of BC froms from a triconsonantal root \textit{**-} \textit{'k{\textsubbar{ə}}'g{\textsubbar{ə}}n{\textsubbar{i}}} ‘1’, for which she assumed a different set of reflexes \citep[396]{Williamson1992}. The changes introduced by Williamson in this article are significant. She adds the reflexes of the reconstructed root in Akpes\il{Akpes} and Nupoid, includes its additional reflexes in Esimbi\il{Esimbi} and Bekwarra\il{Bekwarra} (Bantoid), adjusts its reflexes in Cross and Platoid (e.g. by reinterpreting PUC\il{PUC} \textit{gá-ni/} \textit{*-gwá-n{\`{ɩ}}} previously analysed as an isolated form as a reflex of the root in question), and, finally, omits Kanji and Jukunoid reflexes.

In further interpretation of the BC numeral systems we will use a template chart representing the fourteen branches of BC (\tabref{tab:3:53}). It should be noted that Bantu (as the largest sub-branch of the BC family with the most detailed reconstruction) is treated separately. This means that the Bantoid field will only include non-Bantu forms. The chart below reproduces the data published by Kay Williamson (middle sections) as well as the relevant forms obtained as a result of our step-by-step reconstruction (the rightmost section).

\begin{table}[t]
\caption{\label{tab:3:53}BC *kin/cin `1' and alternative reconstructions}
\begin{tabularx}{\textwidth}{llQ}
\hline
{} & {\textbf{Benue-Congo}} & \\
% \midrule
\cellcolor{gray!20!white}Nupoid &\cellcolor{gray!10!white}  Oko\il{Oko}     &\cellcolor{gray!30!white}  Kainji\\
\cellcolor{gray!20!white}Defoid &\cellcolor{gray!10!white}  Akpes\il{Akpes} &\cellcolor{gray!30!white}  Platoid\\
\cellcolor{gray!20!white}Edoid  &\cellcolor{gray!10!white}  Ikaan\il{Ikaan} &\cellcolor{gray!30!white}  Cross\\
\cellcolor{gray!20!white}Igboid &\cellcolor{gray!10!white}  Lufu\il{Lufu}   &\cellcolor{gray!30!white}  Jukunoid\\
\cellcolor{gray!20!white}Idomoid&\cellcolor{gray!50!white}   Bantu           &\cellcolor{gray!30!white}  Bantoid\\
\tablevspace

\multicolumn{3}{c}{\citealt{Williamson1989b}: \textit{\#-kani} ‘1’}  \\
% \midrule
\cellcolor{gray!20!white}                                   &\cellcolor{gray!10!white}   &\cellcolor{gray!30!white} Basa\il{Basa} kə\\
\cellcolor{gray!20!white}Yoruba\il{Yoruba} {\`{ɔ}}-k{\~{ɔ}} &\cellcolor{gray!10!white}   &\cellcolor{gray!30!white} Pyem\il{Pyem} kēŋ\\
\cellcolor{gray!20!white}                                   &\cellcolor{gray!10!white}   &\cellcolor{gray!30!white} Bete-\il{Bete}Bendi\il{Bete-Bendi} ì-k{\={ə}}n, Bokyi\il{Bokyi} k{\`{ɨ}}n, PLC\il{PLC} *-kèèn\\
\cellcolor{gray!20!white}                                   &\cellcolor{gray!10!white}   &\cellcolor{gray!30!white} Jukun\il{Jukun} kā\\
\cellcolor{gray!20!white}Eloyi\il{Eloyi} kònzé              &\cellcolor{gray!50!white}    &\cellcolor{gray!30!white} Tiba\il{Tiba} a-kina\\
\tablevspace

\multicolumn{3}{c}{\citealt{Williamson1992}: Proto-Atlantic-\il{Proto-Atlantic}Congo \textbf{\textit{**-'k{\textsubbar{ə}}'g{\textsubbar{ə}}n{\textsubbar{i}}}}‘1’}\\
% \midrule
\cellcolor{gray!20!white}Gbagyi gmànyi                      &\cellcolor{gray!10!white}              &\cellcolor{gray!30!white} \\
\cellcolor{gray!20!white}Yoruba\il{Yoruba} {\`{ɔ}}-k{\~{ɔ}} &\cellcolor{gray!10!white}  Ikeram ɛ-ki &\cellcolor{gray!30!white} PP2-J -gini, PP4 -ɣan\\
\cellcolor{gray!20!white}                                   &\cellcolor{gray!10!white}              &\cellcolor{gray!30!white} PUC\il{PUC} gá-ni? , PLC\il{PLC} -kèèn\\
\cellcolor{gray!20!white}                                   &\cellcolor{gray!10!white}              &\cellcolor{gray!30!white} \\
\cellcolor{gray!20!white}Eloyi\il{Eloyi} kònzé              &\cellcolor{gray!50!white}               &\cellcolor{gray!30!white} Tiba\il{Tiba} a-kina, Esimbi\il{Esimbi} keni, Bendi:\il{Bendi} Bekwarra\il{Bekwarra} o-kin\\
\tablevspace

\multicolumn{3}{c}{\textbf{*\textit{kin-/cin-}} forms for `1' (step-by-step data)}\\
% \midrule
\cellcolor{gray!20!white}&\cellcolor{gray!10!white}                    &\cellcolor{gray!30!white} tsin, hin\\
\cellcolor{gray!20!white}&\cellcolor{gray!10!white}  ē-kìnì, *si &\cellcolor{gray!30!white} (y)in, kyeŋ, ɡyin\\
\cellcolor{gray!20!white}&\cellcolor{gray!10!white}  ʃí                &\cellcolor{gray!30!white} kin/cin\\
\cellcolor{gray!20!white}&\cellcolor{gray!10!white}                    &\cellcolor{gray!30!white} ʃ{\'{ɪ}}ʃe?\\
\cellcolor{gray!20!white}&\cellcolor{gray!50!white}                     &\cellcolor{gray!30!white} cin (Mambiloid)\\
\hline
\end{tabularx}\\
\raggedright\medskip\footnotesize
Different colors are used in the charts to distinguish between the Eastern and the Western BC languages. A special marking is used for the Bantu languages due to their overall importance for the reconstruction. The abbreviations in the middle sections follow Williamson op. cit. with PLC\il{PLC}- Proto-Lower Cross\il{Proto-Lower Cross}, PUC\il{PUC} – Proto-Upper Cross\il{Proto-Upper Cross}, PP\il{PP} – Proto-Platoid\il{Proto-Platoid}.
\end{table}

It should be noted that the difference in the results achieved by means of our step-by-step reconstruction (see above) and those of Williamson is significant. According to our evidence, the postulation of the root \textit{**-} \textit{'k{\textsubbar{ə}}'g{\textsubbar{ə}}n{\textsubbar{i}}} ‘1’ for Western Benue-Congo is unsustainable. The existence of this root in Bantoid is also questionable. In her earlier publication, Kay Williamson quoted its only Bantoid reflex (\textit{a-kina} ‘1’) supposedly attested in Northern Bantoid Tiba\il{Tiba} \citep[255]{Williamson1989b}. However, the affiliation of Tiba with the Bantoid languages is debatable (a connection with the Adamawa languages is suggested in \citealt{Boyd1999}). In the article that followed, Williamson quoted another Bantoid form, this time the one attested in Southern Bantoid Esimbi\il{Esimbi} (\textit{keni} ‘1’). As noted above, this form was probably misinterpreted, becaused it includes the root \textit{-ni/-n{\={ə}}}. At the same time, as I tried to demonstrate above, a number of related forms may be attested in the Mambiloid languages (Northern Bantoid): Twendi\il{Twendi} (Cambap) \textit{tʃínī}, Mambila\il{Mambila} \textit{tʃ{\'{ɛ}}n}. Thus, we are possibly dealing with Proto-Eastern Bantoid\il{Proto-Eastern Bantoid} \textit{*cin/kin}. In order to decide whether this form is an innovation or a reflex of an inherent Niger-Congo root (as Kay Williamson says) we need to place it into a wider linguistic context. This issue will be addressed later. At this point we will deal with another root for ‘one’ postulated by Williamson. According to her, the root is a Benue-Congo innovation.

\largerpage
Since the root \textit{n{\={ə}}} \textit{/} \textit{ni} is distinguishable in Esimbi\il{Esimbi}, it seems logical to treat it together with another set of terms for ‘one’ (\textit{\#-diiŋ}). This data (termed BC innovation by Williamson) compared to the results of our step-by-step reconstruction is quoted in the table below (\tabref{tab:3:54}).

\begin{table}
\caption{\label{tab:3:54}BC *ni `1' and alternative reconstructions}
\begin{tabularx}{\textwidth}{llQ}
\hline
{} & {\textbf{Benue-Congo}} & \\
% \midrule
\cellcolor{gray!20!white}Nupoid  &\cellcolor{gray!10!white} Oko\il{Oko}     &\cellcolor{gray!30!white} Kainji\\
\cellcolor{gray!20!white}Defoid  &\cellcolor{gray!10!white} Akpes\il{Akpes} &\cellcolor{gray!30!white} Platoid\\
\cellcolor{gray!20!white}Edoid   &\cellcolor{gray!10!white} Ikaan\il{Ikaan} &\cellcolor{gray!30!white} Cross\\
\cellcolor{gray!20!white}Igboid  &\cellcolor{gray!10!white} Lufu\il{Lufu}   &\cellcolor{gray!30!white} Jukunoid\\
\cellcolor{gray!20!white}Idomoid &\cellcolor{gray!50!white}  Bantu           &\cellcolor{gray!30!white} Bantoid\\
\tablevspace
\multicolumn{3}{c}{\citealt{Williamson1989b}: BC innovations: \textbf{\textit{\#-diiŋ}}}  \\
\cellcolor{gray!20!white}Gwari\il{Gwari} ǹ-ɲ{\={ɩ}} &\cellcolor{gray!10!white}  Oko\il{Oko} {\`{ɔ}}{\'{ɔ}}rɛ &\cellcolor{gray!30!white}  Gurmana\il{Gurmana} nɩ\\
\cellcolor{gray!20!white}PY *i-n{\~{\'ɛ}}           &\cellcolor{gray!10!white}                               &\cellcolor{gray!30!white}  PP2K *-niiŋ\\
\cellcolor{gray!20!white}                           &\cellcolor{gray!10!white}                               &\cellcolor{gray!30!white}  OG {\`{ɛ}}-n{\~{ɛ}}, CD \#-niin\\
\cellcolor{gray!20!white}Ikwere\il{Ikwere} ń-ním    &\cellcolor{gray!10!white}                               &\cellcolor{gray!30!white}  PJ *-yiŋ\\
\cellcolor{gray!20!white}PId *-nyí                  &\cellcolor{gray!50!white}                                &\cellcolor{gray!30!white}  Lamja n{\={u}}n{\'{ɛ}}, Ekoid \#-jid, -jiŋ\\
\tablevspace
\multicolumn{3}{c}{\textbf{\textit{*ni}} forms for `1' (step-by-step data)}\\
\cellcolor{gray!20!white}*ni/nyi         &\cellcolor{gray!10!white}  &\cellcolor{gray!30!white}  Bunu\il{Bunu} ù-{\`{ŋ}}ŋínì\\
\cellcolor{gray!20!white}*ɲ{\'{ɛ}}        &\cellcolor{gray!10!white}  &\cellcolor{gray!30!white}  nìŋ, (y)in, di(n)\\
\cellcolor{gray!20!white}                 &\cellcolor{gray!10!white}  &\cellcolor{gray!30!white}  *ni(n)\\
\cellcolor{gray!20!white}ŋìn{\'{ɛ}}?      &\cellcolor{gray!10!white}  &\cellcolor{gray!30!white}  *-jin?\\
\cellcolor{gray!20!white}nze/je/nye/ye &\cellcolor{gray!50!white}   &\cellcolor{gray!30!white}  Esimbi\il{Esimbi} -nə/-ni\\
\hline
\end{tabularx}
\end{table}

Let us review the distribution of this root within the Benue-Congo branches.\todo{are there only two branches?}

\paragraph*{Western Benue-Congo.}

This root can be reliably reconstructed in Nupoid and Defoid, but not in Edoid. In Igboid it might be attested in Ikpeye: \textit{ŋì-n{\'{ɛ}}} \textit{(ŋ-ìn{\'{ɛ}}}?). The root is possibly found in some of the Idomoid languages as well: Etulo\il{Etulo} \textit{o-ɲiī}, Agatu\il{Agatu} \textit{ó-yè}, Idoma\il{Idoma} \textit{é-yè}, Alago\il{Alago} \textit{ó-je}, Eloyi\il{Eloyi} (dial.) \textit{ò-nzé}, \textit{ńɡwò-nzé}. 

\paragraph*{Eastern Benue-Congo.}

Several Kainji forms deserve closer attention. The Gurmana\il{Gurmana} form quoted by Williamson is unfamiliar to me. It may be related to the Bunu\il{Bunu} form, but the root itself is uncommon for Kainji and thus cannot be reconstructed. Moreover, the root is only marginally attested in the Platoid languages (single occurrences include Eskwa \textit{è-nyí} ‘1’~and possibly Ikulu\il{Ikulu} \textit{í-ń-jí} ‘1’, and \textit{k{\`{ɔ}}p-ìrì-z{\={ɨ}}ŋ} ‘11’). Another rare form is \textit{di}(\textit{n}) with an initial oral consonant (e.g. Ayu\il{Ayu} \textit{ɪ-dɪ} ‘1’, Eggon\il{Eggon} \textit{ò-rí} ‘1’ and its palatalized variant \textit{tʃíŋ} – cf. \textit{{\`{ɔ}}-kb{\'{ɔ}}} \textit{à-tʃíŋ} ‘11’, \textit{{\`{ə}}-kβ}\textit{{\'{ə}}há} \textit{là-tʃíŋ} ‘21’). These (etymologically unrelated?) forms, however, should not be reconstructed for Proto-Platoid\il{Proto-Platoid}, because the root \textit{kin} (see above) is clearly distinguishable in the majority of the Platoid branches. At the same time, the Platoid data discredits the reconstruction of the root as *\textit{kin}/\textit{cin}. Multiple arguments can be adduced in favor of the interpretation of the initial velar as a reflex of an archaic noun class prefix, which would yield a Proto-Platoid form *\textit{k-in}. This invites the possibility of an etymological connection between the Benue-Congo roots studied above, namely *-\textit{in} and *-\textit{ni}. The analysis of the Platoid compound numerals points toward the same conclusion. A number of noteworthy forms can be quoted in support of this, cf.  Hyam\il{Hyam} \textit{ʒìnì} ‘1’ but \textit{twaa-ni} ‘6’ (‘5+1’, \textit{twoo} ‘5’), Mada\il{Mada} \textit{tānn-{\`{ɛ}}n} ‘6’ (‘5+1’, \textit{tun} ‘5’), Ninzo\il{Ninzo} \textit{tānì} ‘6’ (‘5+1’, \textit{ʈʷí} ‘5’), Rukuba\il{Rukuba} \textit{tàiŋ} ‘6’ (‘5+1’, \textit{-túŋ} ‘5’). These Platoid forms bring to mind the case of the Jukonoid term for ‘six’. Kay Williamson quotes a Proto-Jukunoid\il{Proto-Jukunoid} root \textit{*-yiŋ.} The reasons behind this reconstruction are not immediately apparent, since in the majority of the languages other forms are reserved for this meaning. Her reconstruction may be based on the compound terms for ‘six’ that follow the pattern ‘5+1’ (or rather ‘5+X’, with X ${\neq}$ 1), cf. e.g. Jibu\il{Jibu} \textit{sùn-jin} ‘6’ (\textit{swana} ‘5’, \textit{zyun} ‘1’), \textit{cìn-jen/} \textit{ʃì-ʒen} (\textit{tswana} ‘5’, \textit{dzun} ‘1’). As noted above, the root in question is not reconstructable for the Platoid languages. The reconstruction of *\textit{ni(n)} is assured only for the Eastern Benue-Congo branch (Cross), where it is systematically attested in at least three branches out of five, cf. Proto-Upper Cross\il{Proto-Upper Cross} (*\textit{ni}), Central-Cross (\textit{nin}), and Ogoni\il{Ogoni} (\textit{nɛ}). Since *\textit{ni} can be safely reconstructed for Nupoid, Defoid and Cross, its further comparison to the pertinent roots attested in the languages that belong to other NC branches is required.

In conclusion, it should be noted that regardless of whether a conservative or a more speculative reconstruction (i.e. \textit{*kin} and \textit{*ni} vs. \textit{*k-in/} \textit{ni}) is preferred, the resulting root (or roots) is not tri- or disyllabic but rather monosyllabic.

In addition to this, several isolated roots for ‘one’ are attested in Benue-Congo. Undoubtedly, they represent local innovations. At first glance, this is applicable to the most common Bantoid roots for ‘one’, including the Bantu forms \textit{mòì/mòdì} \and \textit{mòtí}. This, however, may not be entirely correct for reasons that will be discussed in the next chapter. Another noteworthy root that may be tentatively described as \textit{*jir} is attested in both Oko\il{Oko} and Platoid.

The table is subject to further interpretation. We will return to it later after the evidence from the other Niger-Congo branches has been collected. A few remarks are in order here: 

\begin{enumerate}
  \item Both Akpes\il{Akpes} terms for ‘one’ (\textit{ē-kìnì,} \textit{í-ɡbōn}) find close parallels in the Cross languages (\textit{*kin/cin,} \textit{*}\textit{ni(n),} \textit{*gboŋ/gwan}). The Icheve\il{Icheve} form \textit{à-m{\'{ɔ}}{\`{ɔ}}} is probably borrowed from one of the Bantu languages; 
  \item The Kainji term finds parallels in the Platoid languages (Ayu\il{Ayu}, Eten\il{Eten}, Tarok\il{Tarok}, Eggon\il{Eggon}) and may be etymologically related to the Bantu and Nupoid terms (the morphological structure of the Proto-Bantu\il{Proto-Bantu} form is, however, unclear: \textit{*mòdì?} \textit{*m-òdì?} \textit{*mò-dì?}); 
  \item  The Oko\il{Oko} form is reminiscent of another Platoid form that is tentatively reconstructed as *\textit{jir}. The Akpes\il{Akpes} root \textit{ɡbōn} ’1’ finds parallels in the Cross (\textit{gboŋ}) and possibly Edoid languages (\textit{gwo/} \textit{wo/} \textit{wu}). 
\end{enumerate}

 
\subsubsection{‘Two’}\label{sec:3.1.4.2}
\begin{table}
\caption{\label{tab:3:55}BC stems for `2'}


\begin{tabularx}{\textwidth}{llXXX}
\lsptoprule

~ & \textbf{~} & `2' & `2' & `2' \\
\midrule
East & {Bantu} &   &   & bà-d{\'{ɩ}} /b{\`{ɩ}}-d{\'{ɩ}}\\
East & {Bantoid} {(–Bantu)} & pa/fe & ba &  \\
East & {Cross} & po/pa & bae &  \\
East & {Jukunoid} & pa(n) /fa(n) &   &  \\
East & {Kainji} & -pu? & *ba/bi & re\\
East & {Platoid} & pa/fa/ha & ba/wa &  \\
West & {Defoid} &   &   & jì\\
West & {Edoid} &   & va/və &  \\
West & {Idomoid} & pa &   &  \\
West & {Igboid} &   & b{\'{ɔ}} &  \\
West & {Nupoid} &   & ba &  \\
West & {Akpes}\il{Akpes} &   &   & ī-dīan(ì)\\
West & {Oko}\il{Oko} &   & {\`{ɛ}}-b{\`{ɔ}}r{\`{ɛ}} &  \\
West & {Ikaan}\il{Ikaan} &   & wà &  \\
\lspbottomrule
\end{tabularx}
\end{table}

The root *\textit{pa} (also found in the Idomoid languages) is reconstructable for Eastern Benue-Congo, but is not systematically attested in Bantu. 

The Bantu form (as represented above) does not seem to be related to other Bantoid forms. However, it finds parallels in Defoid and possibly Akpes\il{Akpes} and Kainji. The most common BC form (*\textit{ba}/\textit{bai}) may go back to *\textit{ba-i}, with *\textit{ba}- being a noun class prefix. In this case, the BC form may be reconstructed as *\textit{ba-di} \textit{/} \textit{ba-ji} > \textit{bai} > \textit{ba}, which would make the Bantu form the most archaic within Benue-Congo. 

These hypotheses will be discussed below, after the evidence from the other BC branches has been reviewed. 

 
\subsubsection{‘Three’, ‘four’, ‘five’}\label{sec:3.1.4.3}
\begin{table}
\caption{\label{tab:3:56}BC stems for `3', `4' and `5'}


\begin{tabularx}{\textwidth}{llllQl}
\lsptoprule
& {~} & `3' & `4' & `5' & `5' \\
\midrule 
{East} & {Bantu} & tat & nàì /(nàí) & táànò &  \\
{East} & {Bantoid} {(–Bantu)} & tat & nai & tan &  \\
{East} & {Cross} & ta(t)/ca(t) & na(n) & tan & *gbo(k)\\
{East} & {Jukunoid} & ta & nye & tsoŋ &  \\
{East} & {Kainji} & tat & nas & tan &  \\
{East} & {Platoid} & tat & nai/nas & tu(ku)n &  \\
{West} & {Defoid} & tā & lɛ(n), ne, je & tu(n) /lú(n) &  \\
{West} & {Edoid} & sa & ni & sien/su(w)on &  \\
{West} & {Idomoid} & ta/la & n{\`{ɛ}}, ndo, he & do/lo, ho, ro/rwo &  \\
{West} & {Igboid} & t{\'{ɔ}} & n{\'{ɔ}} & sé &  \\
{West} & {Nupoid} & ta & na/ni & tun/tnu/ tsun, hi? & hi?\\
{West} & {Akpes}\il{Akpes} & ī-sās(ì) & ī-nīŋ(ì) & ī-ʃōn(ì), *tan &  \\
{West} & {Oko}\il{Oko} & {\`{ɛ}}-ta & {\`{ɛ}}-na &   & ù-pi\\
{West} & {Ikaan}\il{Ikaan} & tāːs/h-rāhr & nāʲ/nā/náh{\textsubtilde{í}} & tòːn/h-r{\`{ʊ}}ːn/ s{\textsubtilde{ò}}n/c{\textsubtilde{\`{ʊ}}}n{\textsubbar{v}} &  \\
\lspbottomrule
\end{tabularx}
\end{table}

This is the most stable group of numerical terms within BC. It comprises the roots \textit{*tat} ‘3’, \textit{*nai} ‘4’, and \textit{*tan/} \textit{ton} ‘5’ that are very well-known among the specialists in NC studies. Issues pertaining to the phonetic realization of their reflexes will be treated in the next chapter. 

\clearpage
\subsubsection{‘Six’}\label{sec:3.1.4.4}
\begin{table}
\caption{\label{tab:3:57}BC stems and patterns for `6'}


\begin{tabularx}{\textwidth}{llllllQ}
\lsptoprule

East & {Bantu} & 3 redupl. &   &   &   &  \\
East & {Bantoid} {(–Bantu)} & < 3 redupl.? &   &   &   &  \\
East & {Cross} & 3+3 & 5+1 & diʔ &   &  \\
East & {Jukunoid} &   & 5+1 &   &   &  \\
East & {Kainji} & < 3? &   &   & ci(hi)n & tel\\
East & {Platoid} & 3PL & 5+1 &   &   &  \\
West & {Defoid} &   &   &   &   & fà\\
West & {Edoid} & 3PL, 3+3 &   &   &   &  \\
West & {Idomoid} &   &   & riwi/rowo & ji & hili\\
West & {Igboid} &   &   &   & ʃ{\H{i}}i &  \\
West & {Nupoid} &   & 5+1 &   &   &  \\
West & {Akpes}\il{Akpes} &   & 5+1? &   &   &  \\
West & {Oko}\il{Oko} &   & 5+1 &   &   &  \\
West & {Ikaan}\il{Ikaan} &   &   &   &   & \mbox{h-ɾàdá/} sàdá/ sàrá\\
\lspbottomrule
\end{tabularx}
\end{table}

As the table shows, there was probably no primary Proto-Benue-Congo\il{Proto-Benue-Congo} root for ‘six’. Two alternative patterns are traceable, namely ‘3PL’ (‘3 redupl.’, ‘3+3’) and ‘5+1’. Other forms are marginal. The phonetic resemblance of the Kainji and Igboid forms is noteworthy. 

\clearpage
\subsubsection{‘Seven’}\label{sec:3.1.4.5}
\begin{table}
\caption{\label{tab:3:58}BC stems and patterns for `7'}


\begin{tabularx}{\textwidth}{llQlXl}
\lsptoprule

East & {Bantu} & càmbà\newline (<**c/saN+2?) &   &   &  \\
East & {Bantoid} {(–Bantu)} & samba\newline (5+2?) &   &   &  \\
East & {Cross} & 5+2 &   &   &  \\
East & {Jukunoid} & 5+2 &   &   &  \\
East & {Kainji} & 5+2 &   &   &  \\
East & {Platoid} & 5+2 &   &   & 4+3\\
West & {Defoid} &   & byē &   &  \\
West & {Edoid} &   & ghie? &   &  \\
West & {Idomoid} & 5+2 &   & renyi &  \\
West & {Igboid} &   &   &   & saà\\
West & {Nupoid} & 5+2 &   &   &  \\
West & {Akpes}\il{Akpes} &   &   &   & ī-tʃēnētʃ(ì)\\
West & {Oko}\il{Oko} & ú-f{\'{ɔ}}mb{\`{ɔ}}r{\`{ɛ}} (5+2) &   &   &  \\
West & {Ikaan}\il{Ikaan} &   &   & \mbox{h-ránèʃì} ('6+1) &  \\
\lspbottomrule
\end{tabularx}
\end{table}

A primary root for ‘seven’ is also indistinguishable. The form *\textit{camba}/\textit{samba} may have lost any phonetic resemblance to its Benue-Congo prototype *7=5+2 in Proto-Bantoid\il{Proto-Bantoid}. The Defoid and Edoid forms are phonetically comparable (a shared innovation?).

\clearpage
\subsubsection{‘Eight’}\label{sec:3.1.4.6}
\begin{table}
\caption{\label{tab:3:59}BC stems and patterns for `8'}


\begin{tabularx}{\textwidth}{llQlll}
\lsptoprule

East & {Bantu} & nai-nai\newline (< 4 redupl.) &   &   &  \\
East & {Bantoid} {(–Bantu)} & na-nai\newline (< 4 redupl.) &   &   &  \\
East & {Cross} & 4+4 &   &   &  \\
East & {Jukunoid} & 4 redupl. & 5+3 &   &  \\
East & {Kainji} &   & 5+3 & ro/ru & kunle(v)/kunlo\\
East & {Platoid} & 4 redupl. & 5+3 &   &  \\
West & {Defoid} &   &   & jo/ro &  \\
West & {Edoid} & 4 redupl. &   &   &  \\
West & {Idomoid} &   & 5+3 &   &  \\
West & {Igboid} &   & 5+3 &   &  \\
West & {Nupoid} &   & 5+3 &   &  \\
West & {Akpes}\il{Akpes} & ā-nāānīŋ(ì)\newline (4 redupl.) &   &   &  \\
West & {Oko}\il{Oko} & {\`{ɔ}}-n{\'{ɔ}}k{\'{ɔ}}-nɔk{\'{ɔ}}nɔ\newline(4 redupl.?) &   &   &  \\
West & {Ikaan}\il{Ikaan} & nàːnáʲ\newline (4 redupl.) &   &   &  \\
\lspbottomrule
\end{tabularx}
\end{table}
In this case, the pattern \textit{*nai} ‘4’ >\textit{*na(i)-nai} ‘8’ fits the reconstruction better than its alternative. The similarity between Kainji and Defoid is peculiar and may be due to innovations. 

\clearpage
\subsubsection{‘Nine’}\label{sec:3.1.4.7}
\begin{table}
\caption{\label{tab:3:60}BC stems and patterns for `9'}


\begin{tabularx}{\textwidth}{lllXlll}
\lsptoprule

East & {Bantu} &   &   & bùá &   &  \\
East & {Bantoid} {(–Bantu)} &   &   & bukV &   &  \\
East & {Cross} & 5+4 & 10-1 &   &   &  \\
East & {Jukunoid} & 5+4 &   &   &   &  \\
East & {Kainji} & 5+4 & 10-1 &   &   & jiro\\
East & {Platoid} & 5+4 & 10-1 &   &   & 12-3, tu(ku)n\\
West & {Defoid} &   &   &   & sá(n) & dà\\
West & {Edoid} &   &   &   & cien/sin &  \\
West & {Idomoid} & 5+4 &   &   &   &  \\
West & {Igboid} &   &   &   &   & totu /tolu \\
West & {Nupoid} & 5+4 &   &   &   &  \\
West & {Akpes}\il{Akpes} &   &   &   &   & {\`{ɔ}}-kp{\={ɔ}}l{\`{ɔ}}ʃ(ì)\\
West & {Oko}\il{Oko} &   & ù-b{\'{ɔ}}{\`{ɔ}}r{\`{ɛ}} (10-1) &   &   &  \\
West & {Ikaan}\il{Ikaan} &   & h-ráòʃì (X-1) &   &   &  \\
\lspbottomrule
\end{tabularx}
\end{table}

The rightmost column of the table includes many isolated forms (among them some primary ones). The term *\textit{buka}, which may appear as an important BC innovation, is reconstructed for Proto-Bantoid\il{Proto-Bantoid}. In addition, the pattern ‘9=5+4’ is distinguishable in Proto-Benue-Congo\il{Proto-Benue-Congo}. Like for '8', Defoid and Edoid forms closely resemble each other. 

\clearpage
\subsubsection{‘Ten’}\label{sec:3.1.4.8}
\begin{table}
\caption{\label{tab:3:61}BC stems for `10'}
\small 
\begin{tabularx}{\textwidth}{ll lQQQQQ}
\lsptoprule
East & {Bantu} &   & k{\'{ʊ}}mì/ kámá &   &   &   &  \\
East & {Bantoid} {(–Bantu)} & fu & kum/ kam &   &   &   &  \\
East & {Cross} & fo? &   & kpo/ kop & ʔo? & job &  \\
East & {Jukunoid} &   &   & wo? & kur? & jwe &  \\
\tablevspace
East & {Kainji} & pwa &   & kup/ kpa & kur? &   & turu\\
East & {Platoid} &   &   & kop & gur/ wur &   &  \\
West & {Defoid} &   &   &   & gwá &   &  \\
\tablevspace
West & {Edoid} &   &   & kpe & gbe &   &  \\
\tablevspace
West & {Idomoid} & (fu `20') &   &   & gwo/ wo & jwo &  \\
West & {Igboid} &   &   &   &   &   & ɗì/ri/ li\\
West & {Nupoid} &  (hu ’20) &   &   & wo &   &  \\
\tablevspace
West & {Akpes}\il{Akpes} &   &   &   &   & \mbox{ī-yōf(ì),} *t-ēfī &  \\
West & {Oko}\il{Oko} & {\`{ɛ}}-fɔ &   &   &   &   &  \\
\tablevspace
West & {Ikaan}\il{Ikaan} & ò-pú/ fú &   &   &   &   &  \\
\lspbottomrule
\end{tabularx}
\end{table}

This is a heterogeneous group of forms. The root *\textit{pu/fu} attested in both Eastern and Western BC is the most likely candidate for BC reconstruction. However, it is missing from Bantoid, for which the term \textit{*kum/kam} is reconstructable. The latter form must be a Bantoid innovation. However, assuming that the second consonant may have undergone nasalization in Proto-Bantoid\il{Proto-Bantoid}, this form is comparabale to a number of other roots, suggesting that \textit{*kup/} \textit{kop} should be reconstructed for Eastern Benue-Congo. As the table shows, other roots should not be neglected either. They will be treated in combination with the evidence from other NC branches. 

\clearpage
\subsubsection{‘Twenty’}\label{sec:3.1.4.9}
\begin{table}
\caption{\label{tab:3:62}BC stems and patterns for `20'}
\small

\begin{tabularx}{\textwidth}{lQ llQQll}
\lsptoprule

East & {Bantu} & 10*2 &   &   &   &   &  \\
East & {Bantoid} {(–Bantu)} & 10*2 &   &   &   &   &  \\
East & {Cross} &   & *ti/ci? &   &   & dip? &  \\
East & {Jukunoid} &   &   &   &   & `body' (di) &  \\
East & {Kainji} & 10*2 & ʃín/ʃík &   &   &   &  \\
East & {Platoid} &   &   &   &   &   & 12+8\\
West & {Defoid} &   &   & gwú(n), gbolo &   &   &  \\
West & {Edoid} &   &   & gie/jie, gboro &   &   &  \\
West & {Idomoid} &   &   &   & fu/hu, su? &   &  \\
West & {Igboid} &   &   & ɡw{\~{\'ʊ}} /ɣʰ{\={ʊ}}, kpɔrɔ &   &   &  \\
West & {Nupoid} &   & ʃi &   & hu? &   &  \\
West & {Akpes}\il{Akpes} &   &   & {\={ɔ}}-ɡb{\={ɔ}}(l{\={ɔ}}) &   &   &  \\
West & {Oko}\il{Oko} &   &   & {\'{ɔ}}-ɡbɔlɔ &   &   &  \\
West & {Ikaan}\il{Ikaan} &   &   & ù-ɡb{\'{ɔ}}r{\'{ɔ}} (<'sack'), *à-ɡbá &   &   &  \\
\lspbottomrule
\end{tabularx}
\end{table}

It is highly unlikely that the Proto-BC term followed the pattern reconstructed for Proto-Bantoid\il{Proto-Bantoid} (*’20=10*2’). In all likelihood there was no root for ‘twenty’ in Proto-BC at all. It should be noted that numerous branches of Western BC use the root (\textit{g})\textit{bolo} (possibly related to the lexical root with the meaning ‘sack’) to make ‘twenty’. A shorter root (*\textit{gba/} \textit{gwe}) is reconstructable in the same Western BC branches as well. Its source is likely lexical: it is well-known that the term for ‘twenty’ in the NC languages often goes back to lexemes with the meaning ‘man’, ‘leader’, and ‘body’ (cf. Jukonoid). The resemblance between the reconstructed Idomoid and Nupoid forms is noteworthy. However, these forms might be etymologically related to the term for ‘ten’.

\clearpage
\subsubsection{‘Hundred’ and ‘thousand’}\label{sec:3.1.4.10}
\begin{table}
\caption{\label{tab:3:63}BC stems and patterns for `100'  and `1000'}


\begin{tabularx}{\textwidth}{lQ QllQ}
\lsptoprule

~ & \textbf{~} & `100' & `100' & `100' & `1000' \\
\midrule
East & {Bantu} &   & kámá & gànà, tʊa, jànda & nùnù, p{\`{ʊ}}mb{\`{ɩ}}, k{\'{ʊ}}t{\`{ʊ}}\\
East & {Bantoid} {(–Bantu)} & 20*5? & kam? & gbi? ki? &?\\
East & {Cross} & 20*5 &   &   &  \\
East & {Jukunoid} & 20*5 &   &   & < Hausa\\
East & {Kainji} & ? &   &   &  \\
East & {Platoid} & ? &   &   &  \\
West & {Defoid} & 20*5 &   &   &  \\
West & {Edoid} & 20*5 &   &   & du, ria/li\\
West & {Idomoid} & 20*5, 10*10 &   &   &  \\
West & {Igboid} & 20*5 &   &   & puk(w)u\\
West & {Nupoid} & 20*5 &   &   & ?\\
West & {Akpes}\il{Akpes} & ī-ɡb{\'{ɔ}} ʃōnì~(20*5) &   &   &  \\
West & {Oko}\il{Oko} &   &   & í-pì &  \\
West & {Ikaan}\il{Ikaan} & à-ɡbá à-h-ruǹ(20*5) &   &   &  \\
\lspbottomrule
\end{tabularx}
\end{table}

If Proto-Benue-Congo\il{Proto-Benue-Congo} did not have the term for ‘twenty’, it probably did not have the term for ‘hundred’ either, because the only pattern it could follow is *’100=20*5’. In this respect the Proto-Bantoid\il{Proto-Bantoid} innovation (*\textit{kam}) is noteworthy. It resembles another Proto-Bantoid innovation, namely the term for ‘ten’ (*\textit{kum}/\textit{kam}), which is hardly a coincidence. The possibility that in the cases of ‘ten’ and ‘hundred’ we are dealing with alignment by analogy cannot be excluded. This could explain the irregular nasalization of the root for ‘ten’, cf. Proto-Bantoid\textit{*kup} ‘10’ ${\rightarrow}$ \textit{kum} by analogy with \textit{*kam} ‘100’.
The term for ‘thousand’ was certainly nonexistent in BC.

\subsubsection{Summary}
Taking this into account, the segmental reconstruction of the Proto-BC numeral system may be suggested (\tabref{tab:3:64}).

\begin{table}
\caption{\label{tab:3:64}Proto-Benue-Congo\il{Proto-Benue-Congo} numeral system (*)}


\begin{tabularx}{\textwidth}{lXll}
\lsptoprule

{1} & ni, kin/cin (<k-in?), gbon,\newline (o-)di(n)?, (o-)ti? & {7} & 5+2\\
{2} & ba-di /ba-ji, pa? ba(i)? & {8} & 4 redupl.\\
{3} & tat & {9} & 5+4\\
{4} & nai & {10} & pu/fu, kup/kop, gwo /jwo\\
{5} & tan/ton & {20} & absent? gwa/gwe? < ‘person’?\\
{6} & 3PL/3 redupl./3+3, 5+1 & {100} & absent? 20*5\\
\lspbottomrule
\end{tabularx}
\end{table}

This table gives an overview of the BC evidence that will be used for further comparison with other NC branches. 


\chapter{Noun classes in the Niger-Congo numeral systems} 
In most NC languages, the numeral stems are combined with noun class markers. More often we are dealing with the dependent markers of noun classes (in particular, in the numeral `1', as well as in the numerals `2'-'5') in those languages where there is an agreement between numerals and nouns. But class markers appear in many languages, ​​even without any agreement. For example, when counting, numerals are often used in a nominal function and include obligatory markers of noun classes. In this case, numerals as nouns and, on the other hand, numerals as proper numerals can have different class markers (and different roots). Thus, in Likile\il{Likile} (Bantu C) \textit{li-yɔɔ} `ten' (Cl5), \textit{mo-túkú} \textit{/} \textit{mi-} {`dozen'} (\textsc{cl}3 / \textsc{cl}4) \citep{Carrington1977}.

In many languages, nominal classes in numerals are easily recognized. In other languages, as a result of phonetic processes at the junction of CM and numeral stem and/or as a result of changes by analogy in the paradigm of numerals, it might be difficult to determine which noun class is included in the numeral, although we can distinguish a lexical root. Thus, in Lulamoji\il{Lulamoji} (Bantu J) in some derivated numerals (\textit{mm-kágá} `60' < \textit{mu-káagá} `6' and \textit{mm-sáánvu} `70' < \textit{mu-sáánvu} `7'), an obscure CM \textbf{mm-} is observed (Larry Hyman, p.c.). It is not homorganic, so we can not treat it as \textsc{cl}10. Meanwhile, in the majority of other languages within this group, it is clearly \textsc{cl}10 which is observed in these forms: cf. for example, in Gwere\il{Gwere} \textit{{\textsubdot{n}}kɑ:} \textit{gɑ} `60', \textit{{\textsubdot{n}}sɑnvú} `70', cf. \textit{lù-kúmì} `1000' / \textit{{\textsubdot{n}}kúmì, } \textit{βìβírì} `2000' (clearly \textsc{cl}11 / \textsc{cl}10)\footnote{The irregular allomorph of \textsc{cl}.10 may have arisen as a result of a change by analogy with the basic numeral `6' and `7': \textbf{N} homorganic (\textsc{cl}.10) in these derivated forms > \textbf{mm-} by analogy with \textbf{mu-} (\textsc{cl}.3).}. Such cases are not sufficiently dramatic for reconstruction.

However, in a number of languages in synchrony we do not have sufficient criteria to decide whether we are dealing with the root of a numeral or with combinations of a root with an archaic noun class marker. In other words, we cannot isolate the root, and therefore we cannot compare it with the roots of other languages. E.g. we posess no formal proof that the Kobiana\il{Kobiana} (Atlantic) term \textit{sana} ‘four’ is composed of \textbf{sa}- being a class prefix adduced to the lexical base (-\textbf{na}). This base is only distinguishable by means of external comparison, although this method alone is admittedly  insufficient, since the Kobiana term may as well be interpreted as an innovation (\textit{sana} ‘4’). 

In more complicated cases, it should be assumed that a noun class affix replaced one of the segments of the stem, thus becoming an integral part. The Wolof\il{Wolof} (Atlantic) numerals provide a good example of this phenomenon. The following numerical terms are attested in Wolof at the synchronic level: \textit{ñaar} ‘2’, \textit{ñett} ‘3’, \textit{ñeent} ‘4’. Normally the noun class affixes are not included in the lexical base in Wolof, so synchronically we do not have to interpret the first consonant of Wolof numerals as a prefix. At the same time, there are a number of important arguments in favor of the presence of the frozen prefix *\textbf{Ñ-} in the Wolof numerals. First, these are the only numerals that agree in the \textbf{Ñ} class, being one of the two plural noun classes preserved in Wolof (cf. \textit{fukk} ‘ten’ which agrees in the singular noun class B). Secondly, the forms \textit{yaar} ‘2’ and \textit{yett} ‘3’ (with the initial consonant being identical to the other plural noun class - \textbf{Y}) which agree in the \textbf{Y} class have been preserved in some Wolof dialects. Finally, as we hope to demonstrate below, the unification of numerals by class in Niger-Congo languages is characteristic of terms covering the sequence from ‘two’ to ‘four’. Thus, in the diachronic perspective, the consonants in question should be viewed as characteristic of class markers rather than stem segments. However, if this assumption is correct, we are forced to conclude that these markers have been integrated into the stem, having replaced the original initial consonants of the terms in question, the more so that VC-roots are uncommon in Wolof (numerical roots most probably had CVC structure, see \citealt{PozdniakovRobert2015}: 615-616). This means that the Wolof terms are of little significance for the reconstruction of the terms for ‘2-4’ in Proto-Atlantic\il{Proto-Atlantic}.

Most of the issues (theoretical ones included) that have complicated our reconstruction while studying noun classes in the families and branches of Niger-Congo pertain to the relationship of noun classes and numerals at the synchronic level. These problems are often left aside in the grammatical descriptions and do not attract sufficient attention from linguists. I am not aware of any work which discusses them systematically. Meanwhile, I am sure this question is worthy of attentive study because it reveals additional characteristics of noun class systems. 

The first five numerals in Niger-Congo usually agree with nouns, for example in Sereer\il{Sereer}: \textit{o-koor} \textit{o-leng} ‘one man’, \textit{a-koy} \textit{a-leng} ‘one monkey’, \textit{Ø-naak} \textit{Ø-leng} ‘one cow’.  In some languages and branches of the macro-family, the inventory of numerals that show agreement is reduced. 

As noted, the noun class marker may appear in numerals in some contexts which are not related to the agreement. 

1. For instance, for counting, the majority of languages include a class marker (CM); moreover, different numerals may have different affixes. For example, in Biafada\il{Biafada} for the numerals ‘1’, ‘6-7’ the class \textbf{N} is used, for ‘2-4’ the class \textbf{bi-}, \textbf{ɡə} – for ‘5’, \textbf{Ø} – for ‘8-9’, \textbf{ba} – for ‘10’. 

A lot of languages use CM in numerals starting from ‘6’ and higher, that is in the numerals that do not show agreement in class, and not only in counting. For example, in Manjak\il{Manjak} \textit{ngə-bʊs} \textit{ng{\`{ə}}-təb} ‘two dogs’ (agreement), \textit{ngə-bʊs} \textit{{\`{ʊ}}-ntaja} ‘ten dogs’ (lack of agreement, numeral ‘10’ with CL \textbf{{\`{ʊ}}-} is used in an independent form). 

The choice of the noun class for numerals in the two aforementioned contexts (in counting forms, and in numerals with no agreement) represent a very interesting case which I will outline hereinafter. 

2. The interaction between noun classes and numerals cannot be limited to the aforementioned contexts. Noun classes emerge as well in derived numerals. The three main cases will be highlighted as follows. 

Firstly, in the majority of Niger-Congo languages (and, apparently, even in Proto-Niger-Congo\il{Proto-Niger-Congo}) the numeral `8' was formed from `4' by the reduplication of the first syllable of the original root *CL-\textit{na(h)i} ‘4’ > *CL-\textit{na-na(h)i} ‘8’. Often the noun class marker of `4' and `8' coincides, but sometimes they do not. A question therefore arises: which factors define the choice of a noun class in a derived numeral? 

Secondly, the Niger-Congo languages use compound numerals extensively, as do the majority of languages in the world. For example, the numeral ‘40’ is formed following the model ‘40’ = ‘4*10’ (in many Bantu languages, for instance) or ’40’ = ‘20*2’ (in the majority of Atlantic languages). The latter model is based on finger-counting, when two hands and two feet give a sum of 20. The numeral ‘20’ goes back to the lexeme ‘chief’ or ‘man’. In these languages the numeral ‘15’ is often formed following the model ‘two hands and one foot’. This model is well known and is discussed in the literature. However, the question of the choice of noun class in the first and second formative of these compound numerals was often left aside. Meanwhile, this question needs more clarification. The following questions will be discussed in the present study. 

In a compound numeral, for example, ‘20’ = ‘10*2’, the class marker is often absent in the second formative. For example, in Bomwali\il{Bomwali} (Bantu, A80) we have: \textit{Ø-kamɔ} ‘10’ (\textsc{cl}9)\footnote{For a reader who is not aware of the tradition of Bantu linguistics, it is necessary to explain that in Bantu languages there is a stable inventory of noun classes, each having a fixed number. The ongoing numeration of Bantu was found useful for the study of noun classes in Niger-Congo in general, where the numeration of classes of non-Bantu languages represents a concrete etymological hypothesis. If a scholar assigns the number '6' to the class -\textbf{ɗam} of Fula\il{Fula} (Atlantic language), it means that etymologically it should be related to the class \textbf{*ma} (CL 6N) of Proto-Bantu\il{Proto-Bantu}.} ,  \textit{ɓe-ɓa}  ‘2’ (\textsc{cl}2),  \textit{mɔ-kamɔ} \textbf{\textit{Ø}}\textit{-ɓa} ‘20’. In this type of language, we have additional causes to discuss derivative words rather than syntagms. 

In a compound numeral, both formatives include class markers, for example, ‘20’ = ‘CL-10*CL-2’. The CM can be different or the same in the two formatives: Pinji\il{Pinji} (B30)~\textbf{\textit{n}}\textit{-dzìm{\`{a}}} \textbf{\textit{d{\'{i}}}}\textit{-b{\`{a}}l{\`{e}}} ‘20’ (10*2), Nsong\il{Nsong} (B80) \textbf{\textit{ma}}\textit{-kw{\v{i}}m} \textbf{\textit{m}}\textit{-ɔːl}~ ‘20’ (10*2). In the latter case, a particular type of \textit{agreement} can be observed, that is, the second formative agrees in class with the first formative. 

If in a compound numeral both formatives include class markers, as in ‘20’ = ‘CL-10*CL-2’ then theoretically we can expect that the noun class of the first formative will coincide with the class of the independent numeral `10'. This strategy is very rare. One of the unique examples comes from Moghamo\il{Moghamo} (Grassfields) \textbf{\textit{ì-}}\textit{ɣ{\`{u}}m-b{\={e}}} ‘20’ (\textbf{\textit{ì-}}\textit{ɣ{\`{u}}m} ‘10’,  \textit{{\'{i}}-b{\={e}}}  ‘2’).  In the majority of cases the noun classes of the two formatives do not coincide. For instance, in the same branch of Benue-Congo (Grassfields): Laimbue\il{Laimbue} \textbf{\textit{m{\`{ɨ}}}}\textit{-ɣ{\'{ɨ}}m-b{\`{o}}} ‘20’ (\textbf{\textit{ɨ-}}\textit{ɣ{\'{ɨ}}m} ‘10', \textit{b{\`{o}}} ‘2’), the number `10' changes its class, being part of the first formative of the numeral `20'. The interpretation of this strategy in Niger-Congo languages will be given later. The same problem arises with the second formative. Very rarely does its class coincide with the noun class of the initial numeral (in the present case we deal with the numeral `2'). In the majority of cases it differs. The cause is, as it was already mentioned above, that the second formative agrees with the first one. For example, in the same group of languages (Grassfields): Mundani\il{Mundani} \textbf{\textit{{\`{e}}}}\textit{-ɣɛm} \textbf{\textit{ye}}\textit{-be} ‘20’  (\textit{{\`{e}}-ɣɛm} ‘10’, \textbf{\textit{be}}\textit{-be} ‘2’). In some languages, noun classes of simple and compound forms differ even if agreement is absent. 

3. Finally, the strategy of forming numerals only by the change of the noun class and with no changes in the lexical root represents a real parade of paradigmatic values of noun classes in numerals. This strategy was systematically developed in one zone of Bantu languages, that is zone J (although it can be encountered sporadically in some other Niger-Congo languages). For example, in Chiga\il{Chiga} (Bantu J): \textbf{\textit{ì}}\textit{-}\textit{βìɾ{\'{i}}}~‘2’ > \textbf{\textit{{\`{ɑ}}ː}}\textit{-}\textit{βìɾ{\'{i}}} ‘20’ ;  \textbf{\textit{m{\`{u}}}}\textit{-k{\^{ɑ}}ːɡ{\`{ɑ}}} ‘6’ >  \textbf{\textit{{\textsubdot{ŋ}}}}\textit{-k{\^{ɑ}}ːɡ{\`{ɑ}}} ‘60’, \textbf{\textit{m{\`{u}}}}\textit{-n{\^{ɑ}}ːn{\`{ɑ}}} ‘8’ > \textbf{\textit{kì}}\textit{-n{\^{ɑ}}ːn{\`{ɑ}}} ‘80’. 

It is interesting that the same language combines all three strategies. Thus, in Chiga\il{Chiga}: 

\begin{enumerate}
\item The numeral ‘8’ is formed by reduplication of ‘4’: \textit{{\'{i}}-n{\`{ɑ}}} ‘4’ > \textit{m{\`{u}}-n{\^{ɑ}}ː-n{\`{ɑ}}} ‘8’ (and we can observe the variation of noun classes 5 (\textbf{{\'{i}}}-) and 3 (\textbf{m{\`{u}}}-);
\item The numeral ‘200’ is formed by a word-combination, but not by the combination of ‘100’ and ‘2’ as we would expect. Instead, it is formed by the combination of ‘10’ and ‘2’: \textit{βì-k{\`{u}}mì} \textit{βì-}\textit{β{\'{i}}ɾì} ‘200’ (\textit{ì-k{\'{u}}mì} ‘10’, \textit{ì-β}\textit{ìɾ{\'{i}}}~‘2’). Thus, ‘200’ (\textsc{cl}.\textsc{pl}) is a plural form of ‘10’ and ‘2’ (\textsc{cl}.\textsc{sg}). Furthermore, the second formative agrees in noun class with the first. 
\item The numeral ‘20’ is formed from ‘10’ by changing the noun class exclusively: \textit{{\`{ɑ}}ː-β}\textit{ìɾ{\'{i}}} ‘20’ (\textit{ì-β}\textit{ìɾ{\'{i}}} ‘2’), and by the use of a different noun class, different from the one we find in ‘200’, that is \textsc{cl}.\textsc{pl} \textbf{{\`{ɑ}}ː}-.  
\end{enumerate}

\section{Noun classes in the counting forms of numerals} %1.1

In some Niger-Congo languages, numerals do not have noun class markers in the counting form, but the number of these languages is very low. In the Atlantic family the only language with this feature is Balant\il{Balant}. In the majority of Niger-Congo languages while naming a numeral (for example, in counting) noun class markers are used. These markers may be the same for all numerals, but this is a rare case. More often, for the numerals 1-10 there are three to four different markers (furthermore, special class markers may be used for the numerals ‘20’, ‘100’, ‘200’ and others). 

A fragment of the Tetela\il{Tetela} (C80) numeral system is presented below (\tabref{tab:1:1}.):


\begin{table}
\caption{Tetela\il{Tetela} numerals}
\label{tab:1:1}

\begin{tabularx}{\textwidth}{lXXX}
\lsptoprule

1 & {\'{o}}-tɔy & 9 & di-vw{\'{a}}\\
2 & ha-{\'{e}}nde & 10 & d{\'{i}}-kumi\\
3 & ha-s{\'{a}}tu & 20 & {\'{a}}-kumi {\'{a}}-ende\\
4 & a-n{\'{e}}y & 90 & {\'{a}}-kumi di-vw{\'{a}}\\
5 & a-t{\'{a}}nu & 100 & lo-k{\'{a}}m{\'{a}}\\
6 & a-sam{\'{a}}le & 200 & n-k{\'{a}}m{\'{a}} y-{\'{e}}nd{\'{e}}\\
7 & e-samb{\'{ɛ}}{\'{ɛ}}l{\'{e}} & 1000 & ki-n{\`{u}}nu (y{\'{i}}nŋa)\\
8 & e-n{\'{a}}{\'{a}}n{\'{e}}yi & 2000 & ø-nunu p-{\'{e}}nde\\
\lspbottomrule
\end{tabularx}
\end{table}

We see here a variety of classes as well as plenty of mini-clusters (note the noun class switch that occurs when a number becomes a part of a compound term; this phenomenon is characteristic of the Niger-Congo languages). The terms for ‘one’ (\textbf{{\'{o}}}- class), ‘hundred’ (\textbf{lo}-) and ‘thousand’ (\textbf{ki}-) appear to be isolated on account of their noun class. At the same time, the following groups of terms are distinguishable: ‘2-3’ (\textbf{ha}-), ‘4-6/20’ (\textbf{a}-, «/» refers to the grouping of non-adjacent numerals), ‘7-8’ (\textbf{e}-), and ‘9-10’ (\textbf{di}-).  It should be noted, however, that even in such systems some numerals can be used without noun class markers (‘2000’). 

Three issues need to be mentioned here.

The noun class markers are easily distinguishable in Tetela\il{Tetela}. However, for the majority of the NC languages (especially the non-Bantu ones) this is not the case. The criteria that would allow for distinguishing between the markers and the segments of stems are often lacking, which means that we have no idea which stem in a language under study is to be used for comparative purposes. The situation is even more grave in those numerous cases where an additional class marker is added to a numeral which contains an archaic class marker integrated in a stem.

The mechanism underlying the grouping of numerals into the mini-clusters (by including them in a common noun class) remains virtually unexplored, although it is certainly worthy of investigation and thorough consideration from the theoretical point of view.  What was the motivation behind the use of the class marker \textbf{ha}- with the Tetela\il{Tetela} terms for ‘two’ and ‘three’, while in case of ‘nine’ and ‘ten’ the class marker \textbf{di}- was preferred in this language? The answer to this question is probably not to be sought within the semantics of a given noun class. On closer examination, the choice of a noun class in such distributions is often unmotivated by anything other than the need to formally distinguish a group of numerals (as opposed to other groups). In this respect, this mechanism is very similar to the alignment by analogy as applied to numerals in many languages. This strategy (implying an irregular alteration of a part of a lexical stem) can be compared to a radical surgery, which is never an easy option. Languages with noun classes have less traumatic means to achieve the same result, e.g. by using different noun class markers to distinguish between the groups of numerals. This elaborate marking technique is widely attested in the Niger-Congo languages. The grouping of numerals is typologically interesting as well: some of the groups are fairly common whereas some are quite rare. Moreover, it is probable that these groups were formed independently in different languages: a situation where a pair of closely related languages exhibit radically different grouping and vice versa is not uncommon.

Some numerals are not normally subject to grouping and tend to be marked with a specific noun class, thus standing in opposition to the rest of the numerical terms. The use of this specific class is especially frequent with the terms for ‘one’, ‘hundred’ and ‘thousand’, cf. e.g. specific noun classes observable in the Tetela\il{Tetela} terms for ‘one’ (\textbf{{\'{o}}}- ) and ‘hundred’ \textbf{(lo}-).

Let’s look at the distribution of numerals in noun classes for the languages where this information is available. This observation will be made on a selection of 254 Benue-Congo languages (among these, 166 are Bantu languages, evenly distributed by zones). Our sampling comprises languages that are known to employ noun classes on the numerical terms used in counting.

\subsection{The specific marking of numerals}%1.1.1.
As mentioned above, specific noun classes are used with the terms for ‘one’ and ‘ten’ especially often: 174 languages out of 254 mark the numeral ‘1’ in a distinguished way, and 151 languages mark the numeral ‘10’ separately. 

Examples of systems with the term for ‘one’ being in opposition to the rest of the numerals (marked with a different noun class)\footnote{Considering the fact that numerals ‘2-9’ belong to the same noun class, the numerals ‘6-9’ are not included in Tables 1.2-1.5.} are provided below (\tabref{tab:1:2}).  


\begin{table}
\caption{Specific noun classes in `1'}
\label{tab:1:2}
\fittable{
\begin{tabular}{llllllll}
\lsptoprule

Branch & Language & `1' & `2' & `3' & `4' & `5' & `10'\\
\midrule
J30 & Nyole\il{Nyole} &  \textbf{ndala} & ebiri & edatu & en{\'{e}} & etaanu & eh{\'{u}}mi njereere\\
Defoid & Ede\il{Ede} Ica &  \textbf{ɔk{\~{ɔ}}} & eɟi & ɛta & {\~{ɛ}}{\~{ɛ}} & ɛwu & ɛya\\
Defoid & Ede\il{Ede} (dial.) &  \textbf{{\`{ɔ}}k{\~{ɛ}}} & m{\'{\~ɛ}}d͡ʒì & m{\'{\~ɛ}}ta & m{\'{\~ɛ}}h{\~{ɛ}} & m{\'{ɛ}}h{\'{u}} & m{\'{\~ɛ}}w{\'{a}}\\
Defoid & If{\`{e}}\il{Ifè} &  \textbf{{\`{ɛ}}nɛ} \textbf{/} \textbf{{\`{ɔ}}k{\`{\~ɔ}}} & m{\'{e}}{\`{e}}dzì & m{\'{ɛ}}ɛta & m{\'{ɛ}}ɛr{\~{ɛ}} & m{\'{ɛ}}ɛr{\'{u}} & ma{\'{a}}\\
Mbe\il{Mbe} & Mbe\il{Mbe} &  \textbf{{\'{o}}m{\`{e}}} & b{\'{ɛ}}pʷ{\^{a}}l & b{\'{ɛ}}s{\'{a}} & b{\'{ɛ}}{\~{n}}î & b{\'{ɛ}}tʃ{\^{a}}n & b{\'{ɛ}}fw{\^{ɔ}}r \\
Mbam & Nomaande\il{Nomaande} &  \textbf{ɔmɔt{\'{ɛ}}} & b{\'{e}}fend{\'{i}} & bat{\'{a}}t{\'{ɔ}} & b{\'{e}}ny{\'{i}}se & bat{\'{a}}{\'{a}}n{\'{ɔ}} & b{\'{ɔ}}{\'{ɔ}}h{\'{a}}ta\\
Mbam & Tuotomb\il{Tuotomb} &  \textbf{{\'{ɔ}}m{\`{ɔ}}} & p{\'{ɛ}}f{\'{a}}ⁿd & p{\'{ɛ}}d{\`{a}}{\`{a}}t & p{\'{i}}nìs & p{\'{ɛ}}t{\`{a}}n & pʷ{\'{o}}w{\`{a}}t\\
Mbam & Tuki\il{Tuki} & \textbf{umw{\^{e}}ːsi{\'{i}}} & m{\'{o}}w{\'{a}} & m{\'{o}}t{\'{a}}t{\'{o}} & mw{\'{e}}ːn{\'{e}} & mot{\'{a}}ːn{\'{o}} & mw{\'{a}}b{\'{ɔ}}t{\'{ɔ}}\\
Mbam & Yambeta\il{Yambeta} &  \textbf{{\'{i}}m{\`{u}}ʔ} & m{\'{ɔ}}b{\`{a}}{\`{a}}n & m{\'{ɔ}}d{\'{a}}{\'{a}}d & m{\'{u}}nìʔ & m{\'{ɔ}}t{\'{a}}{\`{a}}n & m{\'{ɔ}}w{\'{a}}d\\
Mbam & Nubaca\il{Nubaca} &  \textbf{p{\`{o}}m{\'{o}}h{\`{o}}} & mʷ{\v{a}}ntʃì & m{\`{u}}t{\'{a}}t & m{\`{u}}ɲ{\'{i}}hì & m{\`{u}}t{\^{a}}ːn & mʷapʷat\\
Mbam & Yangben\il{Yangben} &  \textbf{p{\`{u}}m{\`{o}}m} & m{\'{a}}nd{\`{ɛ}} & mat{\'{a}}t & m{\'{e}}nì & m{\'{a}}t{\`{a}}n & m{\'{a}}t\\
Mbam & Numaala\il{Numaala} & \textbf{b{\`{u}}mʷ{\`{o}}m} & m{\^{a}}ːnd{\`{ɛ}} & m{\'{a}}d{\'{a}}d̥{\`{ɔ}} & m{\'{e}}nî & m{\'{a}}tʰ{\'{a}}n & m{\'{a}}tʰ\\
Mamfe & Denya\il{Denya} &  \textbf{ɡ{\'{ɛ}}m{\^{a}}} & {\'{o}}p{\'{e}}{\'{a}} & {\'{o}}l{\'{ɛ}} & {\'{o}}nì & {\'{o}}t{\`{a}} & {\'{o}}f{\'{i}}{\`{a}}\\
\lspbottomrule
\end{tabular}
}
\end{table}
Examples of one other strategy (the term for ‘ten’ being a noun remains in opposition to the rest of the numerals by means of a noun class) are given in \tabref{tab:1:3}.  


\begin{table}
\caption{Specific noun classes in `10'}
\label{tab:1:3}

\fittable{
\begin{tabular}{llllllll}
\lsptoprule
Branch & Language & `1' & `2' & `3' & `4' & `5' & 10\\
\midrule
S30 & Kgalagadi\il{Kgalagadi} & (bʊ)ːŋwɪ & (bʊ)bɪr{\'{ɪ}} & (bʊ)r{\'{a}}ːrʊ & (bʊ)ːnɛ & (bʊ)tʰ{\'{a}}ːnʊ &  \textbf{lɪʃ{\'{ʊ}}ːmɪ}\\
S10 & Kalanga\il{Kalanga} & (ku{\`{)}}ŋ{\'{o}}mp{\`{e}}l{\'{a}} & (k{\`{u}})bìl{\'{i}} & (k{\`{u}})t{\`{a}}t{\'{u}} & (k{\`{u}})nn{\`{a}}~ & (k{\`{u}})ʃ{\'{a}}n{\`{u}} & \textbf{ɡ{\`{u}}m{\'{i}}}\\
Cross-River & Bete-\il{Bete}Bendi\il{Bete-Bendi} & ìk{\`{e}}n~ & ìf{\`{e}} & ìk{\'{i}}{\'{e}} & ìn{\`{e}} & ìd{\'{i}}{\'{ɔ}}ŋ~ &  \textbf{l{\`{e}}hʷ{\'{o}}}\\
Mbam & Nugunu\il{Nugunu} & ɡ{\'{i}}mmue & ɡ{\'{a}}andɛ & ɡ{\'{a}}dadɔ & ɡ{\'{e}}nni & ɡ{\'{a}}t{\'{a}}anɔ &  \textbf{s{\'{ɛ}}ɔdɔ}\\
Idomoid & Eloyi\il{Eloyi} & {\'{n}}ɡw{\`{o}}nz{\'{e}} & {\'{n}}ɡw{\`{o}}p{\'{o}} & {\'{n}}ɡw{\`{o}}l{\'{a}} & {\'{n}}ɡw{\`{o}}nd{\'{o}} & {\'{n}}ɡwol{\'{ɔ}} &  \textbf{{\'{u}}w{\'{o}}}\\
Jukunoid & Akum\il{Akum} & ~ {\'{a}}jì & ~ af{\`{\~a}} & ~ ata & ~ aɲ{\`{ɪ}} & ~ ac{\'{o}}ŋ &  \textbf{{\={i}}k{\`{u}}r({\`{u}})} \\
Platoid & Tyap\il{Tyap} (Kataf) & ~ anyuŋ & ~ afeaŋ & ~ atat & ~ anaai & ~ afwuon &  \textbf{swak}\\
\lspbottomrule
\end{tabular}
}
\end{table}

Another strategy with the terms covering the sequence from ‘two’ to ‘nine’ being opposed to the terms for ‘one’ and ‘ten’ is characteristic of the languages represented in the table 1.4.


\begin{table}
\caption{Common noun classes for `2'-'9'}
\label{tab:1:4}

\fittable{
\begin{tabular}{llllllll}
\lsptoprule

Branch & Language & \textbf{`1'} & `2' & `3' & `4' & `5' & \textbf{`10'}\\
\midrule
Cross-River & Ebughu\il{Ebughu} &  \textbf{s{\`{ɪ}}ŋ} & ìb{\`{a}} & ìt{\'{ɛ}} & ìnì{\`{a}}ŋ & ìtîŋ &  \textbf{l{\`{u}}ɡ{\`{o}}}\\
Cross-River & Oro\il{Oro} & \textbf{~} \textbf{ki} & ~ {\'{i}}b{\`{a}} & ~ {\'{i}}t{\'{e}} & ~ {\'{i}}nîaŋ & ~ {\'{i}}tiŋ &  \textbf{luɡhu}\\
Cross-River & Usakade\il{Usakade} &  \textbf{tʃ{\`{ɛ}}n} & {\`{m}}b{\`{a}} & {\`{n}}t{\'{a}} & {\`{n}}nì{\`{ɔ}}ŋ & {\`{n}}tʃ{\^{o}}n &  \textbf{n{\`{u}}{\`{o}}p}\\
Cross-River & Leggbo\il{Leggbo} &  \textbf{w{\`{ɔ}}ni} & {\`{a}}fɔŋ & {\`{a}}t{\textsubdot{t}}an & {\`{a}}nnaŋ & {\`{a}}zen~~~ &  \textbf{ dzɔ}\\
Platoid & Ayu\il{Ayu} &  \textbf{ɪdɪ} & afah & ataar & anaŋaʃ & atuɡen &  \textbf{iʃoɡ} \\
Grassfields & Mundani\il{Mundani} &  \textbf{yea-mɔʔ} & bebe & betat & bekpì & bet{\`{\~a}}{\~{\^a}} &  \textbf{{\`{e}}ɣɛm}\\
Igboid & Ekpeye\il{Ekpeye} &  \textbf{ŋìn{\'{ɛ}}} & ɓ{\^{ɨ}}b{\'{ɔ}}~ & ɓ{\'{ɨ}}t{\'{ɔ}}~~ & ɓ{\'{ɨ}}n{\^{ɔ}}~ & ɓ{\'{i}}s{\^{e}} &  \textbf{ɗì~} \\
Tivoid & Ipulo\il{Ipulo} &  \textbf{{\'{e}}m{\`{ɔ}}} & v{\'{i}}{\`{a}}l & v{\'{e}}t{\`{a}}t & v{\'{e}}ɲì & v{\'{e}}t{\`{a}}n &  \textbf{{\'{e}}p{\'{ɔ}}ːt}\\
Isimbi & Isimbi &  \textbf{k{\={e}}n{\={ə}}} & m{\={ə}}r{\={a}}kp{\={ə}} & m{\={a}}k{\={ə}}l{\={ə}} & m{\={o}}ɲ{\={i}} & m{\={a}}t{\={ə}}n{\`{ə}} &  \textbf{b{\={u}}ɣ{\`{u}}}\\
A40 & Bankon\il{Bankon} &  \textbf{(i)y{\v{a}}} & (bi)ɓ{\'{a}} & (bi){\'{i}}y{\^{a}} & (bi)n{\^{a}}n & (bi)t{\'{a}}n &  \textbf{iɓ{\v{o}}m}\\
A80 & Bekwil\il{Bekwil} &  \textbf{w{\'{a}}t} \textbf{/} \textbf{ŋɡ{\'{ɔ}}t}  & e-ɓ{\'{a}} & e-l{\^{ɛ}}l & e-n{\^{a}} & e-t{\^{ɛ}}n & \textbf{k{\v{a}}m}\\
A80 & Koonzime\il{Koonzime} &  \textbf{ɡw{\'{a}}r} & bìb{\'{a}} & bìl{\^{ɛ}}l & bìn{\^{a}} & bìt{\^{ɛ}}n & \textbf{k{\'{a}}m}\\
B20 & K{\'{e}}l{\'{e}}\il{Kélé} &  \textbf{nw{\'{u}}nt{\`{u}}} & b{\`{a}}b{\'{a}} & b{\`{a}}l{\'{a}}l({\`{e}}) & b{\`{a}}n{\'{a}}yì & b{\`{a}}t{\'{a}}n &  \textbf{dy{\'{u}}m({\`{u}})}\\
B20 & Ntumbede\il{Ntumbede} &  \textbf{{\'{i}}w{\'{o}}t{\'{o}}} & b{\'{ə}}b{\`{a}} & b{\'{ə}}r{\'{a}}r{\`{e}}~ & b{\'{ə}}n{\'{a}}y{\`{ɛ}} & b{\'{ə}}t{\'{a}}n{\`{e}} & \textbf{dʒ{\'{o}}m{\`{ɛ}}}\\
J20 & Jita\il{Jita} &  \textbf{kamʷi} & βiβiɾi & βisatu & βina & βitanu &  \textbf{ɛkumi} \\
K20 & Mbunda\il{Mbunda} & \textbf{~} \textbf{cimo}  & ~ vivali & ~ vitatu & ~ viwana & ~ vitanu &  \textbf{likumi} \\
M20 & Ndali\il{Ndali} & \textbf{kamukene} & fi-{\^{w}}iri & fi-tatʊ~~ & fi-na~ & fi-hano & \textbf{kalo{\ᵑ}ɡo}\\
N30 & Nyanja\il{Nyanja} &  \textbf{cim{\'{ɔ}}dzi} & (zi)β{\'{i}}ri & (zi)t{\'{a}}tu & (zi)n{\'{a}}i & (zi)sanu &  \textbf{kʰ{\'{u}}mi} \\
N20 & Tumbuka\il{Tumbuka} &  \textbf{ka-m{\^{o}}za} & tu-{\^{w}}îri~ ~ & tu-t{\^{a}}tu & tu-n{\^{a}}yi & tu-nkʰonde &  \textbf{kʰ{\^{u}}mi} \\
P20 & Makonde\il{Makonde} &  \textbf{i{\'{i}}mo} & mbi{\'{i}}li & nna{\'{a}}tu & nt͡ʃe:ʃɛ & mwa{\'{a}}nu &  \textbf{liku{\'{u}}mi} \textbf{/} \textbf{ku{\'{u}}mi} \\
\lspbottomrule
\end{tabular}
}
\end{table}

At the same time, the terms for ‘one’ and ‘ten’ can form a group opposed (by means of a noun class) to the rest of the numerals (\tabref{tab:1:5}).


\begin{table}
\caption{Common noun classes for `1' and `10'}
\label{tab:1:5}
\fittable{
\begin{tabular}{llllllll}
\lsptoprule
Branch & Language & \textbf{`1'} & `2' & `3' & `4' & `5' & \textbf{`10'}\\
\midrule
Platoid & Ayu\il{Ayu} &  \textbf{ɪ-dɪ} & a-fah & a-taar & a-naŋaʃ & a-tuɡen &  \textbf{i-ʃoɡ} \\
Tivoid & Ipulo\il{Ipulo} &  \textbf{{\'{e}}-m{\`{ɔ}}} & v-{\'{i}}{\`{a}}l & v{\'{e}}-t{\`{a}}t & v{\'{e}}-ɲì & v{\'{e}}-t{\`{a}}n &  \textbf{{\'{e}}-}\textbf{p{\'{ɔ}}ːt}\\
Bantu-A40 & Bankon\il{Bankon} &  \textbf{(i)y{\v{a}}} & (bi)ɓ{\'{a}} & (bi){\'{i}}y{\^{a}} & (bi)n{\^{a}}n & (bi)t{\'{a}}n &  \textbf{i-ɓ{\v{o}}m}\\
Bantu-M20 & Ndali\il{Ndali} & \textbf{ka-mukene} & fi-{\^{w}}iri & fi-tatʊ~~ & fi-na~ & fi-hano & \textbf{ka-}\textbf{lo{\ᵑ}ɡo}\\
\lspbottomrule
\end{tabular}
}
\end{table}
With the exception of the terms for ‘one’ and ‘ten’, a specific marking of numerals by means of a noun class is rarely attested. A specific noun class (different from noun classes in other numerals) was found in only 6 languages for the numeral ‘3’, and in only 7 for the numeral ‘4’. It should be noted, however, that a specific  marker is often employed for the terms within the sequence from ‘six’ to ‘nine’, e.g. the term for ‘nine’ bears a specific noun class marker in the 151 languages under study.

\subsection{The grouping of numerals by noun class} %1.1.2.
Adjacent numerals are more often grouped by their noun classes. Among different numeral grouping types, several are diffused across all main branches of Benue-Congo. I will list 15 of the more frequent groupings of numerals and illustrate each of them with an example. These groupings are reported in \tabref{tab:1:6}. 

Even limiting \tabref{tab:1:6} to 15 groupings demonstrates the fact that some numerals (for example, `2') are grouped by noun class more often than other numerals (for example, `8'). By analyzing the whole table of groupings (reported in Appendix A-B), the following observations can be made regarding each numeral. 

\textbf{Numeral} \textbf{`1'.} Groupings of the numeral `1' are relatively rare: the majority of languages, obviously, prefer to oppose ‘1’ to all other numerals. In case it is grouped with other numerals, the most frequent grouping is within the first five (‘1-5’) or six (‘1-6’) numerals. In the analyzed database there are four languages which differentiate the first two numerals ‘1-2’. For instance, Ngoreme\il{Ngoreme} (Bantu-E10): \textit{e-mʷe} ‘1’, \textit{e-beɾe}~~‘2’, but \textit{i-satɔ} ‘3’, in Gitonga\il{Gitonga} (S60) \textit{mw{\'{e}}y{\`{o}}} ‘1’, \textit{mbìlì} ‘2’, but \textit{dzì-n{\'{a}}} ‘4’. 

\textbf{Numeral} \textbf{`2'.} The numeral `2' reveals the maximum predisposition to groupings. The most frequent are: ‘2-5’ and ‘2-6’. The grouping ‘2-4’ is significantly less frequent but remains present in the majority of Bantu zones and in other groups of Benue-Congo languages. 

\textbf{Numeral} \textbf{`3'.} `3' is often found in groupings but is very rarely opposed by noun class to `2'. However, some very interesting examples exist. For example, Mbuun\il{Mbuun} (Bantu-B80): \textit{umw{\'{ɛ}}s} ‘1’, \textit{by{\v{ɛ}}l} ’2’, \textit{{\'{i}}-t{\'{a}}r} ‘3’, \textit{{\'{i}}-na} ‘4’, \textit{{\'{i}}-t{\^{a}}n} ’5’. It is worth mentioning that grouping of ‘3-8’ and ‘3-10’ were not encountered in any of the languages examined. 

\textbf{Numerals} \textbf{‘4’} \textbf{and} \textbf{‘5’.} The only frequent grouping involving ‘4’ is ‘2-4’ (except groupings that include four numerals or more) and for ‘5’ it is ‘2-5’ or ‘2-5/10’. The grouping ‘5-9’ was encountered only in five languages and the grouping ‘5-10’ and ‘5-8’ (in combination ‘5-8/10’ – only in one language. The lack of a frequent grouping of ‘5-9’ can seem even more strange because in many languages numerals ‘6-9’ are based on 5 (moreover, this type of derivational model can be reconstructed for Proto-Bantu\il{Proto-Bantu} and, perhaps, for Proto-Benue- Congo, with the sole exception of the numeral ‘8’ which was apparently formed from ‘4’). Another unexpected case is the lack of grouping for ‘5/10’, that is the lack of a specific class for ‘5’ and ‘10’, considering the fact that in many languages ‘10’ is formed from ‘5’. This model was encountered only in one dialect of Eggon\il{Eggon}: \textit{{\`{o}}-tn{\'{o}}} ‘5’, \textcyrillic{а}nd \textit{{\'{o}}-kpo} ‘10’, while in other numerals the noun class is not marked (I am not aware whether the different tone on the prefix indicates a different noun class). 

\textbf{Numeral} \textbf{‘6’.} A high number of groupings of ‘6-9’ is natural. In many languages it becomes ‘6-8’ because of the specific derivation of the number ‘9’. In contrast, groupings ‘6-10’ are very rare. 

\textbf{Numeral} \textbf{‘7’.} It is worth mentioning the frequent grouping of ‘7-8’ (21 languages). We are dealing not with one concrete class in Benue-Congo but rather a similar way of marking the numerals ‘7’ and ‘8’. In the three examples reported in \tabref{tab:1:3} the presumably common \textsc{cl}7 (Cilungu\il{Cilungu} \textbf{tʃ{\'{i}}-}, Sakata\il{Sakata} \textbf{ke-}, Xhosa\il{Xhosa} \textbf{si-}) was found, in other languages a number of different classes can be encountered (\tabref{tab:1:7}). 

\textbf{Numerals} \textbf{‘8’,} \textbf{‘9’,} \textbf{‘10’.} The same charactetistic is typical for the frequent groupings of ‘8-9’ and ‘9-10’, shown in Tables 1.8-9. 


\begin{sidewaystable}
\caption{The most frequent groupings of numerals based on noun classes in Benue-Congo languages}
\label{tab:1:6}

\fittable{
\begin{tabular}{lp{1cm}lllllllllllll}
\lsptoprule

 \textbf{Grouping} & \textbf{Number} \textbf{of} \textbf{languages} & \textbf{Entire} \textbf{grouping} & \textbf{BC} \textbf{branch} & \textbf{Language} & \textbf{‘1'} & \textbf{‘2'} & \textbf{‘3'} & \textbf{‘4'} & \textbf{‘5'} & \textbf{‘6'} & \textbf{‘7'} & \textbf{‘8'} & \textbf{‘9'} & \textbf{‘10'}\\
 \midrule
\textbf{2-5} & 58 & 1,2-5,6,7-8,9,10 & Bantu-F10 & Cilungu\il{Cilungu} & tʃ{\`{o}}{\'{o}}ŋ{\'{a}} & v{\'{i}}-{\'{i}}l{\'{i}} & v{\'{i}}-t{\'{a}}t{\`{u}} & v{\'{i}}-nì & v{\'{i}}-s{\'{a}}{\'{a}}n{\`{o}} & m{\`{u}}-t{\`{a}}{\`{a}}nd{\'{a}} & tʃ{\'{i}}-n{\'{i}}{\'{i}}mb{\'{a}}l{\'{i}} & tʃ{\'{i}}-n{\'{a}}{\'{a}}nì & f{\'{u}}{\'{u}}nd{\'{i}}ìmb{\`{a}}l{\'{i}} & {\'{i}}-k{\'{u}}mì\\
 \textbf{2-6} & 42 & 1,2-6,7-8,9,10 & Bantu-C40 & Sakata\il{Sakata} & n{\'{e}}mo & i-p{\'{e}} & i-sar & i-ni & i-tsir & i-soŋ & ke-ʃo & k{\'{e}}-n{\'{e}} & leva & j{\~{o}}\\
 \textbf{2-4} & 24 & 1,2-4,5,6,7,8,9,10 & Bantu-C50 & Paɡibete\il{Paɡibete} & moti & e-ɓale & e-salo & e-kwaŋane & ɓumoti & motoɓa & sambo & mwambe & libwa & zomi\\
 \textbf{2-9} & 22 & 1,2-9,10 & Grassfields & Mundani\il{Mundani} & yea-mɔʔ & be-be & be-tat & be-kpì & be-t{\`{\~a}}{\~{\^a}} & be-nt{\`{u}}a & be-s{\`{\~a}}{\`{\~a}}mbe & be-f{\`{\~a}}{\~{a}} & be-b{\`{ə}}ʔa & {\`{e}}-ɣɛm\\
 \textbf{7-8} & 21 & 1,2-6,7-8,9-10 & Bantu-S40 & Xhosa\il{Xhosa} & ɲ{\`{ɛ}} & m-bìn{\'{i}} & n-tʼ{\'{a}}tʰ{\`{u}} & *n-n{\`{ɛ}} ? & n-tɬʼ{\`{a}}n{\`{u}} & n-tʼ{\'{a}}nd{\'{a}}tʰ{\`{u}} & s{\'{i}}-ǁʰ{\`{ɛ}}ŋǁ{\`{ɛ}} & s{\'{i}}-b{\textsubumlaut{\`{ɔ}}}z{\'{ɔ}} & l{\'{i}}-tʰ{\`{ɔ}}ɓ{\'{a}} & lî-ʃ{\^{u}}mì\\
 \textbf{6-9} & 20 & 1-5,6-9,10 & Ekoid & Nde-Ndele\il{Nde-Ndele} & n-dʒi & m-ba & n-sa & n-nɛ & n-dɔːn & a-sighasa & a-simma & a-neɡhane & a-sima-wobo & wobo\\
 \textbf{9-10} & 16 & 1,2-5,6,7,8,9-10 & Platoid & Lijili\il{Lijili} & lō̥ & à-bē̥ & à-tʃé̥ & a-nàró̥ & à-só̥ & mì-nzí & mú-tá & rúnó̥ & zà-tʃé̥ & zà-bè̥\\
 \textbf{1-6} & 15 & 1-6,7-8,9,10 & Bantu-E10 & Simbiti\il{Simbiti} & ka-mʷe & ka-βeɾe & ka-tatɔ & ka-nnɛ & ka-taanɔ & ka-saⁿsaβa & mu-hu{\ᵑ}ɡatɛ & mɔ-naanɛ & kɛⁿda & i-kɔmi\\
 \textbf{6-8} & 15 & 1-5,6-8,9-10 & Bantu-F30 & Nilamba\il{Nilamba} & ka-mwe & ka-beli & ka-tatu & ka-nee & ka-l{\'{a}}no & mu-tandatu & mup-unɡate & mu-naana & kyenda & kyumi\\
 \textbf{2-10} & 14 & 1,2-10 & Mbe\il{Mbe} & Mbe\il{Mbe} & {\'{o}}m{\`{e}} & b{\'{ɛ}}-pʷ{\^{a}}l & b{\'{ɛ}}-s{\'{a}} & b{\'{ɛ}}-{\~{n}}î & b{\'{ɛ}}-tʃ{\^{a}}n & b{\`{ɛ}}-s{\^{e}}s{\'{a}}r & b{\`{ɛ}}-t{\^{a}}n{\`{e}}b{\'{ɛ}}pʷ{\^{a}}l & b{\`{ɛ}}-{\~{n}}îb{\`{ɛ}}{\~{n}}î & b{\'{ɛ}}-t{\^{a}}n{\`{e}}b{\'{ɛ}}{\~{n}}î & b{\'{ɛ}}-fw{\^{ɔ}}r\\
 \textbf{1-5} & 14 & 1-5,6,7,8,9,10 & Grassfields & Ghomala\il{Ghomala} & y{\'{ə}}-m{\={u}}ʔ & y{\'{ə}}-pw{\'{ə}} & y{\'{ə}}-t{\'{a}} & y{\'{a}}-pfʉ{\`{ə}} & y{\'{ə}}-t{\^{ɔ}} & nt{\`{ɔ}}k{\'{ə}} & sɔmbw{\'{ə}}ə & h{\v{ɔ}}m & v{\`{ʉ}}ʔ{\'{ʉ}} & ɣ{\v{a}}m\\
 \textbf{8-9} & 12 & 1/7,2-6,8-9,10 & Bantu-H30 & Nɡonɡo\il{Nɡonɡo} & m-wisi & b-wol & b{\'{e}}-tat & be-wan & b{\'{e}}-tan & be-saman & ns-ambwadi & ke-nan & ke-bva & {\'{e}}-kwom\\
 \textbf{1-10} & 9 & 1-10 & Defoid & Ayere\il{Ayere} & {\`{\~i}}-k{\~{\v{a}}} & ì-dʒì & {\={i}}-t{\={a}} & {\~{\={i}}}-j{\~{\={e}}} & {\~{\={i}}}-t{\'{\~u}} & ì-f{\`{a}} & {\={i}}-dʒʷ{\={i}} & {\={i}}-r{\={o}} & {\~{\={i}}}-d{\~{\^a}} & {\={i}}-ɡʷ{\'{a}}\\
 \textbf{7-9} & 9 & 1,2-5,6/10,7-9 & Idomoid & Alago\il{Alago} & {\'{o}}-je & {\`{e}}-p{\`{a}} & {\`{e}}-ta & {\`{e}}-n{\`{ɛ}} & {\`{ɛ}}-hɔ & ì-hirì & {\`{a}}-hap{\`{a}} & {\`{a}}-hat{\'{a}} & {\`{a}}-h{\'{a}}n{\`{ɛ}} & ì-ɡʷ{\'{o}}\\
 \textbf{2-3} & 9 & 1,2-3,4-6/9,7-8/10 & Cross-River & Eleme\il{Eleme} & {\`{n}}-nɛ & {\`{ɔ}}-bɛrɛ & {\`{ɔ}}-taa & {\`{ɛ}}-t{\'{a}}ale & {\`{e}}-w{\`{o}} & {\`{ɛ}}-ʔ{\`{ɔ}}r{\`{ɔ}} & {\`{a}}-ʔ{\`{a}}r{\`{a}}b{\`{a}} & {\`{a}}-ʔaataa & {\`{e}}-siraʔ{\`{o}} & {\`{a}}-ʔ{\`{o}}\\
\lspbottomrule
\end{tabular}
}
\end{sidewaystable}
\begin{enumerate}
\item The first column contains a stable grouping of numerals illustrated by an example. The second column indicates the number of languages which have this grouping (out of 254 languages under consideration). The rows in the table are organized in decreasing order. The third column lists all the groupings based on the noun class for a concrete language. Groupings of the adjacent numerals are indicated by a hyphen. Groupings of non adjacent numerals are indicated by a slash. Thus, the formula in the third column of the last row can be interpreted as follows: in Eleme\il{Eleme} there are three groupings of numerals – ‘2-3’ (class \textbf{{\`{ɔ}}-}), ‘4-6’ and ‘9’ (class \textbf{{\`{ɛ}}-}), and also ‘7-8’ and ‘10’ (class \textbf{{\`{a}}-}).
\end{enumerate}


\begin{table}
\caption{Groupings of `7'-'8' by noun classes}
\label{tab:1:7}
\fittable{
\begin{tabular}{llllll}
\lsptoprule
Branch & Language & ‘6’ & \textbf{‘7’} & \textbf{‘8’} & ‘9’\\
\midrule
Bantu-B70 & Teke-Tyee\il{Teke-Tyee} & b{\'{i}}s{\'{ɛ}}mɛnɛ & \textbf{n-tsaama} & \textbf{m-pw{\'{ɔ}}mɔ} & Ow{\'{a}}\\
Bantu-C80 & Tetela\il{Tetela} & asam{\'{a}}le & \textbf{e-samb{\'{ɛ}}{\'{ɛ}}l{\'{e}}} & \textbf{e-n{\'{a}}{\'{a}}n{\'{e}}yi} & Divw{\'{a}}\\
Bantu-J30 & Nyore\il{Nyore} & bisasaba & \textbf{mu-safu} & \textbf{mu-nane} & Sienda\\
Platoid & Yeskwa\il{Yeskwa} & {\`{e}}nc{\'{i}} & \textbf{t{\`{o}}-nv{\`{a}}} & \textbf{t{\'{o}}-nd{\'{a}}t} & ty{\'{u}}{\^{o}}r{\'{a}}\\
Cross-River & Eleme\il{Eleme} & {\`{ɛ}}ʔ{\`{ɔ}}r{\`{ɔ}} & \textbf{{\`{a}}-ʔ{\`{a}}r{\`{a}}b{\`{a}}} & \textbf{{\`{a}}-ʔaataa} & {\`{e}}siraʔ{\`{o}}\\
\lspbottomrule
\end{tabular}
}
\end{table}


\begin{table}
\caption{Groupings of `8'-'9' by noun classes}
\label{tab:1:8}
\fittable{
\begin{tabular}{llllll}
\lsptoprule
Branch & Language & ‘7’ & ‘8’ & ‘9’ & ‘10’\\
\midrule
Bantu-B10 & Myene\il{Myene} & {\`{o}}-ɾw{\'{a}}ɣ{\'{e}}n{\^{o}} & \textbf{{\`{e}}-n{\'{a}}n{\'{a}}yì} & \textbf{{\`{e}}-n{\'{o}}ɣ{\`{o}}mì} & ì-ɣ{\'{o}}m{\'{i}}\\
Bantu-B20 & Sake\il{Sake} & bì-t{\'{a}}n{\`{ɛ}}n{\`{ɛ}}bìb{\'{a}} & \textbf{rì-mw{\^{a}}mbì} & \textbf{rì-bv{\`{u}}w{\'{ɔ}}} & dʒ{\'{u}}m{\`{u}}\\
Bantu-B80 & Mpiin\il{Mpiin} & n-s{\'{a}}mw{\^{ɛ}}ːn & \textbf{b{\'{i}}-n{\'{a}}n} & \textbf{b{\'{i}}-vwa} & kub\\
Bantu-H10 & Kikongo\il{Kikongo} & s{\`{a}}mb{\'{u}}w{\`{a}}lì & \textbf{{\'{i}}-n{\`{a}}n{\`{a}}} & \textbf{{\'{i}}-v{\`{u}}w{\`{a}}} & k{\'{u}}mì\\
Bantu-B80 & Songo\il{Songo} & n-sambwar & \textbf{ki-nan} & \textbf{ki-va} & kwim\\
Bantu-J40 & Nande\il{Nande} & eri-r{\'{i}}nda & \textbf{om{\'{u}}-nani} & \textbf{omw-{\'{e}}nda} & er{\'{i}}-k{\'{u}}mi\\
Bantu-J50 & Tembo\il{Tembo} & βi-ɾ{\'{ɪ}}nda & \textbf{m{\'{u}}-nanɛ} & \textbf{mw-ɛnda} & {\'{ɛ}}-kumi\\
Grassfields & Ngomba\il{Ngomba} & samb{\'{a}} & \textbf{y{\'{e}}-n{\'{e}}-fom} & \textbf{ye-ne-pf{\'{u}}ʔ{\'{u}}} & ne-ɡ{\'{ʉ}}m\\
\lspbottomrule
\end{tabular}
}
\end{table}

\begin{table}

\caption{\label{tab:1:9} Groupings of `9'-'10' by noun classes}
\begin{tabularx}{\textwidth}{XXXXX}
\lsptoprule

Branch & Language & ‘8’ & \textbf{‘9’} & \textbf{‘10’}\\
\midrule
Bantu-B70 & Teke-Tyee\il{Teke-Tyee} & m-pw{\'{ɔ}}mɔ & \textbf{o-w{\'{a}}} & \textbf{o-kw{\'{u}}{\'{u}}mu}\\
Bantu -C40 & Budza\il{Budza} & mo-n{\'{a}}n{\'{a}}ye & \textbf{li-bw{\'{a}}} & \textbf{ly-{\'{o}}mo}\\
Bantu -C80 & Tetela\il{Tetela} & e-n{\'{a}}{\'{a}}n{\'{e}}yi & \textbf{di-vw{\'{a}}} & \textbf{d{\'{i}}-kumi}\\
Bantu -G60 & Hehe\il{Hehe} & m-nane & \textbf{nyi-ɡonza} & \textbf{nyi-chumi}\\
Bantu -J60 & Rundi\il{Rundi} & umu-na{\'{a}}ni & \textbf{i-tʃe{\'{e}}nda} & \textbf{i-tʃ{\'{u}}mi}\\
Platoid & Lijili\il{Lijili} & rúnó̥ & \textbf{zà-tʃé̥} & \textbf{zà-bè̥}\\
\lspbottomrule
\end{tabularx}
\end{table}

\clearpage
\section{Noun classes in derived (reduplicated) numerals}%1.2.

Reduplication is widely attested as a means of constructing numerical compounds in NC. This is especially applicable to the pattern ‘8 = 4 redupl.’ which, as we hope to demonstrate below, can be reconstructed at the Proto-Niger-Congo\il{Proto-Niger-Congo} level. Another common pattern (attested, however, with a somewhat lesser degree of frequency) is ‘6 = 3 redupl.’. Three main strategies pertaining to the use of the noun classes are employed within this derivation scenario:


\begin{enumerate}
\item Reduplicated terms preserve the class marker of the source-term in both segments, cf. e.g. Ndoe\il{Ndoe} (Ekoid) \textit{be-ra} ‘3’ > \textit{be-ra-ba-ra} ‘6’, \textit{be-ne} ‘4’ > \textit{be-ne} \textit{be-ne} ‘8’; in Alege\il{Alege} (Cross-River) \textit{{\'{e}}-cɛ} ‘3’ > \textit{{\'{e}}-ce-e-ce} `6'.
\item The original class marker is preserved in only the first segment of the reduplicated form, and omitted in the second: Okpamheri\il{Okpamheri} (Edoid) \textit{e-sa} ‘3’ > \textit{e-sa-Ø-sa} ‘6’, \textit{e-ni} ‘4’ > \textit{e-ni-Ø-ni} ‘8’.
\item Finally, the class marker of the first segment of the reduplicated form is different from that of its source-form: Kwa\il{Kwa} (Ekoid) \textit{e-sa} ‘3’ > \textit{a-sa-ka-su} ‘6’, \textit{i-ni} ‘4’ > \textit{a-ni-ka-ni} ‘8’.
\end{enumerate}
The number of these strategies is reduced to two in cases where a derived term is non-separable (e.g. derived by partial reduplication). In such cases, the class marker of the source-term can be either employed (Kikuyu\il{Kikuyu} \textit{i-tatu} ‘3’  >  \textit{i-tatatu} ‘6’), or not (Vinza\il{Vinza} \textit{ka-}\textit{ne} ‘4’ > \textit{mu-}\textit{nane} ‘8’). 

We might expect that while forming ‘8’ from ‘4’, the singular class of the latter would be switched to the plural class of the former. In Bantu languages, however, this is not the case. Apparently already in Proto-Bantu\il{Proto-Bantu} we should reconstruct the derivational model  \textbf{\textit{*ì-n{\`{a}}ì}} ‘4’ (\textsc{cl}.\textsc{sg}.5) > \textbf{\textit{*m{\`{ʊ}}-n{\`{a}}n{\`{a}}ì}} ‘8’ (\textsc{cl}.\textsc{sg}.3). However, from an etymological point of view, the class \textbf{mu-} represents the reflex of the class 6B.\textsc{pl} and not a reflex of the class 3.\textsc{sg} in Niger-Congo. This question raises an additional and very important topic which cannot be examined in the present study (the arguments in favor of class 6B.\textsc{pl}  \textbf{mu} in Proto-Niger-Congo\il{Proto-Niger-Congo} can be found in \citealt{Pozdniakov2013}). 

\textbf{Bantu} \textbf{languages.} The following presents partial data on the numeral system in Myene\il{Myene} (B10)\footnote{Thanks to Odette Ambouroué for some clarifications and for a profiatable discussion on noun classes in Myene\il{Myene}.}~(\tabref{tab:1:10}).



\begin{table}
\caption{\label{tab:1:10} Myene \il{Myene}{ numerals}}
\begin{tabularx}{\textwidth}{lXlX}
\lsptoprule

1. &*N-m{\`{ɔ}}ɾì (> m{\`{ɔ}}ɾì) & \\
2. &*N-b{\`{a}}nì (> mb{\`{a}}nì)  		& 20. &{\`{a}}-ɣ{\'{o}}m {\'{a}}-mb{\'{a}}nì      (10*2)\\
3. &*N- ɾ{\'{a}}ɾ{\`{o}} (> tʃ{\'{a}}ɾ{\'{o}}) 		& 30. & {\`{a}}-ɣ{\'{o}}m {\'{a}}-ɾ{\'{a}}ɾ{\`{o}}\\
4. &*N-n{\'{a}}yì  (> n{\'{a}}yì)  		& 40. & {\`{a}}-ɣ{\'{o}}m {\'{a}}-n{\'{a}}yì\\
5. &{\`{o}}-t{\'{a}}n{\'{i}} 				& 50. & {\`{a}}-ɣ{\'{o}}m {\'{a}}-t{\'{a}}nì\\
6. &{\`{o}}-ɾ{\'{o}}w{\'{a}} 				& 60. & {\`{a}}-ɣ{\'{o}}m {\'{o}}-ɾ{\'{o}}w{\`{a}}\\
7. &{\`{o}}-ɾw{\'{a}}-ɣ{\'{e}}-n{\'{o}}m{\`{o}} (6+1) 	& 70. & {\`{a}}-ɣ{\'{o}}m {\'{o}}-ɾw{\'{a}}-ɣ{\'{e}}n{\^{o}}\\
8. &{\`{e}}-n{\'{a}}-n{\'{a}}yì~   (2*4) 		& 80. & {\`{a}}-ɣ{\'{o}}m {\'{e}}-n{\'{a}}-n{\'{a}}yì \\
9. &{\`{e}}-n{\'{o}}-ɣ{\`{o}}mì   (10–1) 		& 90. & {\`{a}}-ɣ{\'{o}}m {\'{e}}-n{\'{o}}-ɣ{\`{o}}mì\\
10.& ì-ɣ{\'{o}}m{\'{i}} ~ 			& 100.& *N-k{\'{a}}m{\'{a}}. \\
   &           						& 200.& k{\'{a}}m{\'{a}} mb{\'{a}}n{\'{i}}\\
\lspbottomrule
\end{tabularx}
\end{table}

First of all, it is interesting to highlight a variety of noun classes in the left column of the table and their uniformity in the right one. In the numerals from 1 to 10, the system includes four different singular noun classes: \textbf{N-} (\textsc{cl}9) – ‘1-4’, \textbf{{\`{o}}-} (\textsc{cl}3) – ‘5-7’ (the numeral ‘7’ is formed as ‘6+1’, where \textit{n{\'{o}}m{\`{o}}} means «the only one, the same»), \textbf{{\`{e}}-} (\textsc{cl}7) – ‘8-9’ (the numeral ‘8’  is a reduplicated form of ‘4’, the numeral ‘9’ is formed as ‘9 = 10 – 1’) and finally, \textbf{ì-} (\textsc{cl}5) – ‘10’. A homorganic nasal can be quite reliably reconstructed in ‘1-4’, sometimes appealing to indirect characteristics. For example, in \textit{tʃ{\'{a}}ɾ{\'{o}}} ‘3’ the nasal is absent but in Myene\il{Myene} \textbf{tʃ-} is not a reflex of \textbf{*t.}  In this language  \textbf{*t-} > \textbf{r-}, as can also be seen in the second formative of ‘30’. The initial \textbf{tʃ-} can be traced back to \textbf{*N-r-}.

In numerals of dozens only \textsc{cl}6 \textbf{à-}  is used, which is one of the plural classes (with a collective meaning). An interesting detail: in ’20’ – ‘50’ the second formative agrees with the first one in noun class  (\textbf{{\'{a}}-}), and in ‘60’ – ‘90’ there is no agreement (the second formative maintains noun classes which mark the units as in independent forms; its high tone is due to the high tone in the preceding root \textit{ɣ{\'{o}}m}). 

Non-derived numeral ‘100’ belongs, as ‘1’, to the singular class \textsc{cl}9. Does the second formative of ‘200’ agree with the first one? It is impossible to say, because the noun classes of both formatives coincide when used singularly. 

Finally, it is possible to formulate the principle of derivation with reference to the noun classes: the numeral ‘10’, being a formative of numerals ‘20’ – ‘90’, maintains its meaning but changes the singular noun class to a plural noun class following the most standard \textsc{sg} {\textasciitilde} \textsc{pl} correlation in the language. For \textsc{cl}.\textsc{sg}.5 (\textbf{ì-} in Myene\il{Myene}) which is expressed through \textit{ì-}\textit{ɣ{\'{o}}m{\'{i}}} ~‘10’, the standard correlate is \textsc{cl}.\textsc{pl}.6 (\textbf{{\`{a}}-}). Concerning the second correlate (units), it agrees with the first one (dozens) in the numerals that even in independent use show agreement with nouns (in Bantu numerals ‘1-5’ show agreement with nouns). For this reason in numerals ‘20’–‘50’ units from ‘2’ to ‘5’ agree with ‘10’ in its plural form and in ‘60’–‘90’ second formatives ‘6’-‘9’ do not show agreement. 

If we confront the numeric characteristics of simple and derived forms, the formation of numerals in Myene\il{Myene} can be represented by \textsc{sg} > \textsc{pl}-\textsc{pl} and numerals ‘60’ – ‘90’ by \textsc{sg} > \textsc{pl}-\textsc{sg}.

This system is quite typical for Bantu languages, although the variation is considerable. The main variations are illustrated in the Table (1.11), including languages only from the zone J: 

\begin{table} 
\caption{\label{tab:1:11} Number patterns in derived numerals}

\fittable{
\begin{tabular}{llllllll}
\lsptoprule

\textsc{sg} > \textsc{sg}-\textsc{pl} & 10 > 200 & \textsc{cl}5 > 5-8 & Hema\il{Hema} & 10 &  ikumi & 200 & ~ikumi bibiri \\
\textsc{sg} > \textsc{sg}-\textsc{pl} & 1000 > 2000 & \textsc{cl}11 > 11-8 & Hema\il{Hema} & 1000 &  rukumi & 2000 &  rukumi bibiri\\
\textsc{sg} > \textsc{pl}-\textsc{pl} & 2 > 20 & \textsc{cl}5 > 6-6 & Gundu\il{Gundu} & 3 &  ìs{\'{a}}t{\'{ʊ}} & 30 & makumi ɡasatʊ\\
\textsc{sg} > \textsc{pl}-\textsc{pl} & 100 > 200 & \textsc{cl}5 > 6-6 & Shi\il{Shi} & 100 &  iɡana & 200 & ~maɡ{\'{a}}na abiri\\
\textsc{sg} > \textsc{pl}-\textsc{pl} & 10 > 200 & \textsc{cl}5 > 8-8 & Chiga\il{Chiga} & 10 &  ìk{\'{u}}mì & 200 & ~βìk{\`{u}}mì βìβ{\'{i}}ɾì\\
\textsc{sg} > \textsc{pl}-\textsc{pl} & 100 > 200 & \textsc{cl}7 > 8-8 & Ganda\il{Ganda} & 100 &  tʃìk{\'{u}}mì & 200 & ~bìk{\'{u}}mì bìb{\'{i}}rì\\
\textsc{sg} > \textsc{pl}-\textsc{pl} & 1000 > 2000 & \textsc{cl}7 > 8-8 & Shi\il{Shi} & 1000 &  cihumbi & 2000 &  bihumb{\'{i}} bibiri\\
\textsc{sg} > \textsc{pl}-\textsc{pl} & 1000 > 2000 & \textsc{cl}11 > 10-8 & Ganda\il{Ganda} & 1000 &  l{\`{u}}k{\'{u}}mì & 2000 &  {\textsubdot{ŋ}}k{\`{u}}m{\'{i}} bìb{\'{i}}rì\\
\textsc{sg} > \textsc{pl}-\textsc{sg} & 8 > 80 & \textsc{cl}3 > 6-3 & Shi\il{Shi} & 8 &  m{\'{u}}naani & 80 & m{\'{a}}kumi ɡal{\'{i}} m{\'{u}}naani\\
\textsc{sg} > \textsc{pl}-\textsc{sg} & 9 > 90 & \textsc{cl}3 > 6-3 & Shi\il{Shi} & 9 &  m{\'{u}}{\'{e}}nda & 90 & m{\'{a}}kumi ɡal{\'{i}} m{\'{u}}{\'{e}}nda\\
\textsc{sg} > \textsc{pl}-\textsc{sg} & 1000 > 2000 & \textsc{cl}11 > 10-5 & Soga\il{Soga} & 1000 &  l{\`{u}}k{\'{u}}mì & 2000 &  {\'{\textsubdot{ŋ}}}k{\`{u}}m{\'{i}} ìβ{\'{i}}ɾì\\
\textsc{pl} > \textsc{pl}-\textsc{pl} & 2 > 20 & \textsc{cl}8 > 6-6 & Shi\il{Shi} & 2 &  bibiri & 20 &  m{\'{a}}kumi abiri\\
\textsc{pl} > \textsc{pl}-\textsc{pl} & 3 > 30 & \textsc{cl}8 > 6-6 & Shi\il{Shi} & 3 &  biʃarhu & 30 & m{\'{a}}kumi aʃarhu\\
\textsc{pl} > \textsc{pl}-\textsc{pl} & 4 > 40 & \textsc{cl}8 > 6-6 & Shi\il{Shi} & 4 &  b{\'{i}}ni & 40 & m{\'{a}}kumi ani\\
\textsc{pl} > \textsc{pl}-\textsc{pl} & 5 > 50 & \textsc{cl}8 > 6-6 & Shi\il{Shi} & 5 &  birhaanu & 50 & m{\'{a}}kumi arhaanu\\
\lspbottomrule
\end{tabular}
}
\end{table}


The Hema\il{Hema} example demonstrates that the pluralization of the class for the formation of derived numerals is not mandatory (at least, for hundreds and thousands), although it unconditionally dominates in the languages of this group (Shi\il{Shi}, Chiga\il{Chiga}, Ganda\il{Ganda}, Soga\il{Soga}). If the simple numeral is already marked for plural class (there are examples demonstrating this), the first formative of the derived numeral appears with a new plural class (for example, in Shi). In the combination \textsc{sg} > \textsc{pl}-\textsc{pl} the plural classes in a composed derived numeral can be different (Ganda, derivation ‘1000’ > ‘2000’). 

While forming a word combination from one word, the number of possible combinations of singular and plural classes amounts to eight. As shown in the table, only four of these combinations are actually encountered. No languages show combinations \textsc{sg} > \textsc{sg}-\textsc{sg}, \textsc{pl} > \textsc{sg}-\textsc{sg}, \textsc{pl} > \textsc{sg}-\textsc{pl}, \textsc{pl} > \textsc{pl}-\textsc{sg} This distribution demonstrates how pluralization is used for the formation of numerals of higher rank. This strategy can be systematically found in other branches of Niger-Congo. 

\textbf{Atlantic} \textbf{languages.} In order to be able to compare the principles of derivation of numerals in Bantu and in Atlantic languages systematically, we need to first formulate at least three main differences between these systems.  

First of all, it is important to highlight that the system of Bantu is decimal, which is not typical for other branches of Niger-Congo, nor for other branches of Benue-Congo. The overwhelming majority of Altantic languages are ‘20’-based and not decimal. In these languages, accordingly, ‘40 = 20*2’ (and often ‘100 = 20*5’) and very rarely ‘40 = 10*4’.

Secondly, in Atlantic languages the numerals ‘6-9’ are systematically formed following the model ‘5’ + ‘1, 2, 3, 4’. This model does not permit the change of noun classes for the numerals ‘6-7’ and/or ‘7-9’. The numerals ‘6-9’ maintain all the characteristics of `5' (first formative) and ‘1-4’ (second formative). 

Thirdly, contrary to Bantu, the majority of forms of ‘5’ are formed from the lexeme ‘hand’, maintaining the noun class of this lexeme. In Proto-Bantu\il{Proto-Bantu} ‘hand’ and ‘five’ are reconstructed as different roots. 

The sum of the abovementioned factors explains the fact that noun classes in the numerals ‘6-9’ are of no concern to the present study. Nonetheless, as will be further demonstrated, the main principle of interaction between noun classes and numbers in the numeral system of Atlantic languages is similar to that of Bantu. 

Apparently, derived numerals were already formed following the model ‘40 = 20*2’, ‘60 = 20*3’, ‘80 = 20*4’ in Proto-Atlantic\il{Proto-Atlantic}. Different strategies of agreement are partially shown in the table (\tabref{tab:1:12}, (only the most simple cases were reported): 

\begin{table}

\caption{\label{tab:1:12} Atlantic languages: noun classes in the derived numerals}

\fittable{
\begin{tabular}{lllllll}
\lsptoprule
& '20' & CL & '40' & CL-CL & '2' & CL\\
\midrule
Bijogo\il{Bijogo} & o-joko ('person') & \textsc{sg} & ya-joko ya-n-som & \textsc{pl}-\textsc{pl} & n-som & \textsc{pl}\\
Banjal\il{Banjal} & `ə-vːi~~ ('chief') & \textsc{sg} & `u-vːi ɣuː- βɐ~ & \textsc{pl}-\textsc{pl} & `suː-βɐ & \textsc{pl}\\
Kasa\il{Kasa} & ə-yiː~~ ('chief’) & \textsc{sg} & ku-yiː ku-l̥uβə~ & \textsc{pl}-\textsc{pl} & `si-l̥uβə & \textsc{pl}\\
Bayot\il{Bayot} (Sénégal)\il{Bayot (Sénégal)} & `ə-yi ('chief') & \textsc{sg} & `ku-yi kʊ-ɪɾɪɡːə & \textsc{pl}-\textsc{pl} & `ɪɾɪɡːə & \textsc{pl}\\
Bayot\il{Bayot} (Guinea Bissau)\il{Bayot (Guinea Bissau)} & ɡa-bamɔɡol ('person') & \textsc{sg} & ɡʊ-mɔɡol-ɡʊ-ɾɪɡˑɡa & \textsc{pl}-\textsc{pl} & tɪɡˑɡa & \\
Kwaatay\il{Kwaatay} & butuman & \textsc{sg} & ba-k-an ba-ka-suba & \textsc{pl}-\textsc{pl} & ku-suba & \textsc{pl}\\
Nyun\il{Nyun} Gunyamolo & buruhur & \textsc{sg} & ɟamaŋ ɪ-nakk & \textsc{pl}-\textsc{pl} & ha-nakk & \textsc{pl}\\
Karon\il{Karon} & ə-wi & \textsc{sg} & ə-wi e-supək ~ & \textsc{sg}-\textsc{sg} & su-supək & \textsc{pl}\\
\lspbottomrule
\end{tabular}
}
\end{table}


As demonstrated in \tabref{tab:1:12}, the majority of Atlantic languages within the Bak branch (Bijogo\il{Bijogo}, Banjal\il{Banjal}, Kasa\il{Kasa}, Bayot\il{Bayot}) show that in the numeral ‘40’ (‘60’, ‘80’) the units ‘2’ (‘3’, ‘4’) agree in general according to a plural class and not according to the class of the numeral ‘20’. The same principle is characteristic for the languages of Benue-Congo. In all four abovementioned languages, the formation of ‘40’ is based on the agreement in number as for animated nouns  \textsc{cl}1.\textsc{sg} – \textsc{cl}2.\textsc{pl} (this is very clear especially knowing the etymology of the numeral `20'). 

Pluralization as a form of derivation is used when the form of the numeral ‘20’ is not transparent (Kwaatay\il{Kwaatay} \textit{butuman} ‘20’, unclear etymology, Nyun\il{Nyun} Gunyamolo \il{Nyun Gunyamolo}\textit{buruhur} ‘20’ (possibly from «price + man»); in the numeral `40' lexemes are used with the meaning `people'). In some languages (Karon\il{Karon}) the agreement is based on the singular class of the numeral `20' and not on its plural correlate. 

In Atlantic languages that, like Bantu, systematically follow the decimal system, the pluralization of the class permits the formation of new numerals (more often as word combinations) (\tabref{tab:1:13}): 

\begin{table}
\caption{\label{tab:1:13} Agreement in numerals derived from `10'}

\begin{tabularx}{\textwidth}{lXXX}
\lsptoprule
& \textsc{sg} & \textsc{pl} & \textsc{sg}, \textsc{pl}\\ 
& `10' & `40' & `4'\\
\midrule
Basari\il{Basari} & ɛ-pəxw & ɔ-fəxw ɔ-nɐx & ɓə-nɐx\\
Sua\il{Sua} & Ø-tɛŋi & i-tɛŋi i-naŋ & b-nan\\
\lspbottomrule
\end{tabularx}
\end{table}

In such cases agreement of the formatives can be observed, that is the same noun class is used for dozens and units. In the languages where ‘20’ is formed from ‘10’ (10*2), the units  more often do not show agreement: 

\begin{itemize}
\item Mankanya\il{Mankanya} \textit{i-ɲ{\^{ɛ}}n}~‘10’ (literally: «hands»), \textit{i-ɲ{\^{ɛ}}ŋ} \textit{ŋ{\'{ɨ}}-t{\`{ɛ}}p} ‘20’ (\textit{ŋ{\'{ɨ}}-tɛp} ‘2’), \textit{i-ɲ{\^{ɛ}}ŋ} \textit{ŋɨ-bakɨr} ‘40’ (\textit{ŋɨ-bakɨr} ‘4’); 
\item Jaad\il{Jaad} \textit{pa-ppo} ‘10’, \textit{pa-ppo} \textit{ma-ae} ‘20’(\textit{ma-ae} ‘2’), \textit{pa-ppo} \textit{ma-nne} ‘40’ (\textit{ma-nne} ‘4’), 
\item Palor\il{Palor} \textit{dɐːŋkɛh} ‘10’, \textit{dɐːŋkɛh} \textit{kɐ-nɐk} ‘20’ (\textit{kɐ-nɐk} ‘2’), \textit{dɐːŋkɛh} \textit{niːkiːs} ‘40’ (\textit{niːkiːs} ‘4’). 
\end{itemize}
Even in the following case the use of a plural class for units is possible: Baga Fore\il{Baga Fore} \textit{ɛ-tɛlɛ} ‘10’, \textit{ɛ-tɛlɛ mɛn-di} ‘20’ (\textit{ʃi-di} ‘2’), \textit{ɛ-tɛlɛ mɛ-nɛŋ} ‘40’ (\textit{ʃi-nɛŋ} ‘4’).

Finally, in order to complete the description, hybrid composed forms will be reported, that is when `40' can be traced the root `20' and not `10' but in units where `4' is used and not `2'. This means that in ‘20’ – ‘90’ the root ‘10’ is used, which is different from the main root: 

\begin{itemize}
\item Nalu\il{Nalu} \textit{tɛ} \textit{bɪ-lɛ} ‘10’ (literally: «two hands», \textit{bɪ-lɛ} ‘2’), \textit{alafaŋ} \textit{bi-lɛ} ‘20’, \textit{alafaŋ} \textit{biː-naːŋ} ‘40’ (\textit{biː-naːŋ} ‘4’); 
\item Pepel\il{Pepel} \textit{o-diseɲene} ‘10’, \textit{ŋ-taim} \textit{puɡus} ‘20’ (\textit{ŋ-pugus} ‘2’), \textit{ŋ-taim} \textit{ŋ-uakr} ‘40’ (\textit{ŋ-uakr} ‘4’);
\item Limba\il{Limba} \textit{kɔɔ-hi} ‘10’, \textit{kɔ-ntʰɔ} \textit{ka-aye} ‘20’ (\textit{ka-aye} ‘2’), \textit{kɔ-ntʰɔ} \textit{ka-naŋ} ‘40’ (\textit{ka-naŋ} ‘4’). 
\end{itemize}

In spite of plurality of strategies, the modern systems of agreement of units in the dozens reflect a significant distinction that is characteristic of the two main branches of Atlantic languages – Northern and Bak. Apparently, the proto-languages of the Bak group maintained the principle of agreement which was typical for Proto-Niger-Congo\il{Proto-Niger-Congo}, that is, the agreement of units following the plural correlate of ‘10’ or ‘20’. This principle was lost in the system of the Northern branch, where it can be encountered in only one of the Tenda languages, Basari\il{Basari}. It is also present in Nyun\il{Nyun} Gunyamolo, but in this language, as it is highlighted by different scholars, the numeral ‘20’ (and probably the whole agreement model) is borrowed from Joola\il{Joola} (Bak). 

The model of agreement in ‘200’/ ‘2000’ works in a similar way, as shown in Table (1.14): 


\begin{table}
\caption{\label{tab:1:14} Agreement in `200' and `2000'}

\fittable{
\begin{tabular}{lllllll} 
\lsptoprule
& Language & `100' & `200' & `1000' & `2000' & `2'\\
\midrule
1 & Balant\il{Balant} & ɡeme & ɡ-ɡeme ɡ-sibi & wili mbooda (‘1’) & ɡ-wili ɡ-sibi & -sibi\\
2 & Bayot\il{Bayot} & ɛ-tɛmel & ɪ-tɛmel i-ɾiɡˑɡa & ɛ-ʊlɪ & ɪ-ʊlɪ--i-ɾiɡˑɡa & tɪɡˑɡa\\
3 & Banjal\il{Banjal} & ɛ'-kɛmɛ~~ & ~sɪ'-kɛmɛ `suː-βɐ & `e-uli & `s-uːli `suː-βɐ & `suː-βɐ\\
4 & Kwaatay\il{Kwaatay} & temer & si-temer s{\'{u}}-suba & {\~{e}}-{\~{n}}june & s{\'{u}}-{\~{n}}june s{\'{u}}-suba & k{\'{u}}-suba\\
5 & Baga Fore\il{Baga Fore} & bɔ ben (‘1’) & ʃu-bɔ ʃi-di & tɛnɡbeŋ ben (‘1’) & ʃi-tɛnɡbeŋ ʃi-di & ʃi-di\\
6 & Nalu\il{Nalu} & m-laak & a-laak bi-lɛ & m-ɲaak & a-ɲaak bi-lɛ & bi-lɛ\\
7 & Basari\il{Basari} & kɛmɛ & ɔ-kɛmɛ ɔ-ki & wəli & ɔ-wəli ɔ-ki & ɓə-ki\\
8 & Konyagi\il{Konyagi} & keme & wɐ-keme wɐ-ki & wəli & wɐ-wəli wɐ-hi & wɐ-hi\\
\lspbottomrule
\end{tabular}
}
\end{table}


As observed for dozens, the agreement in ‘200’ and ‘2000’ can be systematically observed only in the languages of the Bak group (languages 1-5 in \tabref{tab:1:14}). In the Northern group this agreement is found only in Basari\il{Basari} (7). Even in Konyagi\il{Konyagi}, the fact of agreement is not clear because in this language the CM of ‘2’ in ‘200’ and ‘2000’ coincides with the CM of \textsc{cl}2 in independent use (for the same reason it is not clear whether we encounter agreement in Baga Foré (5). Moverover, there is no agreement in Nalu\il{Nalu} (6), a language of the same branch. 

In the majority of languages, the noun classes of ‘200’ and ‘2000’ systematically differ from the noun classes of units and dozens. This is typical for Niger-Congo, perhaps because in ‘100’/’200’ and ‘1000’/’2000’ we are often dealing with borrowings.

\textbf{Mel} \textbf{languages.} The present analysis will be limited to the data from one Mel language, that is Temne\il{Temne} (Kərata dialect) collected by David Odden (\tabref{tab:1:15}):


\begin{table}
\caption{\label{tab:1:15} Noun classes in Temne\il{Temne} numerals}
\fittable{
\begin{tabular}{llllll}
\lsptoprule

1 &  p-{\'{i}}n & \\
2 &  pɨ-r{\'{ʌ}}ŋ & 20 &  kɨ-ɡb{\'{a}}\\
3 &  pɨ-sas & 30 & ~ kɨ-ɡb{\'{a}} `t{\'{ɔ}}-f{\'{ɔ}}t (20+10)\\
4 &  pa-nlɛ & 40 & ~ tɨ-ɡb{\'{a}} t{\'{ɨ}}ˈ-r{\'{ɨ}}ŋ (20*2)\\
5 &  tam{\'{a}}{\textsubbridge{t}} 5 (*ta-tam-at) & 50 & ~ = 20*2+10\\
6 &  du-k-{\'{i}}n~~~ (X+1) & 60 & ~ tɨ-ɡb{\'{a}} t{\'{ɨ}}-sas (20*3)\\
7 &  dɛ-r{\'{ɨ}}ŋ~~ ~ (X+2) & 70 & ~ = 20*3+10\\
8 &  dɛ-sas~~~~ (X+3) & 80 & ~ tɨ-ɡb{\'{a}} t{\^{a}}-nlɛ (20*4 )\\
9 &  dɛ-ŋanlɛ (X+4) & 90 & ~ = 20*4+10\\
10 &  tɔ-f{\'{ɔ}}t (< * ta-fu-at) & 100 &  k-ɛm{\'{ɛ}} k-{\'{i}}n              &               200 & ~t-ɛm{\'{ɛ}} t{\'{ɨ}}'-r{\'{ɨ}}ŋ\\
  &                               &1000 &  ʌ-w{\'{u}}l `ŋ-{\'{i}}n             &             2000 &  ɛ-w{\'{u}}l jɛ-r{\'{ɨ}}ŋ\\
\lspbottomrule
\end{tabular}
}
\end{table}

The numerals ‘1-4’ in counting forms belong to \textsc{cl}.\textsc{sg} \textbf{pV-}. The numeral ‘5’ can be traced back to the form with positive meaning of definiteness (\textit{*ta-tam-at}) – as well as 10 (< \textit{*ta-fu-at}), initially having the structure CV-CVC-VC, where CV-  and  -VC  are allomorphs of the noun class in a definite form and CVC is the root \citep[143-144]{Pozdniakov1993}\footnote{It is clear that `5' and `hand' have assonance in the languages of the group. Due to space limitations, it is impossible to explain the complicated emergence of this assonance. Let's also leave aside details on the first formative in the numerals ‘6-9’.}. For us, it is important that the numerals in ‘5’ and ‘10’ can be reconstructed with \textsc{cl}.\textsc{sg} \textbf{ta-}. The non-derived numeral `20' can be traced to \textsc{cl}.\textsc{sg}, and  in particular \textbf{kə-}. The numerals ‘40’ – ‘90’ are formed with the change of the noun class in the first formative to \textsc{cl}.\textsc{pl} \textbf{tə-}. Furthermore, the second formative agrees with the first one in noun class and consequently is also included in the class \textbf{tə-}. That is to say, this is the same derivational model as in Bantu and in Atlantic languages. This model emerges as well in the formation of ‘100’ and ‘200’. In the borrowed form \textit{kɛmɛ} ‘100’ the initial root consonant can be interpreted as a singular CM (the same noun class as in ‘20’). That means that ‘200’ is used as its plural correlate and the original root consonant gives us \textbf{t-}. Finally, the correlation of ‘1000’ {\textasciitilde} ‘2000’ can be interpreted as correlation in number but with a new pair of classes: \textsc{cl}.\textsc{sg} \textbf{ʌ}- {\textasciitilde} \textsc{cl}.\textsc{pl} \textbf{ɛ-}.

\textbf{Gur} \textbf{languages.}  An example of an interesting system from the Ditammari\il{Ditammari} language (Oti-Volta) follows (\tabref{tab:1:16}):


\begin{table}
\caption{\label{tab:1:16} Ditammari\il{Ditammari}: agreement in the derived numerals}

\begin{tabularx}{\textwidth}{lXX}
\lsptoprule

\textsc{sg} & \textsc{pl}-\textsc{pl} & \textsc{sg}\\
\midrule
tɛ-pii-tɛ ‘10’ & si-pi-si-dɛ ‘20’ & dɛɛ-ni ‘2’\\
& si-pi-si-t{\^{a}}adi ‘30’ & t{\^{a}}adi ‘3’\\
& si-pi-si-wɛi ‘90’ & n-wɛi ‘9’\\
di-tu-si-di ‘100’ & yɛ-tu-si-ɛ  yɛ-d{\'{ɛ}}ɛ ‘200’ & dɛɛ-ni ‘2’\\
di-yɔɔ-di ‘1000’ & yɛ-yɔɔ-d-ɛ yɛ-d{\`{ɛ}} ‘2000’ & \\
\lspbottomrule
\end{tabularx}
\end{table}


In this example we can see the correlation of number classes in derivatives and «agreement» between the parts of syntagm in ‘200’ and ‘2000’ using different structures of class markers (prefixes, suffixes, confixes, or the lack of marker). 

Similar formation strategies of derived forms can be found in another language from the Gurma\il{Gurma} group (Oti-Volta), Miyobe\il{Miyobe} (\tabref{tab:1:17}):


\begin{table}
\caption{\label{tab:1:17} Miyobe\il{Miyobe}: noun classes in derived numerals}

\begin{tabularx}{\textwidth}{lXX}
\lsptoprule

\textsc{sg} & \textsc{pl}, \textsc{sg}-\textsc{pl}, \textsc{pl}-\textsc{pl} & \textsc{sg}\\
\midrule
\textbf{kɛ}-fi ‘10’ & \textbf{ɑ}-fɛɛ-r{\'{ɛ}} ‘20’ & -t{\'{ɛ}} ‘2’\\
& \textbf{ɑ}-fɛɛ-nɑ ‘40’ & \textbf{n}-nɑ ‘4’\\
\textbf{p{\'{i}}}-lɛ ‘100’ & \textbf{p{\'{i}}}-lɛ-\textbf{p{\'{i}}}-lɛ \textbf{mɛ}-t{\'{ɛ}} ‘200’ & \\
\textbf{k{\'{u}}}-kotok{\'{u}} ‘1000’ & \textbf{{\'{ɑ}}}-kotok{\'{u}} \textbf{ɑ}-t{\'{ɛ}} ‘2000’ & \\
\lspbottomrule
\end{tabularx}
\end{table}


In ‘20’ (10*2) and in ‘2000’ (1000*2) a plural correlate \textsc{cl}.\textsc{sg} \textbf{kV-} (\textsc{cl}.\textsc{pl} \textbf{{\'{ɑ}}-}) is used. In ‘2000’ the numeral ‘2’ agrees in noun class with ‘1000’ (the root is formed from the word with the meaning ‘sack’). In ‘200’ the reduplication of ‘100’ and a special class marker (\textsc{cl}.\textsc{pl} \textbf{mɛ}) for the formative ‘2’ are used. 

Another language from Gurma\il{Gurma} group Ntcham\il{Ntcham} follows the same standard model (\tabref{tab:1:18}):


\begin{table}
\caption{\label{tab:1:18} Ntcham\il{Ntcham}: noun classes in derived numerals}

\begin{tabularx}{\textwidth}{lXlXlX}
\lsptoprule

& \textsc{sg} & & \textsc{pl}-\textsc{pl} & & \textsc{sg}\\
\midrule
20 &  {\'{m}}-m{\`{u}}{\`{ŋ}}k{\'{u}} & 40 &  ì-m{\`{u}}{\`{ŋ}}k{\'{u}} ì-l{\'{i}} & 2 &  {\`{n}}-l{\'{i}} \\
100 &  di-l{\`{a}}{\'{a}}t{\`{a}}{\`{a}}-l & 200 &  k{\'{u}}-l{\`{a}}{\'{a}}faa-u~ & \\
1000 &  Ø-k{\`{u}}t{\`{u}}k{\'{u}} & 2000 &  Ø-k{\`{u}}t{\`{u}}k{\'{u}}-bì  bì-l{\'{i}} & 2 &  {\`{n}}-l{\'{i}} \\
\lspbottomrule
\end{tabularx}
\end{table}


The numeral `200' is formed from `100' by changing from the singular class to the plural one. 

The existence of similar strategies for use of plural class markers for the formation of numerals of higher rank in different areas of Niger-Congo (Benue-Congo, Atlantic languages, Mel languages and Gur languages) permits us to presume that similar principles of interaction between noun classes and numbers were typical for the system of Niger-Congo as well. There are no traces of derivative pluralization in Kru and Ijo languages, but they can surely be found in Kwa\il{Kwa} languages. I did not manage to find similar strategies in the Adamawa and Ubangi languages, nonetheless traces can be found in Kordofanian languages. 

Here is an example from Koalib\il{Koalib}, a Kordofanian language (\tabref{tab:1:19}): 

\begin{table}
\caption{Koalib\il{Koalib} example}
\label{tab:1:19}

\begin{tabularx}{\textwidth}{lllXlX}
\lsptoprule

&\textsc{sg} && \textsc{pl}-\textsc{pl} && \textsc{sg}\\
\midrule
20. &  t-{\'{u}}ɽì & 40.   &  r-{\'{u}}ɽì r-ìɽ{\`{ɐ}}n            & 2. &  -iɽɐn\\
    &              & 2000. &  {\'{a}}-lep (< arab  ) w-ìɽ{\`{ɐ}}n & 200. &  mîɐ kw-ìɽ{\`{ɐ}}n\\
\lspbottomrule
\end{tabularx}
\end{table}
 


A prefix for the plural class is used for the formation of the numeral 40. The formative ‘2’ in ‘40’ agrees with the formative ‘20’ in the noun class. In ‘200’ the prefix of singular class \textsc{cl}1 is used, which includes animated nouns and borrowings. In ‘2000’, in the formative ‘2’ is used for the prefix \textbf{w-}, a standard agreeement marker for vocalic noun classes. 

Traces of pluralization of noun classes as a means of derivation in numerals can be found in Moro\il{Moro} and Acheron\il{Acheron} (both are Kordofanian languages). 

\largerpage
This distribution gives us sufficient grounds to assume that derivation for the formation of dozens in Niger-Congo was similarly established in Proto-Niger-Congo\il{Proto-Niger-Congo}. 


\clearpage
\section{Noun class as a tool for the formation of numerals}%1.3

Finally, there is one (perhaps the most interesting) strategy for formation for derived numerals. It consists exclusively of changing the noun class for the formation of a derived form. The system from Efik\il{Efik} is partially reported below (\tabref{tab:1:20}): 

\begin{table}
\caption{Efik example}
\label{tab:1:20}

\begin{tabularx}{\textwidth}{lXlX}
\lsptoprule

2 & ~ {\'{i}}-b{\'{a}} & 40 & ~ {\`{a}}-b{\`{a}}\\
3 & ~ {\'{i}}-t{\'{a}} & 60 &   {\`{a}}-t{\'{a}}\\
4 & ~ {\'{i}}-n{\'{a}}ŋ & 80 &   {\`{a}}-n{\`{a}}ŋ\\
\lspbottomrule
\end{tabularx}
\end{table}



In Efik\il{Efik}, as in the majority of Niger-Congo languages, a stable correlation in number \textsc{cl}5.\textsc{sg} {\textasciitilde} \textsc{cl}6.\textsc{pl} can be found: in Efik reflexes of these classes are accordingly \textbf{{\'{i}}-} {\textasciitilde} \textbf{{\`{a}}}\textbf{-}.  A simple change of singular class to plural (with no compound forms and no reduplication) is enough to form ‘40’ from ‘2’, ‘60’ from ‘3’ and ‘80’ from ‘4’. Apparently, this system uses ‘20’ as its primary base.  

The formation of new numerals by a change in noun class can be encountered in some languages of Benue-Congo, including Bantu (\tabref{tab:1:21}): 

\begin{table}
\caption{Benue-Congo examples}
\label{tab:1:21}
\begin{tabularx}{\textwidth}{lllXlX}
\lsptoprule
Bantu-B80 & Tiene\il{Tiene} (Tende) & 4 & ~ i-n{\'{i}}ì & 40 & ~ mu-n{\'{i}}ì\\
Bantu-C40 & Sengele\il{Sengele} & 4 &  {\'{i}}-nɛi & 40 & ~ mo-nɛi\\
Bantu-C90 & Ndengese\il{Ndengese} & 4 &  i-nej & 40 & ~ bo-neji\\
Grassfields & Limbum\il{Limbum} & 4 &  Ø-kj{\`{e}}ː & 40 & ~ {\textsubdot{m}}-kj{\`{e}}ː\\
Edoid & Degema\il{Degema} & 2 &  i-β{\'{ə}} & 40 &  ʊ-β{\'{a}}\\
\lspbottomrule
\end{tabularx}
\end{table}

This technique is mostly used in Bantu languages within the zone J. The data reported in \tabref{tab:1:18} does not necessarily signify that the conceptual base for derivation is the pluralization of original forms. In Tiene\il{Tiene}, Sengele\il{Sengele}, and Ndengese\il{Ndengese}, derived numerals, as well as base numerals, belong to singular noun classes. 

For example, for the languages J10 \textsc{sg} > \textsc{sg} is characteristic for four derivations which can be illustrated by Gundu\il{Gundu} language (\tabref{tab:1:22}).


\begin{table}
\caption{Gundu\il{Gundu} number patterns in the derivations of numerals}
\label{tab:1:22}

\begin{tabularx}{\textwidth}{ll ll ll ll}
\lsptoprule

&8 > 80 && 9 > 90 &&10 > 100 && 10 > 1000\\
&\textsc{cl}3 > CL\textbf{7} && \textsc{cl}3 > CL\textbf{7} && \textsc{cl}5 > CL\textbf{7} && \textsc{cl}5 > CL\textbf{11}\\
\midrule
8 &  m{\`{ʊ}}-n{\'{a}}ːn{\`{e}}{\'{i}} & 9 &  mʷ{\`{e}}ː-ⁿd{\'{a}} & 10 &  {\'{i}}-k{\`{u}}m{\'{i}} & 10 &  {\'{i}}-k{\`{u}}m{\'{i}}\\
80 & ~~ki-naːnei & 90 & ~ kʲeː-ⁿda & 100 &  ki-kumi & 1000 &  ɾu-kumi\\
\lspbottomrule
\end{tabularx}
\end{table}

Other derivations \textsc{sg} > \textsc{sg} can be found occasionally. Apparently, the forms \textit{n-datu} ‘6’ > \textit{tʃ{\'{i}}-ɾatu} ‘60’ (\textsc{cl}9 > \textsc{cl}7) and \textit{m{\'{u}}-nanɛ}  ‘8’ > \textit{l{\'{u}}-nanɛ} ‘80’ (\textsc{cl}3 > \textsc{cl}11) were encountered only in Tembo\il{Tembo} (J50). We can see that the choice of nominal classes differs in different languages, that is, it is not the symbolic semantics of nominal classes that is most important, but rather their paradigmatic modification. 

In Bantu J10-J20 we find a triple derivation model \textsc{cl}5-\textit{kumi} (or \textsc{cl}9-) ‘10’ {\textasciitilde} \textsc{cl}7-\textit{kumi} ‘100’ {\textasciitilde} \textsc{cl}11-\textit{kumi} ‘1000’. Thus in Hema\il{Hema}, \textit{i-kumi} ‘10’ {\textasciitilde} \textit{ki-kumi} ‘100’ {\textasciitilde} \textit{ru-kumi} ‘1000’.

This model can be found in Gur languages as well. In Nothern Nuni\il{Nuni} (Grusi group) dozens are formed exclusively by a change in noun class marker. The derivation from ‘20’ to ‘50’ is realized by the change of one singular class to another: \textit{bì-l{\`{ə}}} ‘2’ > \textit{f{\'{i}}ì-l{\`{ə}}} ‘20’, \textit{bì-tw{\`{a}}{\`{a}}} ‘3’ > \textit{f{\'{i}}ì-tw{\`{a}}{\`{a}}}~‘30’, \textit{bì-nu} ‘5’ > \textit{f{\'{i}}ì-nu} ‘50’. Formation of dozens by a change of class is encountered in some Senufo languages as well. 

However, the derivational model \textsc{sg} > \textsc{pl} is much more active. In the Bantu zone J, six derivations are typical, illustrated by the following examples from Gwere\il{Gwere} (J10) (\tabref{tab:1:23}): 


\begin{table}
\caption{\label{tab:1:23} Gwere\il{Gwere} number patterns in the derivations of numerals}

\fittable{
\begin{tabular}{ll ll ll ll ll ll}
\lsptoprule

\multicolumn{2}{l}{2 > 20}&\multicolumn{2}{l}{3 > 30}&\multicolumn{2}{l}{4 > 40 }&\multicolumn{2}{l}{5 > 50 }&\multicolumn{2}{l}{6 > 60}&\multicolumn{2}{l}{7 > 70}\\
\multicolumn{2}{l}{\textsc{cl}5 > \textsc{cl}6 }&\multicolumn{2}{l}{\textsc{cl}5 > \textsc{cl}6 }&\multicolumn{2}{l}{ \textsc{cl}5 > \textsc{cl}6 }&\multicolumn{2}{l}{\textsc{cl}5 > \textsc{cl}6 }&\multicolumn{2}{l}{ \textsc{cl}3 > \textsc{cl}10 }&\multicolumn{2}{l}{\textsc{cl}3 > \textsc{cl}10}\\
\midrule
2 &  ì-β{\'{i}}ɾ{\'{i}}~~~ & 3 &  ì-s{\'{ɑ}}t{\'{u}} & 4 &  ìː-n{\'{ɑ}} & 5 &  ì-t{\'{ɑ}}ːn{\'{u}} & 6 &  m{\`{u}}-k{\^{ɑ}}ːɡ{\'{ɑ}} & 7 &  m{\`{u}}-s{\'{ɑ}}ˑⁿv{\'{u}}\\
20 &  {\`{ɑ}}ː-β{\'{i}}ɾì & 30 & {\`{ɑ}}ː-s{\'{ɑ}}t{\`{u}} & 40 & {\`{ɑ}}ː-n{\^{ɑ}}~ & 50 & {\`{ɑ}}ː-t{\^{ɑ}}ːn{\`{u}} & 60 & ~ {\textsubdot{n}}-k{\^{ɑ}}ːɡ{\`{ɑ}} & 70 & ~ {\textsubdot{n}}-s{\'{ɑ}}ˑⁿv{\'{u}}\\
\lspbottomrule
\end{tabular}
}
\end{table}


For the numerals ‘20’–‘50’ \textsc{cl}6.\textsc{pl} is used, and for ‘60’–‘70’ \textsc{cl}10.\textsc{pl} is used. These classes demonstrate the correlation in number with the classes \textsc{cl}5.\textsc{sg} and \textsc{cl}3.\textsc{sg} respectively. In at least four languages in zone J, the model \textsc{cl}3.\textsc{sg} > \textsc{cl}10.\textsc{pl} was encountered for ‘9’ > ‘90’. In Gwere\il{Gwere} and Tembo\il{Tembo}, the model \textsc{cl}5 > \textsc{cl}6 is used in derivation ‘2’ > ‘20’: Gwere \textit{ì-β}\textit{{\'{i}}ɾ{\'{i}}}~~‘2’~>  \textit{{\`{ɑ}}ː-β}\textit{{\'{i}}ɾì} ‘20’. 

Only one language, and that is Tembo\il{Tembo}, systematically presents model \textsc{pl} > \textsc{pl} in the derivation \textsc{cl}8.\textsc{pl}  > \textsc{cl}6.\textsc{pl} (\tabref{tab:1:24}):


\begin{table}
\caption{\label{tab:1:24} Tembo\il{Tembo} example}

\begin{tabularx}{\textwidth}{lX lX lX lX}
\lsptoprule

3 &  βi-h{\'{a}}tu & 4 &  β{\'{i}}-nɛ & 5 &  βi-t{\'{a}}nɔ & 7 &  βi-ɾ{\'{ɪ}}nda\\
30 & ~ m{\'{a}}-h{\'{a}}tu & 40 & ~ m{\'{a}}-nɛ & 50 & ~ ma-t{\'{a}}nɔ & 70 & ~ ma-l{\'{i}}nda\\
\lspbottomrule
\end{tabularx}
\end{table}


This model is clearly secondary and was implemented as a result of re-interpretation, atypical of zone J, of classes in numerals ‘2-5’, ‘7’ as plural classes opposed to ‘1’. 

The fourth theoretically possible model, that is \textsc{pl} > \textsc{sg}, has never been encountered in any derivation which can be considered indirect evidence for the idea that the pluralization of numerals of higher rank is one of the key strategies for the formation of derived numerals, as was demonstrated. Nonetheless, this strategy does not explain everything. 

In order to present this elegant mechanism of systematic use of noun classes in the derivation of numerals in greater detail, an example from derivation in Soga\il{Soga} using the roots `10' and `2' will be schematically presented. The root meaning `10' matches in Soga with six different class markers, and the root meaning ‘2’ matches with three of them, as shown in Figure 1.1. 
 







\begin{figure}
\centering
\begin{tikzpicture} 
\node[minimum width=3.5cm] (20)                   {\pozdbox{20.}{(\textbf{\RED{m{\'{ɑ}}}}-kùmì) \textbf{\BLUE{{\`{ɑ}}ː}}-βíɾí}{\textsc{cl}.6  \textsc{cl}.6\\}}; 

\node[minimum width=1.5cm] (10)   [left = of 20]    {\pozdbox[2cm]{10.}{\textbf{\RED{í}}-kùmì}{\textsc{cl}.5}};
\node[minimum width=1.5cm] (2)    [right = of 20]    {\pozdbox{~2.}{\textbf{\BLUE{ì}}-βìɾì}{\textsc{cl}.5}};
\node[minimum width=3.5cm] (200)  [above = of 20]  {\pozdbox{200.}{\textbf{\RED{βí}}-kùmì \textbf{\BLUE{βì}}-βíɾì}{\textsc{cl}.8  \textsc{cl}.8}};
\node[minimum width=3.5cm] (2000) [below = of 20] {\pozdbox{2000.}{\textbf{\RED{{\textsubdot{\'{ŋ}}}}}-kùmí \textbf{ì}-βíɾì\vphantom{{\textsubdot{\'{ŋ}}}}}{\textsc{cl}.10 \textsc{cl}.5?10?}};

\node[minimum width=1.5cm] (100)  [left = of 200]  {\pozdbox[2cm]{100.}{\textbf{\RED{tʃí}}-kùmì}{\textsc{cl}.7}};
\node[minimum width=1.5cm] (1000) [left= of 2000] {\pozdbox[2cm]{1000.}{\textbf{\RED{lù}}-kúmì}{\textsc{cl}.11}};

\node[minimum width=3.5cm] (plpl) [above = of 200,yshift=-10mm] {\bfseries \textsc{pl-pl}};
\node[minimum width=1.5cm] (sg1)  [left = of plpl]  {\bfseries \textsc{sg}};
\node[minimum width=1.5cm] (sg2)  [right = of plpl] {\bfseries \textsc{sg}};

\draw[thick,dotted,red,->] (100)  -- (200);
\draw[thick,dotted,red,->] (10)   -- (20);
\draw[thick,dotted,red,->] (1000) -- (2000);

\draw[thick,dashed,red,->] (10) -- (100);
\draw[thick,dashed,red,->] (10) -- (1000);

\draw[thick,blue,->]  (2.west) |- (200);
\draw[thick,blue,->]  (2.west) |- (20);
\draw[thick,blue,->]  (2.west) |- (2000);

\end{tikzpicture}
\caption{Soga\il{Soga} numerals: derivations by noun classes}
\end{figure}






In the Soga\il{Soga} language the root \textbf{\textit{kumi}} takes part in three forms with singular class and three forms with plural class (one is facultative). In the derivations including forms of different numerals it is visible that the most stable correlations in number are: \textsc{cl}5-\textsc{cl}6, \textsc{cl}7-\textsc{cl}8 and \textsc{cl}11-\textsc{cl}10. At the same time, the choice of \textsc{cl}7 and \textsc{cl}11 for the derivations (as shown in Figure 1.1) seems to be arbitrary. According to Larry Hyman (p.c.) in the dialect Lulamoji\il{Lulamoji}, the archaic form of the numeral ‘1000’ belongs to the the \textsc{cl}11 and not to the \textsc{cl}14 (Hyman: «\textit{óBu-kumí} `1000', older usage»).

The root \textbf{\textit{βiɾi}} does not take part in singular derivates but was found in three derivates where \textbf{\textit{kumi}} is marked by plural class markers. The main derivate from \textit{ì-βìɾì} ‘2’ can function separately outside of the word combination (\textit{{\`{ɑ}}ː-β{\'{i}}ɾ{\'{i}}} ‘20’). In this case, the main correlation in number for the class 5 is used (\textsc{cl}5-\textsc{cl}6). The difference in the class markers \textsc{cl}6 \textbf{mɑ-} and \textbf{ɑː-} (in some dialects \textbf{ga-}) is related to the characteristics of the paradigms of agreement markers. A question about the nature of \textit{ì-β{\'{i}}ɾì} in ‘2000’ emerges. Does it belong to \textsc{cl}5 or is this an homonymous form of the agreement marker in \textsc{cl}10? These questions are very hard to answer because we are dealing with derivational forms of class markers (often homonymous) and we cannot check on the context of agreement in order to test it. 

In fact, the number of classes in numerals (both singular and plural) can be even bigger. In Soga\il{Soga}, a singular form of `8' \textit{m{\`{u}}-n{\'{ɑ}}ː-n{\`{ɑ}}} (\textsc{cl}3) is always formed from the numeral `4' \textit{{\'{i}}-n{\`{ɑ}}}  (\textsc{cl}5). In Mpumpong\il{Mpumpong} (Bantu, A80), the system of numerals includes four different plural noun classes, that is \textsc{cl}8 for units - \textit{t{\^{ɛ}}n} \textit{n{\`{ɛ}}} \textbf{\textit{ì}}\textit{-n{\^{a}}} ‘9’~ (5+4), \textsc{cl}6 – for dozens – \textbf{\textit{m{\`{ɛ}}}}\textit{-k{\`{a}}m m{\`{ɛ}}-mb{\'{a}}} ‘20’(10*2), \textsc{cl}4  for hundreds – \textbf{\textit{mì}}\textit{-tsȅt mì-mb{\'{a}}}~‘200’ (100*2), and \textsc{cl}2  for thousands – \textbf{\textit{{\`{o}}}}\textit{-t{\'{ɔ}}sìn {\`{o}}-b{\'{a}}} ‘2000’ (1000*2).

The model of formation that was masterly developed by Soga\il{Soga} has major relevance not only for the history of numerals in Niger-Congo, but for the theoretical analysis of the semantics of noun classes as well. The signifier of morphemes in noun class paradigms has a multilayer structure. This structure presumes that the semantics of each class can be defined through the paradigm at the intersection of four parameters: classificational, paradigmatic, syntagmatic and modal (for a more detailed discussion see \citealt{Pozdniakov2003}). It is useless to discuss the classificational aspect of noun class semantics in Soga numerals as we do when classes for humans, trees or animals are taken in consideration. The paradigmatic aspect of the signifier of the signs is the most relevant because the primary role is given to the correlation of classes in number, while some other paradigmatic correlations remain important as well. 

In conclusion, it should be noted that the noun class switch as a derivation mechanism is not limited to Benue-Congo and can be reconstructed at the Proto-Niger-Congo\il{Proto-Niger-Congo} level in at least one case (see chapter 4). 
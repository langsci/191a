\chapter{Analogical changes in numerals} \label{sec:2}
\section{Issues pertaining to the detection of alignments by analogy}

In addition to the grouping of numbers by noun class, a number of more radical strategies are used in the Niger-Congo languages. One of them is the formal alignment of numbers resulting from the diachronic alignment of forms by analogy. This strategy implies irregular phonetic changes in lexical stems. As a result, contiguous numerals in the Niger-Congo languages often have similar forms, that is they have common phonetic element(s). 

Such cases are not easily distinguishable from phonetic similarities conditioned by morphological changes, when affixes that are no longer productive blend into lexical roots, for instance, or archaic noun class markers exist in the numerals. Thus, in Wolof\il{Wolof}, as shown in the introduction, phonetic similarities arise in the numerals `2'–`4' (\textbf{\textit{ñ}}\textit{aar} ‘2’, \textbf{\textit{ñ}}\textit{ett} ‘3’, \textbf{\textit{ñ}}\textit{eent} ‘4’) as a result of inclusion of the noun class marker Ñ in the lexical roots. 

Only specialists of a concrete language can distinguish between morphological “accidents” and phonetic analogical changes, but sometimes even synchronic competence may not be enough. \tabref{tab:2:1} shows the first six numerals in five Adamawa languages.


\begin{table}
\caption{\label{tab:2:1}Adamawa examples}
\fittable{
\begin{tabular}{llllllll}
\lsptoprule
 & Languages & `1' & `2' & `3' & `4' & `5' & `6'\\
\midrule
(1)         & Tunya\il{Tunya} & sèlì & ari & ata & ana & aluni & nano\\
(2)         & Vere\il{Vere} & muo & ituko & tariko & nariko & gbanara & baburo\\
(3)         & Mom Jango\il{Mom Jango} & muzoz & itez & taz & naz & ɡbana & babez\\
(4)         & Dirrim\il{Dirrim} & nuan & bara & tara & nara & tona & tini\\
(5)         & Pere\il{Pere} & d{\`{ə}}{\={ə}} & {\={i}}r{\={o}} & t{\textsubtilde{a}}{\textsubtilde{\={a}}}r{\={o}} & n{\={a}}r{\={o}} & n{\'{u}}nn{\={o}} & nónd{\'{ə}}{\={ə}} (5+1)\\
\lspbottomrule
\end{tabular}
}
\end{table}


In Tunya\il{Tunya} (1) it is clear that the initial \textbf{a-} in the numerals ‘2’-‘5’ etymologically has the nature of the noun class marker. In Vere\il{Vere} (2) the final syllable \textbf{-ko} can hardly be considered a noun class marker, but it is very likely that we are dealing with a morpheme and not with a phonetic alignment of numerals. In Mom Jango\il{Mom Jango} (3) the final \textbf{-z} in ‘1’-‘4’ and ‘6’ is difficult to comment on; it is likely that this is an analogical change but its direction is not very clear. In Dirrim\il{Dirrim} (4) \textit{bara-tara-nara} is the case of analogical change and, considering the diachronic context, the numerals ‘2’ and ‘4’ were clustered together with ‘3’. In Pere\il{Pere}, the final -o in ‘2’-‘5’ may represent an analogical alignment or a morpheme.

Let us exclude all the cases of integration of noun class markers into stems and consider all the other cases of phonetic (or hidden morphological) clustering in the systems of numerals in Niger-Congo. We will deal mainly with two questions: 

\begin{enumerate}
\item  In which branches of Niger-Congo do analogical alignments have a major role and in which they are practically absent? This question is of crucial importance for the step-by-step reconstruction of numerals in Niger-Congo. 
\item Which numerals phonetically align together and which analogical groupings are rare? This question is important not only for the etymology of numerals but also for the typology of analogical changes in numerals. 
\end{enumerate}
The topic of the present chapter is not relevant to all the branches of Niger-Congo. For instance, in Bantu and Benue-Congo there is no systematic analogic phonetic alignment. But in some other branches it is impossible to discuss the etymology of numerals without considering this factor. In the twelve main branches of Niger-Congo the situation is as shown in \tabref{tab:2:2}. 

\begin{table}
\begin{tabularx}{.66\textwidth}{rXc}
\lsptoprule
& NC family & Analogy in numerals\\
\midrule
1 & Benue-Congo & –\\
2 & Mel & –\\
3 & Ijo & –\\
4 & Kru & –?\\
5 & Mande & –?\\
6 & Atlantic & +\\
7 & Kwa\il{Kwa} & +\\
8 & Adamawa & +\\
9 & Ubangi & +\\
10 & Gur & +\\
11 & Dogon & +\\
12 & Kordofanian & +\\
\lspbottomrule
\end{tabularx}
\caption{\label{tab:2:2}Analogic alignment in NC numerals}

\end{table}

In the first three branches the minus does not mean that there is no phonetic alignment of numerals. Some examples from Benue-Congo languages are given in \tabref{tab:2:3}. 

\begin{table}
\caption{\label{tab:2:3}BC examples of analogic alignments}
\begin{tabularx}{\textwidth}{lXXXXX}
\lsptoprule
Language & `1' & `2' & `3' & `4' & `5'  \\
\midrule
Gweno\il{Gweno} (E30) & -m\textbf{wi} & -\textbf{vi} & -tharu & -nya & -thwanu\\
Tiv\il{Tiv} & mòm' & h\textbf{ar} & -t\textbf{ar} & -nyin & -tan\\
Mmen\il{Mmen} & m{\`{ɔ}}ʔ & b\textbf{ege} & t\textbf{ege} & kaiko & ta\\
Bute\il{Bute} & mui & bam & tare\textbf{b} & nasi\textbf{b} & -gi\\
Kila\il{Kila} & mwe & han & t\textbf{ar} & n\textbf{ar} & tien\\
Mama\il{Mama} & moɁon & mari & tar\textbf{u} & la jin\textbf{u} & ton\textbf{u}\\
\lspbottomrule
\end{tabularx}
\end{table}

Each of these examples is interesting for the study of concrete languages, but these seem to be the only languages, among hundreds of BC languages, where analogical changes have been found; therefore, no systematic changes of this type for the BC family have been attested. 

In Mel there is only one case which is of interest to us, that is the unification of the initial root consonant in Krim\il{Krim}: \textit{yi-}\textbf{\textit{g}}\textit{in} ‘2’, \textit{yi-}\textbf{\textit{g}}\textit{a} ‘3’. The direction of analogical alignment in this case is not clear. It is impossible to study this particular case here, because the discussion of possible hypothesis would require a separate publication. It is important to underline that in other Mel languages cases of phonetic alignment of numerals have not been attested. 

There are virtually no unifications of this type in Kru, excluding the phonetic alignment of the initial consonant in ‘4’-‘5’, reported in \tabref{tab:2:4}.


\begin{table}
\caption{\label{tab:2:4}Kru alignments in `4'–'5'}

\begin{tabularx}{.66\textwidth}{Xllllr}
\lsptoprule
Language & `1' & `2' & `3' & `4' & `5'~~ \\
\midrule
Gbe\il{Gbe} & do & so & ta & hyi{\textasciitilde} & ~hm\\
Southern Grebo\il{Grebo} & do & so & ta & ha & *hm\\
Bassa\il{Bassa} & doo & so & ta & hiye & h{\'{m}}\\
\lspbottomrule
\end{tabularx}
\end{table}
I will dare to assume (based on these data) that the initial consonant in ‘4’ has undergone analogical change with the consonant in ‘5'. The final judgment should be done by specialists. In Ijo this type of alignment is absent. 

 
\section{Mande}%2.2

There are no systematic analogical changes in the systems of numerals in Mande languages.\footnote{I would like to thank Valentin Vydrin for a detailed discussion of the history of numerals in Mande languages.}  Some languages like Busa\il{Busa}, San\il{San} (South-Eastern branch) and Soninke\il{Soninke} (Western branch) present exceptional cases. 

In Busa\il{Busa}, we are probably dealing with the fossilized suffix \textbf{-h{\~{o}}} which can be found inside the lexical roots of ‘3’ and ‘4’: *\textit{a-h{\~{o}}} ‘3’, *\textit{si-h{\~{o}}} ‘4’, i.e. the phonetic similarity can be explained morphologically. 

In San\il{San}, apparently, the regular reflex of the three different consonants of proto-language of South-Eastern Mande is \textbf{s-} (see 3.10 below). Finally, three of the contiguous numerals start with the same consonant: \textit{so} ‘3’, \textit{si} ‘4’, \textit{soro} ‘5’.

Soninke\il{Soninke} represents a more complicated case, wherein the last vowel of each numeral is not distributed randomly (\tabref{tab:2:5}).

\begin{table}
\caption{\label{tab:2:5}Soninke}\il{Soninke}


\begin{tabularx}{.8\textwidth}{lXlX}
\lsptoprule

1 & ba(a)ne & 6 & tu(n)mu\\
2 & filo & 7 & nieru\\
3 & siko & 8 & segu\\
4 & (i-)nakato & 9 & kabu\\
5 & karago & 10 & ta(n)mu\\
\lspbottomrule
\end{tabularx}
\end{table}

In ‘1’ there is a particular vowel \textbf{-e}. “Minor” numerals (‘2’-‘5’) have the final \textbf{-o}, and all the higher numerals (‘6’-‘10’) – final \textbf{-u}.  Following the reconstruction of Nazam Halaoui \citep{Halaouï1990}: \textit{fill-a} ‘2’ (active voice) /\textit{fill-e} ‘2’ (passive voice) > \textit{fill-e-nu} (\textsc{pl})  ‘2’ > \textit{fill-o} (\textsc{pl}) ‘2’. In other words, in the numerals ‘2-5’ the vowel \textbf{-o} is interpreted by Halaouï as a phonetically conditioned allomorph of the plural morpheme \textbf{-nu}. But in the numerals ‘6-10’ another vowel was found, not \textbf{-o}, but \textbf{-u}. Nazam Halaouï explains this in the following way: irregular final vowel \textbf{-u} initially appeared in the numeral ‘6’ as a consequence of progressive assimilation (*\textit{tunm-o} > \textit{tunmu}), and then following the analogy this vowel appeared in numerals ‘7’-‘10’. Halaoui’s hypothesis is not plausible (it presupposes a doubtful phonetic change *\textbf{e-nu} > \textbf{-o} in the numerals ‘2’-‘5’), neither is it the only one possible. 

Valentin \citet[171-204]{Vydrin2006} shows that Soninke\il{Soninke} has two different plural suffixes, \textbf{-u/-o} and \textbf{-ni/-nu} (the allomorphs \textbf{-u} and \textbf{-o} are dialectal variants, the same is true for \textbf{-nu} and \textbf{-ni}). It is not quite clear, do we have the generic plural marker  \textbf{-u} in all the numerals from ‘6’ through ‘10’, or whether it is the alternative plural marker \textbf{-nu} that appears in ‘6’ and ‘10’, while the generic plural \textbf{-u} appears in ‘7’ through ‘9’. In any case, it is evident that in the right column of \tabref{tab:2:5}, the final \textbf{-u} is of morphological origin, rather than a result of an analogical change. The fact of the appearance of a plural marker in the numerals ‘6’-‘10’ by itself is noteworthy; these numerals should be interpreted as pluralia tanta. Interpretation of the final \textbf{-o} in ‘2’-‘5’ is much more problematic. There is a singular morpheme \textbf{-o} in Soninke, however, Vydrin's data do not clarify why it is \textbf{-o}, rather than \textbf{-e} or \textbf{-Ø}. Therefore, it can be conjectured that the final vowel of the numerals ‘2’-‘5’ result from analogical changes.

Now let us move to the branches where analogical changes are systematic. Even in these cases we will encounter different examples.

 
\section{Atlantic}%2.3

In \tabref{tab:2:6}, the data on the first five numerals in ten various Joola\il{Joola} languages will be compared. 


\begin{table}
\caption{\label{tab:2:6}Joola\il{Joola}}

\begin{tabularx}{\textwidth}{lXXXXX}
\lsptoprule

Joola\il{Joola} & `1' & `2' & `3' & `4' & `5' \\
\midrule
Joola\il{Joola} Karon\il{Karon} & ɔ-ɔnɔ(ɔ)l & supə\textbf{k} & həəciil & paa\textbf{k}ɩɩl & sa\textbf{k}\\
Bayot\il{Bayot} & ɛ-ndon & i-rigə\textbf{ʔ} & i-fiigi\textbf{ʔ} & i-βei\textbf{ʔ} & ɔ-r̥ɔ\textbf{ʔ}\\
Joola\il{Joola} Gusilay\il{Gusilay} & ya-nɔ & su-ruba & si-fe\textbf{gir} & si-ba\textbf{gir} & fu-tok\\
Joola\il{Joola} Banjal\il{Banjal} & a-nu & si-gaba & gu-fi\textbf{gir} & si-ba\textbf{gir} & fu-tok\\
Joola\il{Joola} Fogny\il{Fogny} & yɛ-kon & si-gaba & si-feg\textbf{ir} & si-bak\textbf{ir} & fu-tok\\
Joola\il{Joola} Mlomp\il{Mlomp} & yɔ-nɔɔ\textbf{l} & si-sube\textbf{l} & si-heji\textbf{l} & si-baci\textbf{l} & ŋa-suwaŋ\\
Joola\il{Joola} Kasa\il{Kasa} & ya-no(r) & si-lube & si-hej\textbf{i} & si-bak\textbf{i} & hu-tok\\
Joola\il{Joola} Ejamat\il{Ejamat} & a-yɩnka & ku-lube & si-heji & si-bacir & fu-tok\\
Joola\il{Joola} Kerak & ya-nɔr & si-sube & si-heji & si-bacir & hu-tok\\
\lspbottomrule
\end{tabularx}
\end{table}

In the last group, apparently, there is no reason for the establishing phonetic alignments. In the meantime, in the first two groups such alignments are evident. In the first group the velar consonant is spread, and in the second group, the liquid consonant; furthermore, the roots are mostly related. These are classical “symptoms” of analogical change. It is clear that it is useless to etymologize  the numerals without an in-depth analysis of these alignments.

Joola\il{Joola} languages form one of the four branches of the Bak group in Atlantic. In Bijogo\il{Bijogo}, there are no analogical changes in numerals. In the other two branches, these changes of various types can be found, and such changes differ from the type of changes in Joola. 

In Pepel\il{Pepel} (Manjak\il{Manjak} branch) in some sources the numerals ‘2’ and ‘3’ have a final \textbf{-s}, in other sources they have a final \textbf{-ʈ} and in \citet{Koelle1963} the final consonants are different, which can correspond to the situation in proto-language (\tabref{tab:2:7}).

\begin{table}
\caption{\label{tab:2:7}Pepel}\il{Pepel}

\begin{tabularx}{.66\textwidth}{llX}
\lsptoprule
`2' & `3' & Source\\
\midrule
puɡu\textbf{s} & \textstylefun{ŋa-jen}\textstylefun{\textbf{s}} & \citealt{Ndao2011}\\
pugu\textbf{ʈ} & waa-jin\textbf{ʈ} & \citealt{Wilson2007}\\
ge-pugu\textbf{s} & ga-ci\textbf{t} & \citealt{Koelle1963}\\
\lspbottomrule
\end{tabularx}
\end{table}

In the branch that is represented by isolated languages Balant\il{Balant} (Senegal; according to the data from \citealt{CreisselsBiaye2015}) for the numerals ‘2’ and ‘3’ the following forms exist  (\tabref{tab:2:8}).


\begin{table}
\caption{\label{tab:2:8}Balant\il{Balant}}

\begin{tabularx}{.5\textwidth}{Xl}
\lsptoprule

2 & 3\\
\midrule
CL-s{\`{ɩ}}b{\'{ɩ}} & CL-hàbí {\textasciitilde} CL-yàbí\\
s{\`{ɩ}}ɩb{\'{ɩ}} & yàabí\\
\lspbottomrule
\end{tabularx}
\end{table}
Apparently the numeral ‘2’ has undergone the analogical change following the numeral ‘3’. The sources on Balant\il{Balant} Kentohe\il{Kentohe} give different but also phonetically clustered forms: \textit{-sebm} ‘2’, \textit{-abm} ‘3’.

It is important to underline that analogical changes in the three aforementioned branches of Bak languages are not historically related – these changes are of different origin. This means that for this group, the principle of phonetic alignments of numerals is characteristic, but different types of changes by analogy co-exist. A similar situation is also typical of Northern Atlantic languages, which show other types of phonetic alignments. 

In Wolof\il{Wolof}, as previously mentioned, the alignment of the initial consonant in numerals ‘2’-‘4’ is of a morphological nature; these numerals maintain traces of the noun class prefix. Still, for native speakers these forms contain a similar phonetic marker that groups together the numerals for ‘2’-‘4’ and distinguishes them from other numerals. 

In Sereer\il{Sereer} (Northern Atlantic), as in Joola\il{Joola} (Bak Atlantic) the final velar can be clearly seen in the numerals ‘2’-‘5’: ƭ\textbf{ik} ‘2’, tad\textbf{ik} ‘3’, nah\textbf{ik} ‘4’, ƥet\textbf{ik} ‘5’. Here the clustering involves not only the final consonant but the precedent vowel as well, which creates an illusion of the existence of a specific morpheme (‘suffix’ \textbf{-ik}) used for marking the numerals ‘2’-‘5’. As will be demonstrated later, this is a false intuition. In Sereer, for example, we deal with morphophonology and not with morphology. Moreover, the coincidence with Joola is not casual and reflects an important phonetic innovation which took place in Proto-Atlantic\il{Proto-Atlantic}.  

In Nyun\il{Nyun} (the branch Nyun-Buy, Northern Atlantic languages) form clustering occurs through the final velar \textbf{-k} as well: -\textbf{n}du\textbf{k} ‘1’, -\textbf{n}a\textbf{k} ‘2’, -re-\textbf{n}e\textbf{k} ‘4’.  It is worth highlighting that the initial consonant of the aforementioned forms is also unified (\textbf{n}-).

The same isogloss can be encountered, although in its shorter version; in one of the five languages of the Cangin branch, that is in Palor\il{Palor}, ka-na\textbf{k} ‘2, ke-je\textbf{k} ‘3’. For Cangin this alignment is definitely marginal, in all the languages of Cangin branch another analogical change is encountered: the initial consonant in the numerals ‘1’-‘2’ is unified, which is a rare phenomenon. In Proto-Cangin we have \il{Proto-Cangin} \textit{*ji-} \textit{noʔ} ‘1’, \textit{*ka-nak} ‘2’ with the maintenance of the initial \textbf{n-} in all five languages (compare with the unifications in Nyun\il{Nyun}). 

The final \textbf{-n} is the basis for phonetic alignment in Sua\il{Sua}, though the affiliation to Atlantic languages has not been proven: sɔ\textbf{n} ‘1’, m-ce\textbf{n} ‘2’, b-rar ‘3’, m-na\textbf{n} ‘4’, sugu\textbf{n} ‘5’.

 
\section{Kwa%2.4
\il{Kwa}}

54 out of the 111 sources for Kwa\il{Kwa} languages available in our database show a common initial consonant \textbf{n-} for the numerals ‘4’ and ‘5’. For example, in Nzema\il{Nzema}: \textit{na} ‘4’, \textit{nu} ‘5’. In the other half of the sources forms with \textbf{n-} can be found for ‘4’ and with initial \textbf{t-} for ‘5’; for example, in Gbe\il{Gbe}-Fon\il{Fon}: \textit{e-ne} ‘4’, \textit{a-ton} ‘5’. The latter forms correspond to Proto-Bantu\il{Proto-Bantu} numerals:  \textit{*nàì} ‘4’, \textit{*táànò} ‘5’. The question then arises: where do the forms for ‘5’ with initial \textbf{n-} come from? 

Mary Esther Kropp Dakubu (\citealt{KroppDakubu2012}) includes the forms of the numeral ‘4’ in the series of correspondences which go back to \textbf{*n-}   and reflect as \textbf{n-} in all of the main branches of the family except for Ga\il{Ga}-Dangme\il{Dangme} (GD): Proto-Potou-Tano\il{Proto-Potou-Tano} *\textit{-n{\~{a}}}, Tano *-\textit{n{\~{a}}}, GTM (Ghana–Togo Mountain) *\textit{-in{\^{a}}}, Gbe\il{Gbe} \textit{e-ne}. The author includes the numeral ‘5’ in the series 15b where Akan\il{Akan} and GD both have \textbf{n}-, in Gbe \textbf{t}-, and inside GTM are both \textbf{t}- and \textbf{n}- (Na-Togo). Mary Esther Kropp Dakubu suggests the following historical interpretation of these forms: 

\begin{quote}
The fact that GTM is reconstructed with *\textbf{t}-, but its NA sub-group with *\textbf{n}, suggests that the \textbf{n} of Akan\il{Akan} and GD are also secondary, and that these forms are to be reconstructed as beginning in Kwa\il{Kwa} \textbf{*t} (ibid., p.24). 
\end{quote}


All the details of complex reconstruction will not be discussed here, but this shows that modern Kwa\il{Kwa} languages come from *PTB\il{PTB} (Proto-Potou-Tano\il{Proto-Potou-Tano}-Bantu). It is worth underlining that the reported reconstruction does not explain why in some of the Kwa languages the numeral ‘5’ with initial \textbf{*t-} has changed to \textbf{n-}. Furthermore, she does not explain why this irregular change has happened in the aforementioned languages and not in the others. 

The most natural answer to the first question is that in some languages, in the numeral ‘5’ the initial consonant has undergone analogical change with the numeral ‘4’. As a result, the same consonant was formed in both numerals. 

In order to answer the second question, it is necessary to observe the distribution of forms of ‘4’ and ‘5’ in different branches of Kwa\il{Kwa}, adding up in case of necessity forms for ‘3’ and ‘2’. In order to extend the analysis of Mary Esther Kropp Dakubu, the Lagoon languages will be added to her database (\tabref{tab:2:9}). 


\begin{table}
\caption{\label{tab:2:9}Akan\il{Akan}}

\begin{tabularx}{.8\textwidth}{Xllll}
\lsptoprule

Languages & `2' & `3' & `4' & `5' \\
\midrule
Akan\_\il{Akan}Twi\il{Twi} & abie-n & abie-sa & anan & anum\\
Ashanti\il{Ashanti} & mie-n{\~{u}} & mie-s{\~{a}} & en{\~{a}}n & en{\~{u}}m\\
Abron\il{Abron} 1 & mie-nu & mie-sa & nain & num\\
Abron\il{Abron} 2 & mie-nu\textbf{k} & mie-nza\textbf{k} & n-nai & n-num\\
\lspbottomrule
\end{tabularx}
\end{table}

In all the Akan\il{Akan} languages the alignment can be observed not only in ‘4’-‘5’ but (probably morphologically) also in numerals ‘2’-‘3’ (this phenomenon cannot be found outside this cluster). Furthermore, one of the sources clearly indicates a final velar in Abron\il{Abron}. \tabref{tab:2:10} reports data on the main languages of Central Tano.  

\begin{table}
\caption{\label{tab:2:10}Central Tano}


\begin{tabularx}{.8\textwidth}{Xllll}
\lsptoprule

Language & `2' & `3' & `4' & `5' \\
\midrule
Agni (\il{Agni}Anyin) & {\H{ɲ}}-ɲua & n-sa & n-na & n-nu\\
Baule\il{Baule} & nɲo & sa & na & n{\~{u}}\\
Nzema\footnotemark{}\il{Nzema} & ɲ-ɲu & n-sa & n-na & n-nu\\
Anufo\il{Anufo} & ɲɲo & nza & na & nu\\
Baule\il{Baule} (Baoulé)\footnotemark{} & nɲon & san & nan & nun\\
Ahanta\footnotemark{}\il{Ahanta} & ayin & asan & anla & enlu\\
\lspbottomrule
\end{tabularx}
\end{table}
\addtocounter{footnote}{-3}
\stepcounter{footnote}\footnotetext{ One of the sources on Nzema\il{Nzema} gives forms without an initial nasal: \textit{sa} ‘3’, \textit{da} ‘4’, \textit{du} ‘5’. Let us note that even in this case the initial consonant is the same in the numerals `4' and `5'.} 
\stepcounter{footnote}\footnotetext{ In some sources Baule\il{Baule} numerals ‘2’-‘5’ include also a final \textbf{-n}.}
\stepcounter{footnote}\footnotetext{ Thus, in Ahanta\il{Ahanta} the alignment of initial consonants for ‘4’-‘5’ is even more clear: \textbf{nl-}.}
Nearly identical forms are found in the other three branches of Tano (\tabref{tab:2:11}).

\begin{table}
\caption{\label{tab:2:11}Krobu\il{Krobu}-Ega\il{Ega}, Western Tano, Tano Guang\il{Guang}}


\begin{tabularx}{.8\textwidth}{Xllll}
\lsptoprule

Branch & Language & `3' & `4' & `5' \\
\midrule
Tano: Krobu-\il{Krobu}Ega\il{Ega} & Krobu\il{Krobu} & n-sa & n-na & n-nu\\
Tano West & Abure\il{Abure} & ŋ-ŋa & n-na & n-nu\\
Tano West & Eotile\il{Eotile} (Beti) & a-ha & a-ni & a-nu\\
Tano Guang\il{Guang} & Dwang\il{Dwang} (Bekye)\footnotemark{} & a-sa & a-na & a-nu\\
Tano Guang\il{Guang} & Ginyanga\il{Ginyanga} & i-sa & i-na & i-noun\\
Tano Guang\il{Guang} & Foodo\il{Foodo} & sa & naŋ & nu/nuŋ\\
Tano Guang\il{Guang} & Larteh\il{Larteh} & sa & ne & nu\\
Tano Guang\il{Guang} & Cherepon\il{Cherepon} & i-sa & i-ne & i-ni\\
\lspbottomrule
\end{tabularx}
\end{table}
\footnotetext{ The roots \textit{-na} and \textit{-nu} (for ‘4’ and ‘5’ respectively) can also be found in the Guang\il{Guang} group in Awutu\il{Awutu}, Chumburung\il{Chumburung}, Guang, Kplang\il{Kplang}, Krache\il{Krache}, Nawuri\il{Nawuri}, Nchumburu\il{Nchumburu}, Nkonya\il{Nkonya}. For the subsequent exposition it is important that in all these languages the numeral ‘3’ includes an initial \textbf{s-}.}
Among the numerous Tano languages there is just one language in our database which does not have initial \textbf{n-} in ‘4’ and ‘5’. This language is Ega\il{Ega}, which is misleadingly put in the sub-group with Krobu\il{Krobu}; its attribution to Tano is also doubtful, according to the majority of specialists. The forms of these numerals provide one more argument against this grouping. 

Some other languages display unification of the initial consonant in ‘4‘-‘5’ outside of the Tano group. 

As for Potou, forms with the initial \textbf{n-} in both ‘4’ and ‘5’: \textit{ne-ni} ‘4’, \textit{ne-na} ‘5’ were found only in Mbato\il{Mbato}, see \tabref{tab:2:12}.

\begin{table}
\caption{\label{tab:2:12}Potou}


\begin{tabularx}{.75\textwidth}{Xlll}
\lsptoprule

Language & `3' & `4' & `5' \\
\midrule
Mbato\il{Mbato} & ne-je & ne-ni & ne-na\\
Ebrie\il{Ebrie} & bwa-dya & bwe-di & mwa-na\\
\lspbottomrule
\end{tabularx}
\end{table}
Examples from Mbato\il{Mbato} permit us to reconstruct the unification of the initial consonant in ‘4‘-‘5’ in Potou-Tano. Outside of Potou-Tano this unification, following Mary Esther Kropp Dakubu, was found only in some languages of Na-Togo (GTM). The numerals in the languages of this group are represented in \tabref{tab:2:13}.

\begin{table}
\caption{\label{tab:2:13}Na-Togo}

\begin{tabularx}{.75\textwidth}{lXlll}
\lsptoprule
& Language & `3' & `4' & `5'\\
\midrule 
(1)         & Anii\il{Anii} & i-riu & i-naŋ & ~i-nuŋ\\
(2)         & Logba\il{Logba} & i-ta & i-na & i-nu\\
(3)         & Selee\il{Selee} & o-tie & o-na & o-no\\
(4)         & Sekpele\il{Sekpele} & cye & na & no\\
(5)         & Lelemi\il{Lelemi} & i-ti & i-ne & i-lo\\
(6)         & Siwu\il{Siwu} (Akpafu) & i-te & i-na & i-ru\\
(7)         & Adele\il{Adele} & a-si & i-na & ton\\
\lspbottomrule
\end{tabularx}
\end{table}

In languages (1-4) \textbf{n-} appears in ‘4’–‘5’ (Anii\il{Anii} displays an utmost variant of alignment with the unification of the final consonant as well). In language (7) the most ancient proto-language initial \textbf{t-} is attested in ‘5’, and this means that a reconstruction of \textit{*n-} in `5' for Proto-Na-Togo\il{Proto-Na-Togo} is problematic. Furthermore, in languages (5-6) there is no alignment of the forms.  

  
In other Kwa\il{Kwa} languages consonants in ‘4’ and ‘5’ differ. To be more precise, in Adjoukrou initial consonants are aligned but they are not nasals: \textit{jar} ‘4’, \textit{jen} ‘5’.

All the other forms can be grouped into four main types: 

\begin{enumerate}
\item the “basic” type, where, as in Bantu-Kwa\il{Kwa}, there is \textbf{n-} in `4' and \textbf{t-} in `5'; 
\item the type where ‘4’ has initial \textbf{n-} while ‘5’ shows a phonetic change of the initial consonant;
\item the type where ‘5’  keeps \textbf{t-,} while  ‘4’ shows a phonetic deviation;
\item the most complicated type for the analogical interpretation which has \textbf{n-} only in ‘5’ while ‘4’ has a non-nasal initial consonant. 
\end{enumerate}
I will provide some examples followed by interpretations. 

Type 1 is illustrated in (\tabref{tab:2:14}).

\begin{table}
\caption{\label{tab:2:14} n- `4', t- `5' (t- `3')}


\begin{tabularx}{.75\textwidth}{Xllll}
\lsptoprule

Group & Language & `3' & `4' & `5'\\
\midrule
Gbe\il{Gbe} & Aja\il{Aja} & e-to & e-ne & a-to\\
Gbe\il{Gbe} & Ewe\il{Ewe} & e-to & e-ne & a-to\\
Gbe\il{Gbe} & Gen\il{Gen} & e-to & e-ni & a-to \\
Gbe\il{Gbe} & Fon\il{Fon} & a-to & e-ne & a-to\\
Gbe\il{Gbe} & Kotafon\il{Kotafon} & a-to & e-ni & a-to\\
Gbe\il{Gbe} & Saxwe\il{Saxwe} & a-to & i-ne & a-tu\\
Gbe\il{Gbe} & Xwla\il{Xwla} & a-to & e-ne & a-to\\
GTM & Kebu\il{Kebu} & ta & nia & to\\
Ga-\il{Ga}Dangme\il{Dangme} & Dangme\il{Dangme} & e-to & e-ne & a-to\\
Ka-Togo & Akebu\il{Akebu} & ta & nie & tu\\
Ka-Togo & ~Ikposo-\il{Ikposo}Uwi & i-la & i-na & i-tu\\
Na-Togo & Adele\il{Adele} & a-si & i-na & ton\\
\lspbottomrule
\end{tabularx}
\end{table}

It is clear that the basic etymological forms are represented extensively. They are not confined to Potou-Tano or the Lagoon languages but they can be found in four other branches of Kwa\il{Kwa} as well. 

Type 2 is illustrated in (\tabref{tab:2:15}).

\begin{table}
\caption{\label{tab:2:15}n- `4', phonetic deviations in `5'} 


\begin{tabularx}{.75\textwidth}{XXlll}
\lsptoprule

Group & Language & `3' & `4' & `5' \\
\midrule
Ka-Togo & Avatime\il{Avatime} & o-ta & o-ne & o-cu\\
Ka-Togo & Tuwuli\footnotemark{}\il{Tuwuli} & ɛ-lalɛ & ɛ-na & e-lo\\
Na-Togo & Lelemi\il{Lelemi} & i-ti & i-ne & i-lo\\
Na-Togo & Siwu\il{Siwu} (Akpafu~) & it-e & i-na & i-ru\\
Lagoon & Avikam\il{Avikam} & a-za & a-na & a-ɲu\\
\lspbottomrule
\end{tabularx}
\end{table}
\footnotetext{\citet[155]{Harley2005}
% Harley, Matthew W.  2005: 155: 
“With the exception of mɔa – ‘one’ and nvi{\~{a}} – ‘two’, the citation forms of these numerals are derived using the expletive third person pronoun ke, which has become incorporated into the attributive numeral : ke ɛlalɛ ‘3’  > kaalɛ, ke ɛna ‘4’ > kɛna …”.}
Type 2, like Type 1, is not difficult to interpret. In the single languages the reflexes of the original consonant are maintained in ‘4’, while in ‘5’ *\textbf{t-} undergoes phonetic changes. 

Type 3 is illustrated in (\tabref{tab:2:16}).

\begin{table}
\caption{\label{tab:2:16}t- `5',  phonetic deviations  in `4'}
\begin{tabularx}{.75\textwidth}{XXlll}
\lsptoprule
Group & Language & `3' & `4' & `5' \\
\midrule
Ka-Togo & Igo\il{Igo} (Ahlon) & i-ta & a-la & u-to\\
Ka-Togo & Nyangbo\il{Nyangbo} & e-tae & e-le & e-tie \\
\lspbottomrule
\end{tabularx}
\end{table}

The proto-language consonant is maintained in only two languages in ‘5’ (Ka-Togo and GTM) while the initial consonant in ‘4’ undergoes regular phonetic change. 

And finally, the most difficult type 4 is illustrated in (\tabref{tab:2:17}).

\begin{table}
\caption{\label{tab:2:17}n- in `5' but not in `4'}


\begin{tabularx}{\textwidth}{llXXl}
\lsptoprule

Group & Language & `3' & `4' & `5' \\
\midrule
Potou & Ebrie\il{Ebrie} & bwa-dya & bwe-di & mwa-na\\
Potou & Gã & e-t{\~{e}} & e-jwe & e-n{\~{u}}mo\\
Lagoon & Abé(Abbey)\il{Abbey} & a-ri & a-le & u-ni\\
Lagoon & Abiji\il{Abiji} & e {\textprimstress}-ti & a {\textprimstress}-la & e {\textprimstress}-ne\\
Nyo? & Ari\il{Ari} (Abiji)\il{Abiji} & e-ti & a-la & e-ni\\
Central Tano & Ahanta\il{Ahanta} & a-sa & a-la & e-n{\~{u}}\\
Ga-\il{Ga}Dangme\il{Dangme} & Dangme\il{Dangme} & e-te & e-ywi/e-wi & e-nuo\\
Lagoon & Alladian\il{Alladian} & a-o & a-zo & e-nri\\
Lagoon & Adioukrou\il{Adioukrou} & ɲa-hn & ya-r & ye-n\\
\lspbottomrule
\end{tabularx}
\end{table}

Here we see all the counter-examples against the hypothesis on the change *\textbf{t-} > \textbf{n-} in ‘5’ as analogous to \textbf{n-} in ‘4’. The solution is to imagine that in certain languages belonging to different branches of Kwa\il{Kwa} (independently from each other), firstly, this analogical change occurred, the original *\textbf{n-}, which was the basis of the analogical change, but was then lost in the numeral ‘4’. 

Finally, let us get back to the question raised above: why does analogical change in ‘5’ take place in only some Kwa\il{Kwa} languages? Let us have a look at \tabref{tab:2:18}, where different initial root consonants in numerals ‘3’-‘5’ within different groups of Kwa are presented. 


\begin{table}
\caption{\label{tab:2:18}Kwa\il{Kwa} initial consonants in `3'-'5'}

\fittable{
\begin{tabular}{lllllllll}
\lsptoprule

\textbf{Group} & Bantu-Kwa\il{Kwa} & Tano & Tano & Tano & Tano & Gbe\il{Gbe} & GD & GTM\\
\textbf{Sub-Group} &  & Krobu\il{Krobu} & Central Tano & Akan\il{Akan} & Guang\il{Guang} & Gbe\il{Gbe} & Gan-Dangme\il{Dangme} & Ka-Togo\\
\midrule
‘3' & \textbf{*t} & \textbf{s} & \textbf{s} & \textbf{s} & \textbf{s} & \textbf{t} & \textbf{t} & \textbf{t}\\
‘4' & *n & n & n & n & n & n & j/y & n\\
‘5' & \textbf{*t} & \textbf{n} & \textbf{n} & \textbf{n} & \textbf{n} & \textbf{t} & \textbf{t} & \textbf{t}\\
\lspbottomrule
\end{tabular}
}
\end{table}

In the Kwa\il{Kwa} languages we see a clear tendency: in languages with the initial plosive *\textbf{t-} > fricative \textbf{s-}, the described analogical changes can be found. Where the plosive is maintained, this change is more difficult and can be found in only some of the languages (for example, some of the above-mentioned Na-Togo cases). In this case we have not *t- > n- ‘5’, but *t- > s- > n. This observation can be interesting as a candidate for analogical changes – maybe, ‘weak’ consonants (for example, fricatives) can be more easily involved in analogical processes than ‘strong’ ones (plosives). 

It is curious that this analogical isogloss can be found in a number of other branches of Niger-Congo, including Adamawa, Gur and Dogon (as well as Seenku\il{Seenku} from the Mande family). 

 
\section{Adamawa}%2.5

In Adamawa the above-mentioned analogical change can be found in at least a dozen of languages (\tabref{tab:2:19}).

\begin{table}
\caption{\label{tab:2:19}Initial n- in `4'-'5' in Adamawa languages}


\begin{tabularx}{\textwidth}{lXlXXXX}
\lsptoprule

Language & `2' & `3' & `4' & `5' & `6' & `7' \\
\midrule
Tula\il{Tula} & rop & ta & na & nu & ~ & ~\\
Kwa\il{Kwa} & neɡbe & ne mwan & ne nat & ne nu & ~ & ~\\
Burak\il{Burak} & rab & gbunuŋ & net & nob & ~ & ~\\
Chamba\il{Chamba} & bara & te-ra- & nasa & tu-na- & ~ & ~\\
Kolbila\il{Kolbila} & inu & tonu & nereb & nunub &  & \\
Bangunji\il{Bangunji} & yob & t\textbf{ar} & \textbf{nar} & \textbf{n}uŋ & ~ & ~\\
Yendang\il{Yendang} & ini & t\textbf{at} & \textbf{nat} & ɡhi-\textbf{n}an & ~ & ~\\
Dadiya\il{Dadiya} & yo & t\textbf{al} & \textbf{nal} & \textbf{n}u & ~ & ~\\
Peere\il{Peere} & i\textbf{ro} & t\textbf{aro} & \textbf{naro} & \textbf{n}uno & ~ & ~\\
Samba Leko\il{Samba Leko} & ki\textbf{ra}{\textasciitilde}ki\textbf{re} & tu\textbf{re} & \textbf{n}a\textbf{ra} & \textbf{n}unak & ~ & ~\\
Gimme\il{Gimme} & idtiɡe & taɡe & naɡe & noniɡe & nonɡe~ & nokidtiɡe~~\\
\lspbottomrule
\end{tabularx}
\end{table}

However, in Adamawa, analogies are much more widespread than in Kwa\il{Kwa}. For instance, in Gimme\il{Gimme} the numerals ‘2’-‘7’ share the same final syllable (morpheme?). In Chamba\il{Chamba}, only one similarity can be found for ‘4’-‘5’ and for ‘2’-‘3’ (the final syllable \textbf{-ra}).  In Kolbila\il{Kolbila}, the situation is quite similar to the one in Chamba (‘2’-‘3’ share the same final syllable \textbf{-nu}) and in ‘4’-‘5’ both the initial \textbf{n-} and the final \textbf{-b} coincide. 

Phonetic alignment follows more interesting models in Bangunji\il{Bangunji}, Yendang\il{Yendang}, Dadiya\il{Dadiya}, Peere\il{Peere} and Samba Leko\il{Samba Leko}. In these languages, on the one hand, ‘4’-‘5’ are still grouped together (because of the initial consonant) and, on the other hand, (‘2’)-‘3’-‘4’ are also grouped (because of the final syllable). The numerals with the meaning ‘4’ have two simultaneously distinct features which mark two separate groupings. As a result, peculiar minimal pairs arise formed by contiguous numerals; for example, in Yendang: \textit{tat} \textit{–} \textit{nat} ‘3’-‘4’, \textit{nat} \textit{–} \textit{nan} ‘4’-‘5’. 

Another alignment of numerals (2), ‘3’-‘4’ takes place in Adamawa where there is no alignment in numerals ‘4’-‘5’. Minimal pairs like in Dirrim\il{Dirrim} \textit{bara} ‘2’ – \textit{tara} ‘3’ – \textit{nara} ‘4’ are a very widespread phenomenon for the languages within this family. Some examples are presented in the following table (\tabref{tab:2:20}).

\begin{table}
\caption{\label{tab:2:20}Adamawa analogical alignments in `3'-'4'}


\begin{tabularx}{\textwidth}{lXXXX}
\lsptoprule

Language & `1' & `2' & `3' & `4' \\
\midrule
Vere\il{Vere} (Mom Jango)\il{Mom Jango} &  & ituko & tariko & nariko\\
Galke\il{Galke} (Ndai) &  &  & ca-{?}a- & na{?}a\\
Dama\il{Dama} &  &  & sa-i & nai\\
Mono\il{Mono} &  &  & sai & nai\\
Mundang\il{Mundang} &  &  & sa-i & nai\\
Pam\il{Pam} &  &  & sa-i & nai\\
Fali\il{Fali} &  &  & tan & nan\\
Kam\il{Kam} &  &  & car & nar\\
Bali\il{Bali} &  &  & tat & nat\\
Kumba\il{Kumba} &  &  & sat & nat\\
Teme\il{Teme} &  &  & tat & nat\\
Waka\il{Waka} &  &  & tat & nat\\
Yendang\il{Yendang} &  &  & tat & nat\\
Wom\il{Wom} &  & ira & tara & nara\\
Taram\il{Taram} &  & bara & tara & nara\\
Fanya\il{Fanya} &  & liru & taro & naro\\
Duupa\il{Duupa} &  & ito & tato & nato\\
Kotopo\il{Kotopo} & wate & i-to & tato & nato\\
Mom Jango\il{Mom Jango} & muzoz & itez & taz & naz\\
\lspbottomrule
\end{tabularx}
\end{table}
This kind of assonance may seem insignificant, but I would like to underline once more that among hundreds of Benue-Congo languages, it is impossible to find any similar case. 

 
\section{Ubangi}%2.6

Ives \citet{Moñino1995} has reconstructed unified forms for ‘3’-‘4’ and partly for ‘5’ in Proto-Gbaya\il{Proto-Gbaya}. These forms resemble the above-mentioned “minimal pairs” in Adamawa. In Proto-Gbaya\il{Gbaya}: \textit{*tar(a)} ‘3’, \textit{*nar(a)} ‘4’, \textit{*mor} ‘5’ (notably, the numeral ‘5’ coincides with the word ‘hand’). In Ubangi-Sere\il{Sere}, a different type of alignment can be found – the final \textbf{-o} in numerals ‘2’-‘5’ (in Ubangi-Zande\il{Zande} – the final \textbf{-i}) (\tabref{tab:2:21}).

\begin{table}
\caption{\label{tab:2:21}Final vowel alignments in Ubangi}


\begin{tabularx}{\textwidth}{XXXXl}
\lsptoprule

Language & `2' & `3' & `4' & `5' \\
\midrule
Ndogo\il{Ndogo} & so & tao & nao & vo\\
Sere\il{Sere} & so & tao & nao & vo\\
Tagbu\il{Tagbu} & so & tao & nao & vuo\\
Pambia\il{Pambia} & a-vai & wa-tai & (h)avai & boinyaci\\
\lspbottomrule
\end{tabularx}
\end{table}
 
\section{Gur} %2.7

In some languages of the Gur family analogical changes in ‘4’-‘5’ can be found, as observed in Kwa\il{Kwa} and Adamawa (\tabref{tab:2:21}).


\begin{table}
\caption{\label{tab:2:22}Gur initial n- in `4'-'5'} 

\begin{tabularx}{\textwidth}{XXXXl}
\lsptoprule

Language & `2' & `3' & `4' & `5' \\
\midrule
Baatonum\il{Baatonum} & yiru & ita / yita & ne & nobu\\
Chala\il{Chala} (dial.) & -la & -toro & -nara & -nuŋ\\
Buli\il{Buli} & ba-yi & ba-ta & ba-nasi & ba-nu\\
Dagaara\il{Dagaara} & ayi & ata & a-nar & a-nu\\
Delo\il{Delo} & ala & atoro & a-nara & a-noŋ\\
Ditammari\il{Ditammari} & deni & tati / tadi & na & numu\\
Nawdm\il{Nawdm} & mrek & mtak & m-na & m-nu\\
Safaliba\il{Safaliba} & ayik & atak & anaasi & anu\\
\lspbottomrule
\end{tabularx}
\end{table}

Like in Chamba\il{Chamba} (Adamawa), some of the Gur languages have a common feature not only for ‘4’-‘5’ but also for ‘2’-‘3’. For instance, in Nawdm\il{Nawdm} and Safaliba\il{Safaliba}, as can be deduced from \tabref{tab:2:22}, the numerals ‘2’-‘3’ have a final velar consonant. The final velar can be found in ‘2’-‘3’ in Hanga\il{Hanga} (\textit{a-yik} ‘2’, \textit{a-tak} ‘3’), and in Dogose\il{Dogose} it is found in ‘2’-‘5’: \textit{i-yok} ‘2’, \textit{i-sak} ‘3’, \textit{i-yi̬k} ‘4’, \textit{i-wak} ‘5’. Gudrun Miehe (\citealt{MieheEtAl2007}: 157) shows in Khisa\il{Khisa} (Komono) the final -\textbf{Ɂ} in ‘2’-‘5’: \textit{ɲ{\'{ɔ}}{\`{ɔ}}ʔ} ‘2’, \textit{s{\textsubtilde{á}}aʔ} ‘3’, \textit{ɲéèʔ} ‘4’, \textit{ŋwáàʔ} ’5’.

And finally I would like to report a rare case of strong alignment between the numerals ‘1’ and ‘2’ in Mbelime\il{Mbelime}: \textit{y{\~{ɛ}}nde} ‘1’, \textit{yede} ‘2’.

 
\section{Dogon}%2.8

Assimilation of the initial consonant in ‘5’ to the initial consonant \textbf{n-} in ‘4’ (for example, Tommo So\il{Tommo So}: \textit{nay} ‘4’, \textit{no} ‘5’) is characteristic of practically all the Dogon languages and should be reconstructed already for the Proto-Dogon\il{Proto-Dogon}. Other types of unification cannot be found in this family. 

 
\section{Kordofanian}%2.9

Phonetic / morphological alignments in this family are quite rare. In what follows, the most interesting cases are reported (\tabref{tab:2:23}).

\begin{table}
\caption{\label{tab:2:23}Kordofanian alignments}

\begin{tabularx}{\textwidth}{lXXXXXl}
\lsptoprule
Group & Languages & `1' & `2' & `3' & `4' & `5' \\
\midrule
Talodi\il{Talodi} & Tocho\il{Tocho} & pulu\textbf{k} & we-ra\textbf{k} & wa-ta\textbf{k} &  & \\
Talodi\il{Talodi} & Jomang\il{Jomang}
% (Talodi)\il{Talodi}
& y-ílli\textbf{k} & y-ilra\textbf{k} & y-ida\textbf{k} & ~ & ~\\
Talodi\il{Talodi} & Nding\il{Nding} &  & -eta & t-ata\textbf{k} & -ibiɲi\textbf{k} & \\
Talodi\il{Talodi} & Tegem\il{Tegem} & tléedi & \textbf{pad}er\textbf{ig} & \textbf{pad}a\textbf{ig} &  & \\
Katla\il{Katla} & Katla\il{Katla} & te:tá\textbf{k} & se\textbf{k} &  &  & \\
Orig\il{Orig} & Orig\il{Orig} &  &  &  & aru\textbf{m} & wura\textbf{m}\\
Katla\il{Katla} & Tima\il{Tima} &  & \textbf{eh}ek & \textbf{eh}oat & \textbf{eh}alam & \\
\lspbottomrule
\end{tabularx}
\end{table}

In Talodi\il{Talodi} the final velar is present, similarly to other branches of Niger-Congo. Some cases of phonetic alignment can be found, though this alignment is reserved to singular languages rather than to the whole family. 

In sum, the data examined in this chapter can be found in Appendix C where 50 different cases of probable analogical changes in Niger-Congo are highlighted. The Table in Appendix C permits the evaluation of the scale of analogical changes in the system of numerals in Niger-Congo in general. 

It is worth mentioning that in the cases where numerals ‘6’-‘10’ are not derived, it is very unusual to find phonetic alignment in them (exceptional systems, such as that of Soninke\il{Soninke}, were previously discussed). For this reason, only the numerals ‘1’-‘5’ are included in Appendix C. Three main questions are to be answered concerning these numerals: 1) Which groupings of numerals are most typical for the Niger-Congo languages when we deal with analogical changes? 2) Which phonetic (or hidden morphological) means are used to produce the alignment of numerals? 3) Are there any reasons to consider that similar analogical changes in different branches of Niger-Congo can be diachronically related? Otherwise, can these materials be useful for the study of other isoglosses in Niger-Congo?  

As demonstrated in Appendix C, mostly contiguous numerals are aligned (see some rare examples above, for example in Nyun\il{Nyun} languages, where features for ‘1’–‘2’/‘4’ are shared, but not for `3'). 

It is quite rare that ‘1’ shares a submorphemic marker with the numeral ‘2’, while for other contiguous numerals this is more common.  Such rare examples are found in Ha\il{Ha} (Bantu J) and in Mbelime\il{Mbelime} (Gur). In both languages the forms of numerals ‘1’ and ‘2’ have minimal phonetic difference. As will be demonstrated in the following sections dealing with the etymology of numerals ‘1’ and ‘2’, the forms in Ha (\textit{mbele} ‘1’, \textit{bhili} ‘2’) are of great interest for the diachronic interpretation of numerals.

As can be seen in Appendix C, the final phonemes have phonetic alignment much more often than the initial ones. 

The appearance of the diachronically irregular initial \textbf{n-} in the numeral ‘5’ as analogous to the regular form of the numeral ‘4’ represents a common feature in different families of Niger-Congo: Potou-Tano (Kwa\il{Kwa}), Adamawa, Gur and Dogon. More attention should be paid to this phenomenon because it is unlikely that one analogical feature could appear in four different branches of Niger-Congo independently. 

There are two remarkable cases in the alignment of final phonemes which are typical for several branches of Niger-Congo. 

Firstly, there is the appearance of a final velar (\textbf{-k}) in the groupings of the numerals ‘2’-‘5’, ‘2’-‘4’, ‘2’-‘3’, ‘3’-‘4’ (in Kordofanian and Atlantic also ‘1’-‘2’-(‘3’)).  This feature is typical for the Atlantic, Adamawa, Gur and Kordofanian groups (thus, one more common feature can be found for Adamawa-Gur). In Benue-Congo and Mande the reported examples are clearly marginal. 

Secondly, similarly to the regular dental reflexes of the final consonant in the numeral ‘3’ (*\textbf{-t(h)}), in ‘4’ the final consonant undergoes an irregular change (non dental consonant becomes dental). This type of change is particularly characteristic for Atlantic, Adamawa and Gbaya\il{Gbaya} (Ubangi), but it is also found in Kordofanian and in Benue-Congo, which do not have analogic changes as characteristic features. 

The most common case is the appearance of the identical final vowel in some languages of different families (mostly in numerals ‘2’-‘5’): Mama\il{Mama} (Bantoid), Soninke\il{Soninke} (Mande), Peere\il{Peere} (Adamawa) and Ndogo\il{Ndogo}, Pambia\il{Pambia} (Ubangi).

All the reported cases should be taken into consideration for the process of etymologization of numerals, which will be done in the following chapter. 


\author{Konstantin Pozdniakov}
\title{The numeral system of Proto-Niger-Congo}
\subtitle{A step-by-step reconstruction} 
\renewcommand{\lsSeries}{nc}
\renewcommand{\lsSeriesNumber}{2}
\renewcommand{\lsID}{191}
\BackBody{This book proposes the reconstruction of the Proto-Niger-Congo numeral system. The emphasis is placed on providing an exhaustive account of the distribution of forms by families, groups, and branches. The big data bases used for this purpose open prospects for both working with the distribution of words that do exist and with the distribution of gaps in postulated cognates.  The distribution of filled cells and gaps is a useful tool for reconstruction.

Following an introduction in the first chapter, the second chapter of this book is devoted to the study of various uses of noun class markers in numeral terms. 
The third chapter deals with the alignment by analogy in numeral systems. 
Chapter 4 offers a step-by-step reconstruction of number systems of the proto-languages underlying each of the twelve major NC families, on the basis of the step-by-step-reconstruction of numerals within each family. 
Chapter 5 deals with the reconstruction of the Proto-Niger-Congo numeral system on the basis of the step-by-step-reconstructions offered in Chapter 4. 
Chapter 6 traces the history of the numerals of Proto-Niger-Congo, reconstructed in Chapter 5, in each individual family of languages.
}
\dedication{Ирине Поздняковой}
\BookDOI{10.5281/zenodo.1311704}
\renewcommand{\lsISBNdigital}{978-3-96110-098-9}
\renewcommand{\lsISBNhardcover}{978-3-96110-099-6}	
\typesetter{Sebastian Nordhoff}
\proofreader{
Ahmet Bilal Özdemir,
Alena Wwitzlack-Makarevich,
Amir Ghorbanpour,
Aniefon Daniel,
Brett Reynolds,
Eitan Grossman,
Ezekiel Bolaji,
Jeroen van de Weijer,
Jonathan Brindle,
Jean Nitzke,
Lynell Zogbo,
Rosetta Berger,
Valentin Vydrin
}


